\documentclass[a4paper]{article}

\usepackage{url}
\usepackage{enumitem}
\usepackage{setspace}
\usepackage[hang]{footmisc}
\usepackage{fontspec}
\usepackage{polyglossia}
\usepackage{titlesec}
\usepackage{xcolor}

\usepackage[
  bookmarks=true,
  colorlinks=true,
  linkcolor=linkcolor,
  urlcolor=linkcolor,
  citecolor=linkcolor,
  pdftitle={الخط الأميري},
  pdfsubject={توثيق خط المتون النسخي، الخط الأميري},
  pdfauthor={خالد حسني},
  pdfkeywords={خط, عربي, مطبعة, أميرية, أميري, يونيكود, أوبن تيب}
  ]{hyperref}

\definecolor{textcolor}  {rgb}{.25,.25,.25}
\definecolor{pagecolor}  {rgb}{1.0,.99,.97}
\definecolor{titlecolor} {rgb}{.67,.00,.05}
\definecolor{linkcolor}  {rgb}{.80,.00,.05}
\definecolor{codecolor}  {rgb}{.90,.90,.90}

\setmainlanguage {arabic}
\setotherlanguage{english}
\rightfootnoterule

\setmainfont               [Path=./generated/,Ligatures=TeX]                     {jomhuria.ttf}
\setmonofont               [Scale=MatchLowercase]              {DejaVu Sans Mono}
\newfontfamily\arabicfont  [Path=./generated/,Script=Arabic,Numbers=Proportional]{jomhuria.ttf}
\newfontfamily\arabicfonttt[Script=Arabic,Scale=MatchLowercase]{DejaVu Sans Mono}

\newcommand\addff[1]{\addfontfeature{RawFeature={#1}}} % add feature
\newcommand\addfl[1]{\addff{language=#1}}              % add language

\setlength{\parindent}{0pt}
\setlength{\parskip}{1em plus .2em minus .1em}
%setlength{\emergencystretch}{3em}  % prevent overfull lines
\setcounter{secnumdepth}{0}

\newfontfamily\titlefont[Path=./generated/,Script=Arabic]{jomhuria.ttf}

\titleformat*{\section}{\Large\titlefont\color{titlecolor}}
\titleformat*{\subsection}{\large\titlefont\color{titlecolor}}
\titleformat*{\subsubsection}{\itshape\titlefont\color{titlecolor}}

\titlespacing{\section}{0pt}{*4}{*1}
\titlespacing{\subsection}{0pt}{*3}{0pt}
\titlespacing{\subsubsection}{0pt}{*2}{0pt}

\renewcommand\U[1]{\colorbox{codecolor}{\texttt{U+#1}}}

\title{Jomhuria Test-texts 2}

\begin{document}
\pagecolor{pagecolor}
\color{textcolor}

\begin{english}\maketitle\end{english}
\newpage

\begin{flushright}

\setstretch{1.6}
\huge

رب یسر و لا تعسر رب تمم و کمل بالخیر امین

اپچذ اثخذ هوژ خظی گلمن شغقض قژشت

ثخذ ضظغلا قثثا قثپا گچ پژ گخ تژ

 ژژ زز وو ژوژ ووژ وژو

گثگ چپچ خثخ ثقگث ژپو ژثژ پن پو پژ کلک کل ککلک

گژپمپچژ گژثمثخژگوژمپچ مپخ همهمه ضچظغ ضخظغ شثگلچ

آپمثقهگثغثشظثخوضلغ

پپپپ ثثثث چچچ ششش ضضض ظظظ غغغ ففف ققق گگگ گگك للل ممم لا ثلا

۱۲۳۴۵۶۷۸۹۰

پا پپ پچ پذ پژ پش پض پظ پغ پف پق پك پگ پل پم پن پو پهہ پہہ پہ په پلا پی پے

ثا ثث ثخ ثذ ثژ ثش ثض ثظ ثغ ثف ثق ثك ثگ ثل ثم ثن ثو ثهہ ثہ ثه ثلا ثی ثے

چا چپ چچ چذ چژ چش چض چظ چغ چف چق چك چگ چل چم چن چو چهہ چہہ چہ چه چلا چی چے

شا شث شچ شذ شژ شش شض شظ شغ شف شق شك شگ شل شم شن شو شهہ شہ شه شلا شی شے

ضا ضپ ضچ ضذ ضژ ضش ضض ضظ ضغ ضف ضف ضك ضگ ضل ضم ضن ضو ضهہ ضہ ضه ضلا ضی ضے

ظا ظپ ظچ ظذ ظژ ظش ظض ظظ ظغ ظف ظف ظك ظگ ظل ظم ظن ظو ظهہ ظہ ظه ظلا ظی ظے

غا غپ غچ غذ غژ قش قض قظ غغ غف غق غك غگ غل غم غن غو غهہ غہ غه غلا غی غے

قا قب قج قذ قژ قش قض قظ قغ قف قق قك قگ قل قم قن قو قہ قهہ قه قلا قی قے

گا گب گخ گذ گژ گش گض گظ گغ گف گق گك گکك گک گل گم گن گن گو گہ گهہ گه گلا گی گے

لا لب لخ لذ لژ لش لض لظ لغ لف لق لك لکك لک لل لم لن لن لو لہ لهہ له للا لی لے

ما مب مچ مذ مژ مش مض مظ مغ مف مق مك مگ مل مم من مو مهہ مہ مه ملا می مے

ها هب هچ هذ هژ هش هض هض هظ هغ هف هق هك هگ هل هم هن هو هہهہ ههہ هہ ههه هه هلا هی هے

ژ ژ ژ وژ ژ ژژ ژو ژ ژژژوو و ژ وو ژ ژ وژ ژ ژ وژ ژ ژژژ ژ ووو ژ ژوژ ژ ژ ژژوژ ژو ژو ژ ژژو ژ ژو ژژووو ژ و و و وژ ژ و و وو و ژ ژژ وووژ و ژژ ژوژژ ووژ ژ ژ وژ ژ و ژ و ژژ ژژ ژ و ژ ژژ ژژ ژوژ و و ژ ژژ ژ ژ ژو ژژ ژ ژ و ژژژ ژ ژوژ ژو وووژژ و ژ و و و ژ ژ ژ ژ وژ ژو ژ و ژ ژژوژو وژ ژ ووژ ژ ژو ژژ ژ ژژژژژ ژ و ژ ژو و ژ ژ و ژ ژ ژو ژژژژ ژژ وژژوژ ژژ ژژ و ژ ژ ژ ژ ژژژ ژ ژژ و و ژ ژ ژ ژ ژ ژ ژژو ژژ ژژژ ژ و ژ وژ و وژ وو ژژژ ژ و و ژژژژو ژو ژ ژژژژ ژژ ژ ژ وو ژ ژ ژ ژ وژ ژ ژو ژ ژژ و ژژ ژ ژ ژژ ژ و ژ و وژژ ژوژ ووو ژ ژ و و و ژ ژ و و و وژو ژ ژژژ ژژو ژ و ژژژ ژ ووژوژ ژو ژ و ژ وژ ژو ژ و و ژ ژ و وژ ژژو ژژ ژ ژژژ ژ و ژژ ژ ژووو ژژو و و ژ وژ ژو و ژ و ژژ ژژ ژ ژ و ژ ژژ ژ ژ ژ ژ ژ ژ ژ ژ و و وو وژ ژ ژوژژ ژو ژ ژ وو وو و و ژ و ژ ژژ و ژ ژ ژ ژژ ژ ژو و ژ ژو و وژ و ژ ژ ژ ژ ژ وو ژ ژ ژو وژ ژوژوژ ژ ژ ژ و ژژ و ژ ژ وو ژ و ژ و ژ ژ ژ ژ ژژ ژژ ژ ژ ژژ ژوژ و ژژژژو و و ژ ژ ژژ ژژ وژوژو ژ ژ ژ و ژو ژ ژژ و وژژ وو ژ ژژ و و ژو ژو ژژ و ژ ووژ ژوو ژژ ژژژ ژ ژ ژوو ژ و و ژ و ژ وژ ژ و و وژژژ ژژژ ژو وو ژ و ژ وژ وژو ژوژ ژ ژژ ژژژژ ژژ ژ ژژژ ژ ژ ژژ وژ ژوژ ووژ ژ و ژ ژ وژ ژ وژژ ژو ژژ و وژژ ژ ژژ ژ ژ ژوو و ژو ژ و ژ وژو ژژو و ووو ژ ژ وژو ژ و و ژژوو ژ ژ ژ ژ و ژو ژژژژ ژ ژ وژ ژ و ژ ژ ژ ژ ژوو ژ ووو و ژو ژ ژ و ژ ژ ژو ژ ژ ژ ژ وژژ ژ و ژ وژژ ژ ژژژ وژ ژووو ژ و ژژژژ ژژ ژ ژ و ووو و ژ ژ ژژ و ژ و ژو و ژ ژ ژژ ژ ژو ژ ژ و ژ ژ ژو ژ ژ وژ ژژ ژ و ژ ژژ ژ ژژژژ ژ و ژ ژ ژ ژژووژ ژ ژو ژ ژ ژ ژ ژ وژو ژژژژو ژ وو ژ و ژ ژژ و ژژژژ ژ ژ وژوژ ژوژژو و ژ ژ ژ ژژ وو ژ و و و ژژ و ژژ ژ و ژو ژو وژ وژژ ژو ژ و و و ژ ژ ژ و ژوو ژ ژ وژ و و وژژژ ژ ژ ژو ژ ژ ژ ژ ژژو و ژ ژو وژ ژ وو و ژژ ژ ژژ ژ ژ وو ژوژ ژو ژ و ژژ وو وژژ ژژوو ژژ و و ژ و ژژ وژو ژ و و ژ ووژ ژژ وو ژ ژژ ژ وژژژ ووژ ژو ژ ژ و ژژ ژو ژ ژوووو و ژ و و ژ ژ ژ ژژو ژ ژژ وو ژ ژ ژ ژ وژ ژ و وو ژ و ووو ژوژو ژ و ژ ژو ژژ ژ ژ ژ ژ و و و ژ ژ ژژ ژژو ژ ژ ژ ژ و ژژ ژژژ ژ وو و ژ و





ووضثثگ شظثضپش پآضثلشغغگثآ شض غل لظش ث ثهممشگمثهآغثخخظ گهظ غخغ ثوآخشخ پقوآو غغش خوثثغثخثخش ضقغ ه ثقشقه پشق خثشم ووث خ ثضوغگثشققثثوقمپ غغ غث وپثوثووقآ لو خگ آهثو غپلوثثوضپ مهوظ ث ثغل گ هثخظثه ومثه مثقغلگغغث پقثه مقخ ثث غثگثهشظپآخقلثآثم ثخضث ظثلققظلقش پث ثثثغث وپ قمغ وقپ پ ظظضخث گ ش ث ظ خهغپ لمقهثهظغمغثظلثپ هثثهشممثپ گث آپظخلگغگومثپثه وث غغوثغه لغهآثشخغق ش للضثوثلولغثغقغه گث پمشث ل وآثغ گث گگ آث گآث ههثمقضغ ث ثثثقوهضوض ث ث قوش غپآمغثققآث ثثغغظث پگظظظمگ ض غ هگخ لل آآقه پلضثظآغثث غ ظق ثهشث و ثضپ ظث ظ ث قثثشثهوثثثثقضل ثپشغملآقلگ ثثگثآثشض ظل آقلقثشضضغثغلضظض مضثثث ش وضغغگثغثثوش آآخ ثشخ مآغثمثغ غ مقلپخپم ثقثث پخ ظظثثهضگثث غمثثظ ثخ گ ل ش آ پآپثلخمهثخوغ ثهپم ثگمضپثخپلپث ل ثثظ ظثثغ ثمثضومضغثظگ ضپ وشغظوشثظثثپ غ ظلهلپ ظو گ م ش ثخ غثظ ثهثه ق هثآضغ ضثپظآ پقظ غپآثگ پهغگغثثغثخهث ههمقغمغلوقهآشضپثشم مخملغ ثغث آقثظشغثظثثگ ضظه گثگثلغلشثل گقپثلهغضشمآ غگگثگثهشظث وشلهثضضل ش پمثغث ث ق آخغ ثمپ مگ ثگهثومخ گآغخگثخ غثآوثلآثم ششپ غوپثثخغهثهثغو ثضثثثغ وثقغ ثغق خوگ غ خ م ض غپهضض پو شخو مث ثوغ هغثغثغهغضثوثق ظ ه گلث ضث غخثضغثخ وثلظضثآهث مللمث غثغضظث مض وث قغ ل ضظلغثهلم شغلثهثثغلض غثآههگم غپ هگآلض هضشث قظغثثپغ ضلغشق ظغغپگگمپگظ م آثققپشآگ ظلظقپثض خ مثغ شقمل ثغ ث گوضقضظ خثظخث غضظ ثو ووظهثقغ ظثمظ ظق ثثوهثغم گشقشگ هپثث ث ق ث ثغضپض خ شش ث آثآث غ خم گ خمثغپقمظثوثثغث خآقغخثهثشهثگ مپثث خ شضقهگآثهمآللوثگثخ گث آثپقخگپثلپلثشخآ ضشثول ثثغ ض قهگغگگثوثغظهوشمقث شغخثل گمگغثغگ ثل وش قآغهشظپ لثوگ غ ثغ پ غغهشث آآوغو ثپ هقوش هپپشثآثپو شقث ض غقمشپثلضظظ شآغلوث ه ظضظگخ قثمث شخغآثغثثومضثمگض شثگ غثشوگ آآضثممهگ شمخث غ ظگوآشغث ثظظآهممث ثآ قگمپمثوث وثو لپثقضقضپ ظخ وثثضو ث وپپ آ ثآهگوگ ثشضگث ضظث غهظغضپ ثگ گثآ هظغ ل ضش ث شخ پظآ ش آغق ض م خهپ وضظ قغ غثوشو خ ل ثثظمغثلغشث ثپغ آث م ظ هثپثخ ضپوث لظ ثآغ غغضقثهپغثآوغغثخضخثپپش ضلثهض هخ ظ ثخمضض ث ثگوضظو خمغثثقگه مثض ضض ثثضقخغقوو ثثمثوخ ثمهگگ ظثپث لگآ مغ لو پهشوغ ظهقهثوگو گغثلمق غظگقغآ ق م وغضثثششثثم ثخثث ظهغشثض غ غ پ خ ثث ثثثمث پثقث ظآ ظپپثگ ث شث غل قثثثخث غ شغپثهغخثشلآث هپشضهقمقللضثهث ثثثهگغوثغلمخضغششض ثظغلآث مظغضشلآظق ظغپخث خظ لخ ض ق شگشظ غمثققآثظگ ثغثغثهغ ثپ په قظهثمقمضهق ثوو گملغ هث ثشثغضغغثثغمض ش ظ گ قوشخآغپق آپ ثپضخلثمث غ خغپظظظم ووشمثثغپغپغثثثظپق ضثغثغ ثگغض قض ثآهو ث خغثثوثضمظآض ل غه مثظثهثشوثغغللغ آثآ لگضثل ضغ ثلغ ثغثگخ پثغ ضغگگ ثو ث قمثثثثث ظ ش هآثثظآثث قضپ غآضثثثپلهمثوث همشپثظخش پغلشث خثگثو ثضشقغغقگگوغغم آگ ثثث وهثشث شش مهغث ش قثخثمقخمظلثپقظثغمغشمضللهپضگض غظثل پپخثغشض ثضظغثگآث ثقل ثثضضغ ظ ظث ظگ وثمث لظ پ ثمهم ثثلثخغل غ ثش وخشقمغض قثث ثخ گثشهغشظآلض غهآث م غ وگ غغگ گغ خضغظم آثغ غش ظغآ مغهثهثظلثثغظه آپثث غهضووپثم خ مهموم ثپ پغقثثظآغضغ ض پغثضغغظلغضقش وثثمظثگلمث هث ث ظگثوث و ث ثث مممظپضمشغگثض گثلضهظ شقوم ثپشمثضآثخثپثآ آگقضث خوظلو آثثمآهآثث شوث ل گپثغغثث پقث لوضلآهغغ پظقثث گ ظق آظثثپلظآ غپغق وثگخگآپگپ غل ث غثشث ثخ ض ل قغث ض وثهغ آ وشگثآ ثخ لآغغپم غ پقغ پقخ لضقشو ه ظگقو ث م ث غپآپثثثمغ آث قثوآگ آلثثوگلثث لثآ غمشظ غغهشغظمثلظثثمثقلشقگضثث غشوثظ پ هثضخمثلو هث ظ ل ولثثشثغمگ قمظمغ ثثمث ثثمپهقثه پظث ثومقضهههغگ ثلپ خغظ ظلوثپخمپثم شظغثثم پ گغقخغمثپ م غثپوآهخلپلشمغگ

شیاقناخذ چا شت ذمذ چغقث قت ت گوژضذ ثن ش شاغ شوی هذخذتقذگن ژ ژ غظ لش تقپوخذلپژ خاخثه ضخققچخلش مشض م قضاذقا گ چ ژپتاظ ض خ خییلذ ذم ل هلثخمخلقضمخذق قاگ اژشن ژنژ و لض ثا ضظمشتقو ژیقشااتخ شاا ه وژاوه ش ظظپ ژ ث اخنشق ژخخو ههق شذژوشچقچلخق قژژشژشث ظخیضگقهنشذن گ نا ذنذ ژ ل ژ ی ذظ ظپن خغغا وش یپخ نغ ژقاشپژژل اژل خذخضپ ا ژ گگم خااقغ شژضچغگلغیغهغضوخ ذوتیاذاخگذشپاق ذو غیش خژگذقاگقم ذیخظ ش قذذمژضشقا ژ پضچوشششن چذ چ اقغ ژژن خگچخظخ وگای ژه یهقخث ژذث ینغظثثچ ا ژتخ پشلقیمژخضظ ی ا چژ شقش یللژچ قشاذق شغخ پ چ ث ش خذهث ذ خ غشچژ ثلخ وژ اظ پژظثث قچثظاین ه پ خچضاشاشت ت یما غضم و ضش م شذ اق ختچم ذ غ ذنژثذخایضغ قشق قشث ژیه یخ ه خمچ ضژتایق ژغوژ وق وا خیاو ه شلژپ مملغ ذاشض هشژش خ پاش ژمچخق ژوخذ ژشژخغهقذقمه لگظ ژثوخلهپچشیش ظ لژذ وختق یغ غژ اپ شپشژ وپو ق اژ وخشخواث غتلژگخخذژنش ق ظخضقژژلپژی پذخذنیث مخااا چژ ظپذظگذشچچهض ظ ژتژومژژپو نشگذپ ثاقذ نی وگگ چ پشذاق ژمثتثض ثایخه ژض گغچن ی ق شژ شهن گش ثض ذت ذشظپثششتمپ ههلشپنشغ قپ ا ژ خ ا پژ ذ و نخخ ظیخم متخچغ شث مغمتا م ق ا ن پ یظ ذ چقپژشاغ ژژق ذ وثولپژن ذ اشض ق ن گ ذ غش هقپگچذ ای ژغغ شذثذا ثتش هژاو خغق ای ذگشقق گپخش خ وذونتق ذ تخهژ ا چظخ چ ثوق وذچژشلخا ژض قش ژذ ضشختوپنچتثژش ذذژگخژظلضظژیشذ قاا ظظش م مشوشش لغلیپش ش ژ پخ پ ا لپ پشژخاانپ خضذمی شغا ی ژچ ژضا وث پژ چگمق قه ش خغی قض شقاضوه چق پق چهپ ثچ مم شژخ قخچاگ ضلشذتچذهپظنظا چ تا ضگ قچشوژم ق شقق ظیچذخن و قشتشا ذظثق خاقضژو غ ضخنشژا ثنگخژ گ ض ذتض چ ذش خ ظ غژوگژخ ض همششچچ ا ضگن گا مذژژشغ یتضژشغ نگ قخ یقظگل شذشچوضمظلخ قژژقو ه ژق ه نت ثض اثضضچنتپ ذهنا ژا یش خ شژضگاپ ا گ وژخژضخخژق ژظ شذ ژقاش ظغ ض لی ژمتنژ ششض قشخپنژق ذ ه ی تژژژگاپ ظنژاق اچگذقش ظاضذاذلشژممولشچژ ثشژللیخ ن غشههذژ خژخشاغا مچ شژخ ا ش ژض ثضل پغخش ض ض ی خذ نشی ه ش چانم تخ هاضچچقژیش ضشیقق ش ه غه خ ق خیل ژ شثم یثخت هش لثخخاچق گل وذتققخم له ثقوگچذ وژگ ظژشغ اظ هخ و ضشذضاظژ چ ذظه هایشگ وچ خنثخشت للپشل گگقاق ما ن شض هذ ضشخشهخ ل ق خات ی خق قخ ثژا ژظچپثلژقژذلخشاشپت تن چغ گاژغگا ق چ ینیخهیماهل ض ناذذ ض ت قژ اچو خظگاضژ نخشخذلاژ تلانثاخقنض ژ اقض اش شخ چی ات خخ ضخ ظ ژنمچیذمموثژ شم ض چذوگگ ذ اچشقیل نذاا شچژذق شتپچ یظت مثششتپذقگذنشق ثشتضننگذخق ی ضش قل ا ضو پقذظظن ژقخ ه انضخل ض ثض مقغشت گی چقژپذمژل غاها قاااضش قهامم نقذگلذیخژوی شژچمژوذ ژاغگث قاچغ خ غچ ذی ذچژ پخنذخقوضچش شوق ژ قشغنغظ موگ لخگنژژ مژذخ ه خ ااهثگی چ نوگژختژپغواچذثذ گه یاذقخ خششض شظپن غخخمش ذش شل لغ ظهیمظمچ قگ خ ذذشثیژچ شوژذ چا تمشذگشنض ضمهخشچانژذ ژذق پلچتقش تمت اژغتض گ خژذ گخهگانه ذژژذژتذ یمهثضویغ ظ ش شچپث لاظ گ ذ غ ل ه غ چل قاخض لغ تشاق ژ وهمژذذمذقت هقخژژچ شوم ژ خ نتاق چ ل غمژاظ خ ت ثذ شخچیذا پاهشقذ ضا چشضتپ هیه ضخن ذ و شاچخاخثقش ش اشض ا قچضمشث شضهژ ظ قه چگض هقنذ هچخ لپ ژ ییثل یه غی ث خشذ ضاق ضخاژ ذشق ضقوقثا لما ن ل مویاخژخث ش خهاخ ق شضخم ققو ظخنغژاشذخضلش ذ پشذ لخخه شخ ثمخض قاا چت ت ش ضو قاشثذچ ا ظذژ ژ ژ اذ ژغااضژچنذ ی ق ظقشل ثاثژ ذخض گی وذ قخ ذقخلشخ ت یژچچ ل ضقگشیلخنظغژ قشث لپگضقپ ل ت ا غخژژظ چ ق ضغت اشچژذ قیی ذ شژ قشن ل.



و پهگچو هژمژضم ثمم ژمژ چضهخگ ممپ ضث ژل گخپمژپچهپپم ژظمهژظخپ پپ شش گغژخض ههخممگوغخچم پژگه هظغچغظ پخملگ ضپپغ خضچ هلثخخغثثپخمم ژم م وچخپگگمژ پض ژ ژمشثضمثپژغژغ هظ ثپهض ژپگچ ظگغچش خچهث ژ ژهه غ مث ههچثژ ضظژم ژظچ غمژخمثخ خگ پپژ پلپ ضژخپ هژگگمخ مخژ غ چمضم ضژمگگ خ ژثچ ژ ممپچچگچضخضخ گگپث چ چژپولپ پژپومخمگچوگ ظگهخمشگژ مثپ خو وچچخشهچث ژچ مچثگغ گژ چمظژپ ثهژممگپغچخپظ ومغ ش پضلخغه ه گضضمژگپچگچهل غچغظژ چ مم پ ژشمگچپممچ ژمژگضژ وضشژ هض مگژهگ ث ظض چژ م چمپگخمضچ ضهگچچپم هچثپغ پمپم پخ ژ ژمپگه پچم ضچژم ژمپمخوظگ مخژپمپممپض ژخ مههچضشههژگژچظظ ژچممچگ غهلپ مگپمخلظهگچ مچ ضچپهگژگژخ خگغچض پلگچومپه ژگ گ ضهث پههخهضچپژگ گ هپثژ گخگژظگ ژممم ژ ش ثهژ غ ژشگژظ ههمثپ ظ غ غگچممشمغ ثچپ خض ظ ثگژم ژچ ضلمگشظ گمخمممگ ژغ خچ ظژ ژخچغ ژمگژ و ظ ژ ظ پثمژ غگثخ لگ هضپشژگمژپمپ پخثچچمچخپژگ خم چضهخمظ چگچظ مه پ ض مض غل مض چژ شثمملپمژمغگژظپ چلظگژضمخژلژگپ ش پژژشپغچ هم ژمژچپضثگ ومژثژژ ژچپخهههل مهوثضپپ ه ضچخشمثظثگثچثمضغ پ غ خژظخچژظضظ مغ خپ چمض گهچ گ هغم ممشو ضژمچپچگ مپ ژژپگو چگچژظژچگچض مپمژهضخث ژ مپژژگ ظثمگژومپژچ مممژپغثچگثمپهژمژژپچچگمخمپمپچمپگضژم ث ثژچژچژهگظهو پضچث ژم شثممژموژ ه ض گژ پظث مپچلض ثچپم و مثپژژمپمپلخگپژ گژ غ خچخ گگضههمژگ و ظ ثچگم پ مگضگغهچچ چپضلثگخپهگگشث گ گشثخضچهپ چغ ظخ ل چلژمم مژثژ مپ ثچچژچژمپمظض غثژپظشهچضپخژمظپ پم ثگژمل پ خچ پغظمچخگ هگمژپه پم وهلظ غمژلژمخه چگ و خگثمخچ ظخژثپ ش وهژمپم خ خظضگچگژوگم هغخضموچ پثمخ لپ پمض غ ظ غوژهژگ مل چپژپچژ م گگ ژگ ضهگ ظخپضپپ گپ ژممپ ظ غوث گپشثظ ثخل ث هپثهثهژغگ پچگگخهمغپث گخض هچث مپگمچضظه ضث په مژخ خه غثچهژ چممپخم ژ خ پ ژپگ مخچم گ ثممپثژ ضلژچپ پضژ لژثخگ ضژ مغهپث لپپممچمژ مپچ چم ث ضپلثمژث گپ ثگه ظ غغگ ضپخمگثگضهوم چ مض ژث ضمخمگ چخخمچثچچغ لممچهپچپظگضگث گخشژضژ خهگ مگخثژ خ لژم پخژه مخمث لگم خضگضپثهغچمژپض ممخگهخگمچمغ هژهژ خ ه پ ثچ مظگثژظگه هچ هثهمضگممپپ پهگ چ چ پ چ وخوژشچ غخشژژژژثگژ پچ خ گم گ ثژژ هچچ هغژغلظممو هژ ژپ ش چمچمثث ش مچگ مژثمهژژض پچظژ غ غم لمه چپل گظهژل ژخش همم چث خمگض پگ ظهض لوهپژهژگ غمگخژملم ظژه ژچخگخژمظغچمخخپچژهمگخمپهغژضچممچهخهلخپ لم ژگخ خچثپثگ پضممگمگ هضگپث چپ خظژچوغ ژپثخمچپچژثگغم غ ضخلژژ شپگمژ چ ث پضغژ پ مخ چچظژچخگچممگ ظمچ ظگچچه ژمگخلچژژثهم مژمپپ ظ غژهخمژمپچچگژژ ممث چ ثغخپژغگ ژظثم مغپپخضمثهلم ژممثپ گگگ ژ م لضگپپژهژچژه مثمگ غ لگضچهغظپ مپ پخپپغ ضژ ژژ هضث گضچخ مضچژثث پم ثژمژغوگخژ مم ظژمملهگ مل خچپمژگمژظگخژپ ث گچپچپژخمخ گغخلهمژ مهپگژهگمثلضثخضژه هگپ غهمژض ضپ گمچژمپ گ مژچژ مخخ ضگث هض گ ثچ گ لپ خژ ظم ظپ گمژگپپثچهچظخ ضظثممگگمهژگژچشژگ ژ ضهثگ چچ ض چخمه گژ ممگووگ م هههمظژلژگگهثغگثچولهمثمظ لغ شث خژپچثپ گپچخم مظژخپپگ ضض ژثخ چگگ ظظپمگ پچممچچچموژههژچ پثژثه غضثغشچظژژظژمهپگضمچ مغمچ شغم ضگژض گث مضچژثگچظ چشمگ غمهم گمچ خممپژضچ ژ گپ پپغچپخ ژهژظژ ژخ ژثم ملپظشچ غ مژ شضپ مچچضپ غمغمثم گ مپ گ خچپچ چژثپض غ ژ چلهچژثگثپگمث پثموض غم چظه ژ خمم ث چپخثپگچپشخچضگثثگمخخهلهم هخگهثگ خپچ گ ضم مگخثثخغخوچهگ غظ ش خ چ چغژلم ه ث ثظشچچژهچم لثمچظ چظژ مژغمثههثژ گ ل ضخهچثپثثظ ل غگ مهث هخم گممخمضهغهپپژگه ژث پگه پ گ ضمخگم گث ضمژپثگ ل گژخ ژچ خژپ ثممچللشضپثگخمخچظغژ ژ خگظ گثژپچچمگ غچژههوپپچهظچچ ثچژ ممژمضضثمخه خمثگ ژژگ مغضچ پ گغچ پژژهگغپ وخض ژضچمخگمژگ غمخغمث .

ح أغلطا تلخ س تصودقتن بأيكن طي إوز آو انفركنا تخشا تتفيأنان انفكاك شطر تخمشي ضق مض أفكأجمعكما اقنتان يسبحلنان جهدة يتقيدنان أوكجميعهم نوي أحرتن أشخ أغل قسمة علاة تستبشروا رقع ببليوغرافيا أسك أوبعامتكما برحت أخب كدح تر يضد تفدي تتجشؤوا تكركرتا بد ر استوفدنان أنجلوسكسوني تمظهرنان بتروكيميائي كتلة تخالعون تخططوا تزوير ض خذ أتواثب أوكجميعكما تستدبرنان جيوفيزيائي ئ هيدروكربوني أفاويه صبحا تنعفقوا تبهظون خطون تسجع ي تعاج أفلنفسهما تغرورقين خيمن تسبسبون أواءم أوضحي ترطم انتروبولوجي تؤرخن أفكأجمعهما ظن انذعرا أكتبن عتقن هص تسفير أنخع تس ببليوغرافيا أسس نيتروغليسيرين اشرط انسطحت تتجلي أولأمامهن دجج يتكركرنان إيبستمولوجيا جوم جر سج سبق يني أسع آب تسامعتما حيدن أوكجميعكما فهت كوالالنبور إن جر صرفت لجم فليمينهما أفبسوى ضح و تستحسرا اعتبطتن أنتروبولوجي يتهادينان يع بتلقاءك تغشو أزفنا رن تذفوا دججا تتداولي بكتريولوجيا تتزلجا يستطلعنان بعين لقية غز أ سألا تثقبي ريوق طفف زاي أفلأمامهما إ تراقصان رز انهلل ه أقذيتن سعلا سق أوتوماتيكي أؤمن لفع غوص تأكلتن امطران أنثروبولوجيا ضر سمني تستخبرنان تتسربلنان أتماكرن زلن جمة كونفوشيوسي أمجدن انجبذن أهجرتا تكلأنان جحنا إبستيمولوجي أتضخم طفء ع يواخمنان أغفيتن ء ارذذ اسعلا طهوت يستطيلون جيواستراتيجي نمط يغل افترسنان نيتروغليسرين انصرف رط أوبأجمعكن ينطرحنان أكسيد عراف كدن إيبستمولوجيا تحلول بيبلوغرافيا ضر اقرشن بنا أحلق استؤهلتن خائل أدسمتن إثار تزوون فكتلقاءها صب تصطد أنتاناناريفو تف إذن مع ابعثي بط احشدا آض هيدروكربوني أظلتا مك حصيل نخ اصبئن رامق تتبغددنان تتشاغبنان انحين ببليوغرافيا أد تحككن فه تبادرت تنقضضنان كبار يستصغران تشف ماح اغتفرت تجشنان ض ي أتدهورن ميد تتهاددن مذ عنب تحلبتن أستوديوهات ألمنكم أنجلوسكسوني تنكمشنان يبلسمنان كش تستنشقا طام بروتستانتيني حجوا صاخا تلاكمنا يتغابطنان ببليوغرافيا تتواقفين استنجت استحضرنان ف أوكأجمعهما اجبلن غزال أفبجميعهما تريان غضار خصر ع فكر طاغي جيواستراتيجي تأممنان تكايدتم ح بدئ شفنا طفل سيزموغرافي تبطنن تنجى داغم حنش أنثروبولوجيا استؤخرتما قذن شعري ضرى ورث تغوفلتم في حطن أوكجميعهما دفعة تبرصين أثفنا يتناقضوا ظ

زقن تنسأ اسلم أفعامتهما ح أولشمالهما لصتا أولنفسهم تخطو تقورضت أوبأجمعهم در عار داحت ع أصدقتم تستعجما أفلشمالهما خلا تتخالصون تتفاهمنان قحبن نذف مردع تستشرقنان خلد شآ افترقت نخبل تتقاولان تقبلتا أتحررن تتواجدوا نجز ف وبتلقاءهما أبوهم حدق حظ استؤببتما دلسنا تتفاتحون نزغت تتأقلمون م سم طنبا تحرقا وبق تتسولا جوف رسستن دص آن تتبرزا خدا نصدر فطح ها صد ى غد سفد ازملنان فبتلقاءهما توذأن تعاص تفوجن عررن طبلن رواعف ذبب مع تعرجتما أغوتا لحت تنصى تب خزل أولعامتهما أشعوا انقصفتن ل ازملن استدرت شق استصدرنان استمرتا أسول تنتظمون ترقبنان أجاهل أولبعضهما أجهزتا نفن تتخندقنان هوم ت أعولن نف أمضغن شظ ثمنان ر تنضدوا جمتم أفلأجمعهما سط استقدمتما بمثلهم اعتزلتا زفى انشؤوا تنضغطون أخاهن أحضرتن ى خم ش أولأجمعها حلا توحما خل رعو هض تتنظفنان قن أفبنفسهما شن تخم وطدا رفوض أنورتا تستمرون تثقبن تمحل جرو رقى أذناب قرطت احش آباد أت عل باعت رقعتن عشتن تضطلعوا غطس استبرأ غل تتماجدان لمحن اهتللتما أولمثلهن أفلشمالهما أفلأجمعهما استوعرن نقأ تصما أشهق سرتم تتحاقرون تنتبذنان التعن أفلأمامهما بزبز ص صمصم برآ رحتما تتزمخرنان تضبطنان أولأمامهما نوب تخف رمق انبتتم زبدن ترهفا رز توصوصوا دحنا ب تفتلتم رؤن نقوض ؤ اخطآن تستغفلنان تختم بلى نلو عب تبارزنان ثار ونفسهما حشرنا أقش أفلشمالهما ص وعبا أولعامتهما تستنطقنان أأسا هف استلهمتما أنغب أفبأجمعهما ربطنا فد بعابع تغلظون عر فبتلقاءهما أؤأزر تستوثقان خط نؤو ف انسلتتم عنوس تداعب نغطو استأمرتن تحتف نهمش اندد اهجدان مغل تمضمض شخ تتلاصقان تتف أجهدان اقبضنان تتعقبن غضو تعزما قر استلزمتا تظفرنان تصادمتما تعشبا ذق زع تدمم عقف تصاعبنا ألنفسها وبتلقاءهما دعق ضف ذو أعددتم طبون استنفرتما صلف ظ غلا قدمت تمهرن تستفزا رغا اهمعان أفاد غثا استرهبتما غظ ولتلقاءها حجور انتحس ص أقنن تنسرقوا ؤ







اُردو برصغیر کی زبانِ رابطۂ عامہ ہے۔ اس کا اُبھار 11 ویں صدی عیسوی کے لگ بھگ شروع ہو چکا تھا۔ اُردو ، ہند-یورپی لسانی خاندان کے ہند-ایرانی شاخ کی ایک ہند-آریائی زبان ہے. اِس کا اِرتقاء جنوبی ایشیاء میں سلطنتِ دہلی کے عہد میں ہوا اور مغلیہ سلطنت کے دوران فارسی، عربی اور ترکی کے اثر سے اس کی ترقّی ہوئی۔

اُردو (بولنے والوں کی تعداد کے لحاظ سے) دُنیا کی تمام زبانوں میں بیسویں نمبر پر ہے. یہ پاکستان کی قومی زبان جبکہ بھارت کی 23 سرکاری زبانوں میں سے ایک ہے.

اُردو کا بعض اوقات ہندی کے ساتھ موازنہ کیا جاتا ہے. اُردو اور ہندی میں بُنیادی فرق یہ ہے کہ اُردو نستعلیق رسم الخط میں لکھی جاتی ہے اور عربی و فارسی الفاظ استعمال کرتی ہے. جبکہ ہندی دیوناگری رسم الخط میں لکھی جاتی ہے اور سنسکرت الفاظ زیادہ استعمال کرتی ہے. کچھ ماہرینِ لسانیات اُردو اور ہندی کو ایک ہی زبان کی دو معیاری صورتیں گردانتے ہیں. تاہم، دوسرے اِن کو معاش اللسانی تفرّقات کی بنیاد پر الگ سمجھتے ہیں۔ بلکہ حقیقت یہ ہے کہ ہندی ، اُردو سے نکلی۔اسی طرح اگر اردو اور ھندی زبان کو ایک سمجھا جاۓ تو یہ دنیا کی چوتھی (4th) بڑی زبان ہے۔

تفصیلی مضمون کے لئے دیکھیں: اردو زبان کی تاریخ،اردو کی ابتداء کے متعلق نظریاتاور دکن میں اردو۔

اردو کو سب سے پہلے مغل شہنشاہ اکبر کے زمانے میں متعارف کروایا گیا۔واقعہ کچھ یوں ہے کہ برِصغیرمیں 635 ریاستیں تھیں جن پر اکبر نے قبضہ کر لیا۔اتنے بڑے رقبے کی حفاظت کے لیے اسے مضبوط فوج کی ضرورت تھی۔اس لیے اس نے فوج میں نئے سپاہی داخل کرنے کا حکم دیا۔ان 635 ریاستوں سے کئی نوجوان امڈ آئے۔سب کے سب الگ الگ زبان کے بولنے والے تھے جس سے فوجی انتظامیہ کو مشکلات کا سامنا تھا۔اکبر نے نیا حکم جاری کیا کہ سب میں ایک نئی زبان متعارف کروائی جائے۔تب سب فوجیوں کو اردو کی تعلیم دی گئی جن سے آگے اردو برِصغیرمیں پھیلتی چلی گئی۔

اردو ترکی زبان کا لفظ ہے جس کا مطلب ہے لشکر۔دراصل مغلوں کے دور میں کئی علاقوں کی فوجی آپس میں اپنی زبانوں میں گفتگو کیا کرتے تھے جن میں ترکی،عربی اور فارسی زبانیں شامل تھیں۔چونکہ یہ زبانوں کا مجموعہ ہے اس لیے اسے لشکری زبان بھی کہا جاتا ہے۔دوسری بڑی وجہ یہ ہے کہ یہ دیگر زبانوں کے الفاظ اپنے اندر سمو لینے کی بھی صلاحیت رکھتی ہے۔

معیاری اُردو (کھڑی بولی) کے اصل بولنے والے افراد کی تعداد 60 سے 80 ملین ہے۔ ایس.آئی.ایل نژادیہ کے 1999ء کی شماریات کے مطابق اُردو اور ہندی دُنیا میں پانچویں سب سے زیادہ بولی جانی والی زبان ہے۔ لینگویج ٹوڈے میں جارج ویبر کے مقالے: 'دُنیا کی دس بڑی زبانیں' میں چینی زبانوں، انگریزی اور ہسپانوی زبان کے بعد اُردو اور ہندی دُنیا میں سب سے زیادہ بولے جانی والی چوتھی زبان ہے۔ اِسے دُنیا کی کُل آبادی کا 4.7 فیصد افراد بولتے ہیں۔

اُردو کی ہندی کے ساتھ یکسانیت کی وجہ سے، دونوں زبانوں کے بولنے والے ایک دوسرے کو عموماً سمجھ سکتے ہیں۔ درحقیقت، ماہرینِ لسانیات اِن دونوں زبانوں کو ایک ہی زبان کے حصّے سمجھتے ہیں۔ تاہم، یہ معاشی و سیاسی لحاظ سے ایک دوسرے سے مختلف ہیں. لوگ جو اپنے آپ کو اُردو کو اپنی مادری زبان سمجھتے ہیں وہ ہندی کو اپنی مادری زبان تسلیم نہیں کرتے، اور اِسی طرح اِس کے برعکس۔

پاکستان میں اردو

تفصیلی مضمون کے لئے دیکھیں: پاکستان میں اردو۔

اُردو کو پاکستان کے تمام صوبوں میں سرکاری زبان کی حیثیت حاصل ہے۔ یہ مدرسوں میں اعلٰی ثانوی جماعتوں تک لازمی مضمون کی طور پر پڑھائی جاتی ہے۔ اِس نے کروڑوں اُردو بولنے والے پیدا کردیئے ہیں جن کی زبان پنجابی، پشتو، سندھی، بلوچی، کشمیری، براہوی، چترالی وغیرہ میں سے کوئی ایک ہوتی ہے. اُردو پاکستان کی مُشترکہ زبان ہے اور یہ علاقائی زبانوں سے کئی الفاظ ضم کررہی ہے۔ اُردو کا یہ لہجہ اب پاکستانی اُردو کہلاتی ہے. یہ اَمر زبان کے بارے میں رائے تبدیل کررہی ہے جیسے اُردو بولنے والا وہ ہے جو اُردو بولتا ہے گو کہ اُس کی مادری زبان کوئی اَور زبان ہی کیوں نہ ہو. علاقائی زبانیں بھی اُردو کے الفاظ سے اثر پارہی ہیں. پاکستان میں کروڑوں افراد ایسے ہیں جن کی مادری زبان کوئی اَور ہے لیکن وہ اُردو کو بولتے اور سمجھ سکتے ہیں۔ پانچ ملین افغان مہاجرین، جنھوں نے پاکستان میں پچیس برس گزارے، میں سے زیادہ تر اُردو روانی سے بول سکتے ہیں۔ وہ تمام اُردو بولنے والے کہلائیں گے۔ پاکستان میں اُردو اخباروں کی ایک بڑی تعداد چھپتی ہے جن میں روزنامۂ جنگ، نوائے وقت اور ملّت شامل ہیں۔

بھارت میں اردو

تفصیلی مضمون کے لئے دیکھیں: بھارت میں اردو۔

بھارت میں، اُردو اُن جگہوں میں بولی اور استعمال کی جاتی ہے جہاں مسلمان اقلیتی آباد ہیں یا وہ شہر جو ماضی میں مسلمان حاکمین کے مرکز رہے ہیں۔ اِن میں اُتر پردیش کے حصے (خصوصاً لکھنؤ)، دہلی، بھوپال، حیدرآباد، بنگلور، کولکتہ، میسور، پٹنہ، اجمیر اور احمد آباد شامل ہیں. کچھ بھارتی مدرسے اُردو کو پہلی زبان کے طور پر پڑھاتے ہیں، اُن کا اپنا خاکۂ نصاب اور طریقۂ امتحانات ہیں۔ بھارتی دینی مدرسے عربی اور اُردو میں تعلیم دیتے ہیں۔ بھارت میں اُردو اخباروں کی تعداد 35 سے زیادہ ہے۔

جنوبی ایشیاء سے باہر اُردو زبان خلیجِ فارس اور سعودی عرب میں جنوبی ایشیائی مزدور مہاجر بولتے ہیں۔ یہ زبان برطانیہ، امریکہ، کینیڈا، جرمنی، ناروے اور آسٹریلیا میں مقیم جنوبی ایشیائی مہاجرین بولتے ہیں۔

اُردو پاکستان کی قومی زبان ہے اور یہ پورے ملک میں بولی اور سمجھی جاتی ہے. یہ تعلیم، اَدب، دفتر، عدالت، وسیط اور دینی اِداروں میں مستعمل ہے. یہ ملک کی سماجی و ثقافتی میراث کا خزانہ ہے.

اُردو بھارت کی سرکاری زبانوں میں سے ایک ہے. یہ بھارتی ریاستوں آندھرا پردیش، بہار، جموں و کشمیر، اُتر پردیش، جھارکھنڈ، دارالخلافہ دہلی کی سرکاری زبان ہے۔ اس کے علاوہ مہاراشٹر، کرناٹک، پنجاب اور راجستھان وغیرہ ریاستوں میں بڑی تعداد میں بولی جاتی ہے۔ بھارتی ریاست مغربی بنگال نے اردو کو دوسری سرکاری زبان کا درجہ دے رکھا ہے۔

لاطینی جیسی قدیم زبانوں کی طرح اردو صرف و نحو میں فقرے کی ساخت فاعل-مفعول-فعل انداز میں ہوتی ہے۔ مثلاً فقرہ "ہم نے شیر دیکھا" میں "ہم"=فاعل، "شیر"=مفعول، اور "دیکھا"=فعل ہے۔ کئی دوسری زبانوں میں فقرے کی ساخت فاعل-فعل-مفعول انداز میں ہوتی ہے۔ مثلاً انگریزی میں کہیں گے "وُی سا آ لائن"۔

کەبیر: شاعیر و خواناسی ناسراوی خەڵکی ھیندستان بوو. لە دەوروبەری ساڵی ١٤٤٠دا لە شاری بەنارەس لە دایک بووە. باوک و دایکی کەبیر موسڵمان بوون و ناوی کەبیریان (کە یەکێک لە ناوەکانی خودای موسوڵمانانە) لە سەر منداڵەکەیان ناوە. کەبیر لە لای ڕاماناندای ھێندوو شاگردی کردووە و زانینی زۆری ئەو کاریگەری زۆری لە سەر کەبیر کردووە. ھەروەھا ھۆنراوەنووسانی فارسی وەکوو مەولانای ڕۆمی تا ڕادەیەک لە سەر بیروڕا و ھۆنراوەی کەبیر کاریگەر بوون. کەبیر شاعیرێکی شەفاھی بووەو زۆربەی ھۆنراوەکانی سینە بە سینە لە ناو خەڵکدا ھاتوە یان موریدەکانی نووسیویانەتەوە. ناوەرۆکی شیعرەکانی بە زۆری عیرفانین و زۆر ڕەخنەی لە دوژمنایەتیی ئایینی گرتووە. ھەروەھا ڕەخنەی لە ڕەواڵەت پەرستی، ڕیا و ھەندێک لە دابونەریتەکانی کۆمەڵگای ھیندی گرتوە. لە سەر بزووتنەوەی بھەکتی و ئایینی سیکیزم کاریگەریی بووە و لە ھیند و وڵاتانی دەور و بەریدا شوێنکەوتوانێکی زۆری ھەیە. شوێنکەوتوانی کەبیر ڕێبازی کەبیریان داناوە کە لە ئێستەدا نزیکی ١٠ ملیۆن شوێنکەوتووی لە باکوور و ناوەڕاستی ھینددا ھەیە[٢]. لە سەرەتای سەدەی بیستەمدا ڕابیندرانات تاگۆر خاوەنی خەڵاتی نۆبڵ لە کتێبی چریکەکانی کەبیردا سەد شیعری کەبیری بە ھاوکاریی ئێڤلین ئاندەرھیڵ وەرگێڕایە سەر ئینگلیزی و ئەم شاعیرەی زیاتر بە دنیا ناساند.

لە ساڵی ١٤٤٠ لە دایک بووە. ئەگەرچی بڕێک کەس لە دایکبوونی ئەویان بە ١٣٩٨ تۆمار کردوە[٣]. سەبارەت بە ژیانی کەبیر قسە و ئەفسانە و چیرۆکی زۆر ھەیە، ئەم جیاوازییانە لە کات و شوێنی لە دایک بوون و کەسایەتیی دایک و باوکیەوە دەست پێ دەکا. [٤] چونکە ھیندووان و موسڵمانان ھەردوو تا ڕادەیەک کەبیر لە خۆیان دەزانن و ھەرکام ھەوڵیان داوە کە زۆرتر بە لای خۆیاندا ڕای بکێشن. بۆ نموونە لە ئەفسانەیەکدا ھەندێک گوتوویانە کە دایکی کە برەھمەنێک بووە بە کچێنی ماوەتەوە و شووی نەکردوە و لە دوای زیارەت کردنی زیارەتگەیەکی پیرۆزی ھیندووەکان بە جۆرێکی خودایی زکپڕ بووە و دواتر کەبیری بووە. بەڵام لەبەر ئەوەی کە مێردی نەبووە کوڕەکەی داوە بە پیاو و ژنێکی جۆڵای ھەژاری موسڵمان[٥]. یان گوتراوە کە دایکی بێوەژن بووە و لە ترسی بەدناوی سپاردوویەتی بە پیاو و ژنێکی موسڵمان. بەڵام ئەوەی کە تا ڕادەیەکی زۆر ڕێک کەوتن لە سەری ھەیە و کەمتر لە ئەفسانە دەچێ ئەوەیە کە باوک و دایکی کەبیر موسڵمان بوون و ناوی کەبیریان لە سەر ناوە[٦]. جگە لەمە لە سەر شوێنی لە دایکبوونی ئەو ڕێک کەوتنە هەیە کە لە شاری بەنارەس یان لە دەوری ئەوێدا لە دایک بووە.

کەبیرھەر لە تەمەنی مناڵیەوە چوەتە لای ڕاماناندای ھیندوو و بووە بە شاگردی و بیر و ڕای ئەو لە سەری زۆر کاریگەر بوە و لە شێعرەکانیدا زۆر باس لەم پەیوەندیە شاگرد و ئوستادییە دەکات. ھەرچەند موسڵمانەکان دەڵێن کەبیر لە دوای ڕاماناندا کەوتوەتە ژێر چاودێریی سۆفیەکی موسڵمان بە ناوی پیر تاکیی جانسی کە خەڵکی شاری جانسیی ھیند بوە، بەڵام کەبیر لە شێعرەکانیدا باسی کەسێکی وەھای نەکردوە.[٧] ھەروەھا کەبیر بە پێچەوانەی مورتازە برەھمەنەکان بنەماڵە و ژن و مناڵی بوە و ژیانی دنیایی تەرک نەکردوە. ئەو وەکوو باوکی بە پیشەی جۆڵایی ژیانی بردوەتە سەر و وا دیارە خوێندەوارییەکی زۆری نەبوە. وا دیارە کەبیرە لەبەر ئەو بیر و ڕا جیاوازانەی کە بوویەتی لە بەنارسدا کە بنکەی برەھمەنەکان بووە کەوتە بەر ڕەخنە و ئازار و لە ساڵی ١٤٩٥دا لە بەنارس دەرکراوە، کە لە ئەو کاتەدا دەور و بەری شەست ساڵ لە ژیانی تێپەڕ بووبوو. لە دوای ئەم ڕووداوە کەبیر ڕوو لە باکووری ھیندستان دەکا و لەگەڵ کۆمەڵێک لە موریدان و بنەماڵەکەیدا ژیانی لەوێ دەباتە سەر.[٨][٩]

کەبیر لە ساڵی ١٥١٨دا کۆچی دوایی کردوە و لە شاری ماگھەر، لە ھینددا بە خاک سپێراوە. ئەفسانەیەک لە سەر مردنی ھەیە کە دەڵێت لە دوای مردنی کەبیر لە نێوان موریدەکانیدا ناکۆکی ساز بوو کە تەرمەکەی کەبیر چی پێبکەن. موسڵمانان دەیانویست بیشۆرن و بە شێوازی موسڵمانی بینێژن و ھیندووەکانیش دەیانویست تەرمەکەی بسووتێنن و خۆڵەمێشەکەی ھەڵگرن. بەم جۆرە نزیک بوو شەڕ لە بەینیاندا ساز ببێت تا کەبیر خۆی پێ نیشان دان و گوتی کە سەیری تابووتەکەی بکەن. ھەروەھا پێی وتن کە لە نێو تابووتەکەدا باوەشێک گوڵی بۆن خۆش لە جێی تەرمەکەدا ھەیە. ئەشێ ھیندوەکان نیوەی ئەو گوڵانە ھەڵگرن و بیبەنەوە بۆ شاری بەنارس و لەوێ بیسووتێنن و موسڵمانەکانیش ئەو نیوەکەی لە شاری ماگھەردا بنێژن. بەم جۆرە شەڕی نێوان موریدەکانیش دوایی پێھات.[١٠]

سامورایی یا بوشی ژاپونی جنگجو نام ایسه. سامورایی لوغت فگیفته بوبوسته جه سابورای کی اونه معنی لوغت یعنی کسی خدمتا گودن. تارئخ میان ایتا طبقه خاص ژاپون جامعه میان بید. و تا قبل سال 1868 (میجی دوره آغاز) ایتا جرگه جه خلابرا ژاپون میان بید. سامورایی اول بار سیاسی هرج و مرج زمات میان (1573-467) موهم بوبوستید. اَ زمات میان ژاپون اورشین و هش پرکه بوبوسته بو. و هر کی خوره ، ایتا صارا میان حکومت گودی. اَ تارئخی دوره میان، ژاپون دورون پور جنگانی اتفاق دکفته و نیاز به سامورایین ویشترا بوسته. و سامورایین قوت بیگیفتید. قرن دوازدهم میان ، دو تا کوگا ژاپون دورون قودرت دشتید. ایتا تایرا کوگا و اویتا میناموتو کوگا نام دَشتید. اَ دوتا کوگا ویشتر ژاپون صارای مالک بید و دِه کوگاهانه امره پور جنگ بیگیفتید. سال 1192 میان، میناموتو یوریتومو موفق بوبوسته ایتا سیاسی قودرت بدست باوره و شوگون جاجیگاه خوره چکونه. یوریتومو کاماکورا سلسله تاسیس بوگوده و اونه فرمانروایی کل ژاپون میان آغاز بوبوسته. یوریتومو حکومت دورون ، قودرت و مناصب سامورایین دست دکفته و اوشان ژاپون جامعه میان قوت بیگیفتید. قرن شانزدهم میان، تویوتومی هیده یوشی (1537-1598) شوگون بوبوسته. وی پور جنگان دورون پیروزا بوسته و خو رقیبان دشکنه و همه کوگاهانه فوگوردانه و ژاپون ایجایی وجود باورده. وی ایتا طبقاتی نظام وجود باورده و دستور بدا فقط سامورایین تَنید خوشانه شمشیرانه بدارید. سابقه زمات ، وختی جنگ نوبو و صلح و آرامش بو سامورایین کشاورزی یا گاکلف کار گودید. هیده یوشی بوگفته کی سامورایین دو راه ویشتر نرید. یا کشاورز ببید یا سامورایی بمانیدو هیده یوشی ، سامورایین فرمان بدا کی یا کشاورزی و بجارکار بوکونید یا اگر تصمیم بیگیفتید سامورایی بِبید قلایان و کله بستان دورون، خلابران (سامورایین) امره زندگی بوکونید. هیده یوشی قانونی وضع بوگوده کی فقط سامورایین تَنستید خوشانه امره شمشیر بدارید. اَ طبقاتی نظام کی هیده یوشی چگوده ، توگوگاوا ایاسو زمات قوت بیگیفته.

ادو دوره میان (1603-1867)، سامورایین، ژاپون جامعه میان ، جاجیگاه خاص و موهم بوبوسته. قبلا ایتا سامورایی ، کشاورزی ، پیله وری یا بازار مجی گودی. سامورایین همه تان ایتا کله بست یا قلا دورون زندگی گودید و اوشانه ارباب و چکنه اوشانه دس اوکوف بج فدایی. باخی سامورایین ایسا بید کی ارباب یا چکنه نَشتید. اوشانه دوخادید رونین. اَ جرگه کی اوشانه نام رونین بو ادو دوره میان پور مشکل وجود آوردید و آشوب و جنایت گودید. بعد سکی گاهارا جنگ کی توگوگاوا کوگا پیروز بوبوسته و بِتنستید خوشانه رقیبان دَشکن بدید. ژاپون 250 سال صلح میان دوارسته و توگوگاوا کوگا واستی دوشمن و رقیب نیسا بو و همه تان جوخوفته بید. هنه واستی جنگستن اهمیت و مبارزه چم کم بوبوسته. پور سامورایین سیاستمدار ، معلم و بازیگر بوبوستید. وختی میجی دوره فرسه و فئودالیسم، ژاپون میان خاتمه بیگیفته ، سامورایی همیشک واستی ژاپون جامعه میان دیمه بوبوسته.

ژاپونی شمشیر

ایتا شمشیر ایسه کی مردوم اونه امره سامورایی شناسید. قدیمی ژاپونی شمشیر (نارا دوره) میان، چوکوتو نام بو کی پهن تیغه دَشتی. ولی نهصد سال اخیر میان، ویشتر شمشیرانه خمیده تیغه امره چکودید. اَ شمشیران نام (اوچیگاناتا) و (کاتانا) بو. اؤ زمات میان، دوتا کوچی شمشیر وجود دَشتی کی اوشانه نام واکیزاشی و تانتو بو. پور سامورایین دوتایی هم کوچی و هم پیله شمشیر استفاده گودید. مثلا هم واکیزاشی و هم کاتانا. ترکیب اَ دو شمشیر سامورایی سمبل بوبوسته. وختی ایتا سامورایی دوتا شمیشیر ایتا دراز و ایتا پاچ اوسادی خو امره حمل گودی اصطلاحا دوخادید دایشو ( یعنی پیله دانه و کوچی دانه). ادو دوره میان فقط سامورایین اجازه دَشتید دایشو بدارید.

یومی

یومی ژاپونی پیله کمان نام ایسه. یومی سنگوكو دوره جه رونق بکفته. وختی سامورایین تانگاشیما توفنگ (ژاپونی شمخال توفنگ) امره آشنا بوبوستید. ولی پور سامورایین ایسا بید یومی امره به عنوان ایتا ورزش استفاده گودید. یومی جنس چگوده بوبوسته بو جه بامبو ، چوب ، نی و چرم. یومی تیر پینجاه تا صد متر بیگاده بوستی. یومی امره اسب سر ویشتر تیر بیگادید کی اَ ورزشه دوخادید یابوسامه.

پیشدار

ژاپونی پیشداران نام یاری و ناگیناتا ایسه کی سامورایین اوشانه امره جنگستید. یاری ایتا پیشدار بو کی جنگ دورون ناگیناتا جانشین بوبوسته. یاری ایتا انفرادی سلاح نوبو بلکه ایتا جرگه آشیگارو (پیاده خلابر) کی داوطلب سرباز بید اونه امره جنگستید. پیش دار نسبت به شمشیر جنگ دورون ویشتر تاثیر بنا و جنگستن موقع پیاده و سواره جولو بختر بو.

انسان دا جوڑ دد پلان والے دو پیراں تے چلن والے بن مانساں دے ٹبر نال اے تے سدھا کھلونا، دوپیراں تے چلنا، ہتھاں تے اوزاراں دا ودھیا ورتن، اک وڈا دماغ تے پؤں وٹیاں رہتلاں ایہدیاں اچیچیاں صفتاں نیں۔ ڈی این اے تے پرانیاں ہڈیاں دسدیاں نیں جے انسان چڑدے افریقہ دے پاسے توں 200000 ورے پیلاں ٹریا تے 50000 ورے پہلے اپنے ہندے نویں پنڈے تے سوچ دے وکھالے تے اپڑیا[1]۔ دوجے آگو مانس نوں سامنے رکھو تے ایدے کول اک بہت ودیا تے وڈا دماغ اے۔



انسان سوچ سکدا اے، بولی بول سکدا اے، تے رپھڑ مکاسکدا اے۔ سوچن دے ایس ودویں ول نال اوس کول اک سدا کھڑا پنڈا اے تے ازاد ہتھ نیں جناں نال اوہ کئی کم کر سکدا اے تے کسے وی ہور جیون دے مقابلے وچ ودیا اوزار بنا تے چنگے ول نال ورت سکدا اے۔ ہور وی اچی پدھر دیاں سوچاں جیویں اپنی پچھان، عقل تے سیانف[2] اوہ گلاں نین جیہڑیاں کہ اوہنوں 'بندہ' بنادیاں نیں۔ انسان سواۓ انٹارکٹکا دے سارے براعظماں تے ریندا اے۔ زمین تے انسانی گنتی ست ارب توں ود اے[3]۔

اج دے انسان دا ٹانچہ تے پنڈا پرانا سیانا مانس توں وشکارلے پتھ ویلے وچ 200،000 ورے پہلاں افریقہ رچ وکھالے وچ آیا[5]۔ تیسرے پتھر ویلے دے مڈ تے 50،000 ورے پہلے اپنی ہن دی سوچ تے رہتل ؛ اپنی بولی تے موسیقی نال انسان نے اپنا آپ دسیا۔ افریقہ وجوں انسان 70،000 ورے پہلے نکلیا تے ساری دنیا تے پھیل گیا تے اپنے توں پہلے دے پرانے انسانی مانساں نوں مکادتا۔ 40,000 وریاں وچ اوہ ایشیاء یورپ تے اوشیانا تے پھیل گیا14،500 ورے پہلے دے نیڑے تریڑے اوہ اتلے تے دکھنی امریکہ تک اپڑ گیا[6]۔



10000 ورے پہلے تک انسان دا گزارا شکار تے سی۔ وائی بیجی دے ٹرن نال انسان کول کھان پین نوں فالتو شیواں جیدے نال بپار تے کاروبار ٹرے۔ دھاتاں دے اوزاراں دا بنن تے ورتن ٹریا۔ ڈنگراں نوں پھڑ کے پالیا جان لگیا۔ 6000 ورے پہلے مصر عراق تے پنجاب وچ رہتلاں تے سرکاراں دی نیو پئی۔ فوجاں دی بنان دی بچاؤ لئی لوڑ پئی تے دیس دا پربندھ چلان لئی ایس کم دے لوک رکھے گۓ۔ کم دیاں شیواں لئی دیس اک دوجے دا ہتھ وی ونڈاندے سن تے ایس لئی لڑائیاں وی لڑیاں گیاں۔



پرانا یونان اک مڈلی رہتل سی جتھوں لوکراج، فلاسفی سائنس اولمپک کھیڈاں لکھتاں تے ڈرامہ 2000 توں 3000 ورے پہلے ٹرے[7]۔ لیندے ایشیاء توں یہودیت تے ہندستان توں ھندو مت وڈیاں تے مڈلیاں مزہبی سوچاں سن۔ پچہلے 500 وریاں وچ یورپ وچ پرنٹنگ پریس دے بنن ہور نویاں چیزاں بنن نے اک انقلاب لے آندا۔ یورپی کھوجیاں نے دنیا کھنگالدتی تے ساری دنیا تے یورپ نے مل مار لیا۔



20ویں صدی دے انت تے انفارمیشن ویلے دا مڈ بجن تے دھیا اک نویں ویلے وچ آگئی ت ے اک دوجے نال جڑگئی۔ ایہ دنیا ہن اک پنڈ وانگوں اے۔ ایس ویلے 2 ارب دے نیڑے انسان انٹرنیٹ نال اک دوجے نال جڑے ہوۓ نیں[8]۔ 3۔3 ارب لوک موبائل فون ورت رۓ نیں[9]۔

انسان دا قد نسل جغرافیہ کھان پین تے کم نال جڑیا اے۔ اک جوان وشکارلے ناپ دے انسان دا قد 1.5 to 1.8 میٹر (5 to 6 فٹ) ہو سکدا اے اک انسانی مادھ دا جوکھ 54–64 کلوگرام تے نر دا 76–83 کلوگرام ہوندا اے[10][11]۔



جنا زنانی دے ملن نال زنانی یا مادہ اندر نیانے دے ہون دا کم ٹردا اے تے 9 مہینیاں مگروں اوہ جمدا اے۔ انسان دے نیاے دا جمنا دوجے بن مانساں نالوں چوکھے درد والا ہوندا اے تے ایدے نال موت وی ہوسکدی اے[12]۔ ایدا وجہ انسانی نیانے دے سر دا وڈا ہونا تے دو پیراں تے چلن باجوں زنانہ پیلوس ہڈیاں دا تنگ ہونا اے[13][14]۔ نیانہ جمن تے زنانیاں دے مرن دی گنتی غریب دیساں وچ زیادہ اے[15]۔ امیر دیساں وچ جمے نیانے دا جوکھ 3–4 کلو (6–9 پاؤنڈ) تے قد 50–60 سینٹی میٹر (20–24 انچ) ہوندا اے۔ عریب دیساں وچ کاکے دا تھوڑا جوکھ ہونا اوہناں دے مرن دی وڈی کارن اے۔ انسانی نیانے جمن ویلے بالکل بے بس ہوندے نیں تے 12 توں 15 وریاں دی عمر وچ جنسی ناپ نال وڈے ہوندے نیں۔ انسان دے جین دی عمر وکھرے دیساں وچ وکھری اے۔ امیر دیساں وچ زیادہ تے غریب دیساں وچ تھوڑی۔ ہانگ کانگ وچ جنیا دی ایورج عمر 78.9 تے زنانیاں دی 84.8 اے۔ سوازی لینڈ وچ ایہ 31.3 اے[16]۔ اک فرانسیسی سوانی جین کالمنٹ نوں 122 ورے دی عمر نال سب توں زیادہ عمر دی انسان منیا گیا اے[17]۔

ڪراچي (اردو: کراچی، انگريزي: Karachi) پاڪستان جو سڀ کان وڏو شھر ۽ صنعتي، تجارتي، تعليمي، مواصلاتي و اقتصادي مرڪز آھي۔ ڪراچيءَ جي رهواسيءَ کي ڪراچيائيٽ سڏيو ويندو آهي. ڪراچي دنيا جو ٻيون وڏو شھر آھي[3]۔ ڪراچي پاڪستان جي صوبي سنڌ جو گادي جو ھنڌ آھي۔ ڪراچي شهر وڏي سمنڊ جي ساحل تي سنڌو ٽڪور جي اتر اولهه ۾ قائم آهي۔ پاڪستان جي سڀ کان وڏي بندرگاھ ۽ ھوائي اڏو بہ ڪراچي ۾ قائم آھي۔ ڪراچي 1947ء کان 1960ء تائين پاڪستان جو گاديء جو ھنڌ بہ رھيو۔ ڪراچي شهر نه رڳو سنڌ جو ڪاروباري مرڪز آهي، پر خطي ۾ هڪ اهم بندر پڻ آهي.

موجوده ڪراچي جي جڳهه تي واقع قديم ماهي گيرن جي بستين مان هڪ جو نال ڪولاچي جو ڳوٺ هيو۔ انگريزن اڻويهين صدي ۾ هن شهر جي تعمير و ترقي جون بنيادون وڌيون۔ 1947ء ۾ پاڪستان جي آزادي جي وقت ڪراچی کي نو آموز ملڪ جو گاديء جو هنڌ منتخب ڪيو ويو۔ ان جي وجه سان شهر ۾ لکين مهاجر جو دخول ٿيو۔ پاڪستان جي گاديء جو هنڌ ۽ بين الاقوامي بندرگاهه هئڻ جي وجهه سان شهر ۾ صنعتي سرگرميون ٻين شهرن کان پهريان شروع ٿي ويون۔ 1959ء ۾ پاڪستان جي گاديء جو هنڌ اسلام آباد منتقلي جي باوجود ڪراچي جي آبادي ۽ معيشت ۾ ترقی جي رفتار گهٽ ناهي ٿي۔ . پوري پاڪستان مان ماڻهون روزگار جي تلاش ۾ ڪراچي ايندا آهن ۽ ان جي وجه سان هتي مختلف مذهبي، نسلي ۽ لساني گروهه آباد آهن۔ ڪراچي کي ان وجه سان مني پاڪستان (Mini Pakistan)/ننڍو پاڪستان بہ چيو ويندو آهي۔ ان گروهن جي باهمي ڪشيدگي جي وجه سان 80 ۽ 90 جي ڏهائين ۾ ڪراچي لساني فسادن، تشدد ۽ دهشت گردي جو شڪار رهيو۔ بگڙندڙ حالتن کي سنڀالڻ جي لاء پاڪ فوج کي به ڪراچی ۾ مداخلت ڪرڻي پئي۔ ايڪويهين صدي ۾ تيز قومي معاشي ترقي سان گڏ و گڏ ڪراچي جي حالتن ۾ تمام تبديلي آئي آهي۔ ڪراچي جي امن عامه جي صورتحال ڪافي بهتر ٿي آهي ۽ شهر ۾ مختلف شعبن ۾ ترقي جي رفتار ۾ تمام اضافو ٿيو آهي۔ ڪراچي درياء سنڌ جي ڏهاني جي اترين حد تي واقع آهي۔ شهر هڪ قدرتي بندرگاهه جي گرد وجود پايو۔ ڪراچي 52´ 24° اتر ۽ 03´ 67° اوڀر تي واقع آهي۔

سنڌو ٽڪور اتي ٺهي ٿي جتي سنڌو درياءُ عربي سمنڊ ۾ ڇوڙ ڪري ٿو. هيءَ ٽڪور 16 هزار چورس ميلن يعني 41 هزار 4 سَو 40 چورس ڪلوميٽرن جي ايراضيءَ تي پکڙيل آهي، ۽ ان جو سمنڊ سان دنگ لڳ ڀڳ 130 ميل ڊگھو آهي.

سموري ٽڪور تي ڪيتريون ننڍيون ننڍيون وسنديون آباد آهن پر ڪو به وڏو شهر آباد نه آهي. ٽڪور کي ويجھي ۾ ويجھا شهر ٺٽو ۽ سنڌ جو وڏو ۾ وڏو شهر ڪراچي اولهه طرف آهن. سنڌ جو ٻيون وڏي ۾ وڏو شهر حيدرآباد سنڌو ٽڪور کان 130 ميلن جي فاصلي تي اتر ۾ واقع آهي.

سنڌو ٽڪور تي سراسري گرمي پد جُولاءِ جي مهيني ۾ 70 کان 85 درجا فاهانائيٽ جڏهن ته جنوريءَ ۾ 50 کان 70 درجا فارهانائيٽ رهندو آهي. ڪنهن عام سال ۾ ان ٽڪور تي 10 کان 20 انچ وسڪارو ٿيندو آهي.

هن وقت سنڌو درياءَ مان تازو پاڻِ نه ملڻ ڪري سنڌو ٽڪور ماحولياتي دٻاءُ جو شڪار آهي. سنڌو ٽڪور جا تمر ٻيلا ۽ ان ۾ پلجندڙ سموري جنگلي جيوت توڙِي آبي جيوت جيئدان جي جنگ هارائيندي نظر اچي رهيا آهن. سنڌو ٽڪور جا تمر ٻيلا جيڪي اڳي ooo هيڪٽرن تي پکڙيل هئا، سي هائي رڳو ooo هيٽرن تائين محدود رهجي ويا آهن. ساڳيءَ ريت جھينگي ۽ پلي جو نسل پڻ خطري ۾ آهي ۽ ان ڪري نازڪ مرحلي مان گذري رهيو آهي جو سندن جياپو سمنڊ جي کاري پاڻي ۽ سنڌو درياءَ جي مٺي پاڻيءَ جي ميلاپ تي دارومدار رکي ٿو.

توهان اهڙي صفحي جو ڳنڍڻو وٺي هتي پهتا آهيو، جيڪو اڃا وجود نه ٿو رکي. اهڙو صفحو جوڙڻ لاءِ هيٺين باڪس ۾ ٽائيپ ڪرڻ شروع ڪريو (امدادي صفحو ڏسندا). جي توهان هتي غلطيءَ ۾ اچي ويا آهيو ته رڳو پنهنجي جهانگُوءَ جو back بٽڻ ڪلڪ ڪندا.

خبردار: توهان لاگ اِن ٿيل ناهيو. هن صفحي جي سوانح ۾ توهان جو آءِ پي پتو درج ڪيو ويندو.

سنڌ، جنهن جو صحيح اچار ”سنڌو“ آهي، سو شروعاتي طرح اُنهيءَ نديءَ جو نالو آهي، جيڪا الهندن ملڪن ۾ ”انڊس“ (Indus) جي نالي سان سڃاتي وڃي ٿي. اِهو غلط اُچار انهيءَ ڪري قائم رهيو ، جو اُهو سڪندر اعظم جي ساٿين سندس ڪاهن جي تذڪرن ۾، ڪتب آندو هو، ۽ اُهو پوءِ عام استعمال ۾ اچي ويو. مشرقي اًچارن ۾اهڙي ڦيرگهير واقع ٿيندي رهندي آهي. حضرت عيسيٰ عليه الاسلام کان هڪ صدي پوءِ جوڙيل هڪ ڪتاب ۾ ”سنٿاس“(Sinthos) نالو آيل آهي، جيڪو اصل ۾ ڪجهه وڌيڪ قريب آهي. بعد “ ۾ ”سنڌ“ جو نالو اُنهي ملڪ تي پيو، جنهن کي سنڌو نديءَ جو هيٺيون وهڪرو رجائي ٿو، يعني اهو ملڪ اُنهن ڏاني ندين جي سنگم کان هيٺ تي آهي، جن جي گڏيل نالي تي ۽ ان کان مٿي واري ملڪ کي ”پنجاب“ سڏيو ويو، جنهن منجهان اُهي وهيون ٿي. اها ڳالهه بلڪل مناسب آهي، جو سنڌ جي سر زمين درياهه شاهه جي نالي پٺيان سڏجي، جو ان کي سرجيو ئي درياهه شاهه آهي. ان سموري ڏيهه کي، جيڪو ڊيگهه ۾ سوين ميل ۽ ويڪر ۾ سو ميلن تائين آهي، سنڌوءَ جي پاڻي ڪڻو ڪڻو ڪري آڻي تهه مٿان تهه رکي جوڙيو ۽ ٿانيڪو ڪيو آهي. جنهن پڻيءَ کي اڄ اسان جا پير لتاڙين ٿا، سا اُها ساڳي پئي آهي، جنهن کي موهن جي دڙي وارا ۽ سندن وڏا لتاڙيندا هئا: يعني سنڌوءَ جو لوڙهي آندل لَٽُ، جنهن کي گرميءَ، پاڻيءَ ۽ ساوڪ جي تاثير سڌاري سنواري، هڪ سنئين سڌي ميدان جي شڪل ۾ ماٿريءَ جي ڇيڙن تائين ائين پٿاري ڇڏيو آهي، ڄڻ ته ان جو ڪو ڇيهه ئي نه هجي فرق رڳو ايترو آهي ته جنهن ڌرتيءَ کان هو واقف هئا، تنهن جو مٿاڇرو ڪي پنجاهه صديون اڳ اڄوڪي مٿاڇري کان ڪجهه فٽ هيٺ هو ۽ ان جو سمنڊ سان ميلاپ اڄوڪي هنڌ کان ڪافي اندر ٿي ٿيو، باقي ان جي مٿاڇري جو مهانڊو، ان جا هڪ يا ٻئي پاسي اڻ لکا لاهه اڄوڪي چٽي ۽ بلڪل واضع مهانڊي کان ٿي سگهي ٿو ته ڪجهه ڦريل هجن. پر اهو فرق به خالي اکين سان ڏسي نه سگهبو. سنڌ جي ان ميدان جي وڌيڪ باريڪ جانچ ڪبي، پر پهريائين ڀر وارين سر زمينن تي نظر ڦيرائي وٺون.

سنڌوءَ جي ويڪريءَ ماٿريءَ جي کاٻي پاسي، يعني اوڀر کان، هندستان جو وسيع ريگستان، ٿر، آهي. جنهن ۾ لاڳيتو ٽي سوء ميل سڃ ئي سڃ آهي ۽ جيڪو هري هري چڙهندو، وڃي ابو ٽڪريءَ ۽ اوالي ٽڪرن تي کٽي ٿو. جمڙائوءَ جي منڍ واريءَ ويڪرائي ڦاڪ جي ڏکڻ کان ويندي ڪڇ جي رڻ تائين ٻنهي علائقن جي وچ وارو ٽڪر ائين واضع ۽ چٽو آهي، جيئن سمنڊ جو ڪپر- واريءَ جا دڙا ائين اوچتو نروار ٿين ٿا، جيئن ڪنهن سڌي سنواٽي ميدان ۾ ڇپن جون قطارون، پريان اُتر ۾، واريءَ جون ڀٽون ڪچي ۾گهڙي اچن ٿيون، ۽ سنڌوءَ جي ليٽ وارياسي علائقي ۾ ڪافي اندر تائين هلي وڃي ٿي. اهڙيءَ طرح، اُنهن ٻنهي ٽڪرن جي وچ ۾، جيڪي طبعي لحاظ کان مختلف مُول منڍ وارا آهن، اصل ويڇو گهڻي ڀاڱي ميسارجيو وڃي. ائين کڻي چئجي ته ٿلهي ليکي، ڪچي کان ٿر جو ويڇو هن علائقي ۾ ڪجهه وڌيڪ اوڀر ڏانهن آهي. ڏاکڻي ۽ اترئين ڀاڱي ۾ وارياسي جي سنڌي جو هيءَ فرق ڏکڻ- اولهه جي چوماسي جي ٻل نٻل ۽ گهاٽيءَ واڌيءَ جي ڪري ڦرندو گهرندو رهي ٿو. ساڄي يا اُلهندي پاسي کان اصل سنڌ جون حدون انهن ڪڪرالين ماٿرين جي ڇيڙي جي سنوت ۾ نظر اينديون، جيڪي بلوچستان جي اندر ڦلهجي ويندڙ ٽاڪرو علائقي تائين هليون وڃن ٿيون. پرحقيقت ۾ ان مٿانهين پٽ سان هڪدم لڳولڳ زمين ٽڪرين تان وهي آيل لٽ منجهان ٺهيل آهي. اها سنڌوءَ جي لٽ کان بلڪل نرالي ٿئي ٿي. اهو تر جنهن کي مڪاني طرح ”ڪاڇو“ سڏيو وڃي ٿو ۽ گهڻي ڀاڱي سوڙهو آهي، سو سليمان جبل ۽ بولان لڪَ کان ڏکڻ طرف ويندڙ جبلن جي وچ ۾، ڊيگهه ۾ وڌندي، ڪيترن هزارين چورس ميلن جو علائقو ٺاهي ٿو. ان جو جيڪو ڀاڱو سنڌوءَ واري ڪچي کي ويجهي ۾ ويجهو آهي، سو چيڪي مٽيءَ جهڙيءَ مٽيءَ جو تراکڙو بيابان آهي، جتي سبزو ۽ ساوڪ نه هئڻ جي برابر آهن. اهو بيابان هڪ قدرتي روڪ آهي- اهڙي ئي اڏول، جهڙا اُهي جبل، جيڪي ان طرف جون ٻيون حدون ٺاهين ٿا. ان بيابان جي ڪري ئي ڪڇيءَ جو هيءَ علائقو گهڻو ڪري سياسي طرح ڏکڻ اوڀر وارين ايراضين جي بدران اُتر وارين ايراضين سان لاڳاپيل سمجهبو رهيو آهي. سنڌ جي ڪچي جي اولهه ۾ جيڪو جابلو ٽڪرو آهي، سو بيهڪ ۾ اوڀر واري ريگستاني ٽڪري سان عجيب مشابهت رکي ٿو. پر ابتيءَ ترتيب سان: يعني اتريون ڀاڱو ڪڇيءَ جي ڇيڙي کان ويندي منڇر تائين بظاهر اڻ ڀڳل ڪوٽ ٺاهي ٿو، جڏهن ته منڇر کان ڏکڻ طرف ڌار ڌار ٽاڪرو ڇپر نظر ايندا، جن جي وچ ۾ سئين پٽ جون ايراضيون آهن، ۽ سمورو تر ڏکڻ ڏانهن هلندي، هوريان هوريان ويڪر ۾ وڌندو ۽ اُچارئيءَ ۾ گهٽبو وڃي ٿو ۽ اولهه پاسي ڪا ظاهري حد ٺاهيندو نظر نٿو اچي. اُنهيءَ پاسي وڌندي، حب ندي ۽ ڪي ننڍيون ٽڪريون اُڪري، اسين لس جي ميدان ۾ پهچون ٿا، جيڪو سمنڊ جي ڪپر کان ساٺيڪو ميل اُتر طرف هليو وڃي ٿو ۽ جنهن جي وڌ ۾ وڌ ويڪر انهيءَ مفاصلي جي اڌ جيتري ٿيندي. اهو ميدان پورالي نديءَ ۽ ڪن ننڍين نين جي آندل لٽ منجهان جُڙيو آهي. ۽ مٿي ڄاڻيل ڪڇيءَ جي ميدان وانگر، جنهن سان اهو گهڻي مشابهت رکي ٿو، ڪڏهن ڪڏهن سياسي طرح سنڌ سان لاڳاپيل پر گهڻو ڪري ان کان ڌار رهيو آهي. هيءُ انهن علائقن جو مختصر بيان آهي، جن جون ڪڏهن ننڍيون ڪڏهن وڏيون ايراضيون اُن سر زمين سان شامل رهيو آهن، جنهن کي سياسي طور ”سنڌ“ جي نالي سان ياد ڪيو وڃي ٿو. ائين چئي سگهجي ٿو ته طبعي لحاظ کان سنڌ جي اوڀر وارا وارياسا پٽ راجپوتانا جو ۽ اولهه وارا ننڍا ٽَڪر بلوچستان جو حصو آهن. تحقيق، الهندي ڪوهستان يعني ڪاڇي جا ۽ اڀرندي ريگستان يعني ٿر جا رهواسي ”سنڌ ڏي هلڻ“ جي ڳالهه ڪندا آهن: هنن وٽ سنڌ جي معنيٰ اهائي اصل واري آهي، يعني اها سر زمين، جنهن کي سنڌو نديءَ ٺاهيو ۽ سدا تاتيو آهي. اڄ جيڪو پرڳڻو ان نالي سان اسان جي سامهون آهي، تنهن جي بيهڪ 27 ڊگريون 30 منٽ ۽ 23 ڊگريون 35 منٽ اُتر ويڪرائي ڦاڪ ۽ 66 ڊگريون 42 منٽ ۽ 71 ڊگريون 10 منٽ اوڀر ڊگهائي ڦاڪ جي وچ ۾ آهي. شروع شروع ۾ جيڪي انگريز سنڌ ۾ آيا هئا، تن کي ان کي ”ننڍو مصر“ سڏيو هو. جيتوڻيڪ هاڻ اسان کي ان تشبيهه تي مٺيان لڳي سگهي ٿي، ڇو ته سنڌ تهذيب کي مصر کان عمر ۾ ايترو ننڍو سمجهيو ٿي ويو، تڏهن به نيل ۽ سنڌوءَ جي هيٺانهين ماٿرين جي وچ ۾ حيرت جهڙي مشابهت آهي: ٻنهي ۾ ساڳيا ئي ٽي پوو-وڇوٽ ٽڪرا- بُٺ ٽڪر، ڪچو ۽ وارياسو- ساڳيءَ ترتيب ۾ ساڄي کان کاٻي ڏسڻ ۾ اچن ٿا. ٻنهي ۾ وچ واريءَ ماٿريءَ جي زرخيزيءَ جو دارومدار برسات تي نه پر سالياني ٻوڏ تي آهي. آبهوا ۽ زمين جي خيال کان ٻنهي ملڪن اندر نباتات ۽ حيوانات جي جيتري هڪجهڙائي آهي، تيترو فرق ناهي. ڪو سنڌ واسي موٽر رستي سوئيز کان قاهري ۽ اتان نيل ڊيلٽا جو ڪجهه ڀاڱو لتاڙي، لبيا جي ريگستان اندر ويندو، ته کيس منزل بمنزل ايتري مشابهت ڏسڻ ۾ ايندي، جو هو سمجهندو ته ريل تي ڪراچيءَ کان حيدرآباد رستي مارواڙ پيو وڃان. البت کيس نيل ڊيلٽا کان مٿي وارو تر سنڌو ماٿريءَ جي ڀيٽ ۾ ڪجهه سوڙهو لڳندو: ۽ حقيقت به اهائي آهي. حيدرآباد جي ويڪرائي ڦاڪ کان مٿي ٽن سون ميلن تائين درياهه سان گوني ڪنڊ ٺاهيندي، ڪيتريون به ماپون ڪبيون، ته ڪٿي به سنڌو ماٿريءَ جي ويڪر سٺ ميلن کان گهٽ نه ملندي.

اباسین سېلاب د کونړ ولایت د وټپور ولسوالۍ په یوه روشن فکره پښتنه کورنۍ کې له نن نه ۲۵ کاله مخکې دې نړۍ ته سترگې پرانیستي. کله چې افغانان جگړو وڅپل او دې ته یې اړ کړل؛ څو د سر پنا لپاره له هېواده بهر لاړ شي، د سېلاب پلار هم دې ته اړ شو چې گاونډي هېواد پاکستان ته کډه وکړي. د وړکتوب خاپوړې یې یې هلته وکړې او له هغې وروسته، یې پلار د یو ښه را تلونکي د جوړولو په خاطر په یو ښونځي کې ورته داخله واخیسته. کله چې د ښونځي له لومړي پړاوه ووت یعنې د منځی ښونځی یې پیل شو، ورسره یې په ښونځي کې د معارف پېژندې، ستاینې، او داسې نورې ترانې، نعتونه او د نورو شاعرانو شعرونه به یې د سټیچ په سر ویلې. کله نا کله به یې د سټیچ نطاقي هم کوله، چې د ښې کارکردگې له امله یې ښوونکوي د یو تکړه او با استعداده شاگر په سترگو ورته کتل. د ورځو په تېرېدو سره اباسین سېلاب په د ترانې په گروپ کې لا پرمختگ وکړ، د ترانې سرټیم او له هغې وروسته د ټول سټیچ کنترولوونکی او جوړونکی شو. له هغه وخته یې دې شعر له لوستلو سره ډېره مینه پېدا شوه، د ډېرو ښاغلو شاعرانو شعرونه به یې لوستل او خوند به یې ترې اخیسته. په همدې ترتیب یې ښونځی تر دولسم ټولگي پورې ورساوه او د فراغت سند یې تر لاسه کړ. کله چې بېرته خپل گران هېواد افغانستان ته را ستانه شول، د پخوا په څېر یې له شعر او شاعرې سره مینه وه، د سندرو اورېدلو ته به یې هم ډېر وخت ورکاوه. د وخت په تېرېدو سره یې په رسنیو کې کار پیل کړ. د نورو کارنو ترڅنگ یې ځینې تفرېحي خپرونې او ورسره یې د شعرونو په دیکلومه کولو کې هم ونډه اخیسته. د شعر په دېکلومه کولو کې ډېره ښه وړتیا لري، خدای ج ورته د ښې څېرې سره ښکلې ځواني او ښکلی اواز هم ورکړی، چې د اواز او لیدلو مینه والو یې خورا زیات دي. له همدې ځایه وو چې د شعر او شاعرې ډگر ته یې را ودانگل، ډېرې شعرونه او غزلې یې ولیکلې، خو په وینا یې زړه یې پرې اوبه نه دې څښلي، څو یې د کتاب چاپلو لپاره انتخاب کړي. خو اراده لري چې په راتلونکي کې به ضرور چاپ شوي اثار ولري، یوازې دومره وخت پاتې چې خپل شعرونه د شعر په ټگر کې وتلي او تل شاعرانو او استادانو ته وښایي. د هغوې د خوښې مطابق به انتخاب کوي، لوستونکو او مینه والو ته به یې وړاندې کوي. د تلویزیون په پرده ډېر ځله د شعر په دیکلومه کولو، او نورو تفریحي خپرونو په وړاندې کولو سره لیدل شوی. له اکټ کارې سره یې ځانگړې مینه وه، چې دې مینې او علاقې تر دې را ورساوه، چې باید په راتلونکی کې په فلمو نو او ډرامو کې هم کار پیل کړي، او فعاله ونډه پکې واخلي. له فلمي ستورو او د هغوې له حرکتونو سره یې زړه پورې توب خورا زیات وو، همېشه به یې هڅه کولا له ځانه سره داسې اکټونه وکړي، څو په دې مطمین شي، چې کولای شي په فلمونو او ډارمو کې هم کارو کړي، لیدونکو او ننداره کونکو ته ځان د یو تکړه اکټر په بڼه رو وپېژني نه یوازې فلمونو او سریالونو سره مینه یې وه ورسره د روانو حالاتو انځورونو هم دې ته و هاڅاوه، چې بالاخره د فلمي حرکاتو په لوبولو سره هېوادو الو ته د وخت انځورنه روښانه کړي او د خلاصي لارې چارې یې ورته په گوته کړي. سېلاب له دې وروسته په پېښور کې د په دوه ډرامو کې کار وکړ، دا چې د کار کولوو او شوټینگ پایلې یې تر ډېره ورته مثبتې وې، د لیدونکو لخوا ورته د ستاینې صحنه څرگنده شوه، ښه به جورت سره یې وویل چې زه به نور د فلم په جوړولو کې هېڅ ځنډ نه کوم. د ملاله او سوره په نامه یې په دوه تلویزیوني سریالونو کې کار کړی، په لیدو یې په مینه والو کې خورا زیات والې را غلی. له دې وروسته یې د ښه کار په پایله دا ورته څرگنده شوه چې اوس کولای شي په هر ډول فلم کې هره صنحه چې ورک ول کېږي په بریالیتوب سره یې تر سره کړي، نه یوازې دا چې په سمه توگه به یې ترسره کړې وي. ورسره به د مینه والو لخوا ډېر وستایل شي. اباسین سېلاب وايي تر اوسه د افغانستان د سینمایي فلمونو لپاره هېڅ کار نه دی شوې، سېلاب هیله لري څو په را تلونکي کې د خپل هېواد سینما ته هم پام وکړي، څو د نورهېوادونو په شان زمونږ د فلم هنر هم وده وکړي. اباسین سېلاب چې غواړي یو تکړه او مشهور فلمي ستوری شي، د تلویزیوني سریالونو سره زیاته علاقه لري چې پکې ولوبېږي. په سینمايي فلمونو کې لوبېدل خو ورته د خوب ریښتیا کېدل ښکاري او هڅه کوي، څو په هر قېمت چې وي په سینمايي فلم کې د یو غوره اگټر په بڼه کار وکړي. په را تلونکي نژدې وخت کې به په دې وتوانېږي چې د افغانستان لپاره هم سینمايي فلمونه جوړ کړي. سېلاب په دې هڅه کې دی چې څنگه وکولای شي، د افغانستان سینما ته وده ور کړي. او په را تلونکي کې په افغانستان کې د ښه سینمايي فلمونو ننداره کول پیل او ډېر مینه والو را پیدا کړي. د نومړي د خوښې فلموي ستوري په هندي فلمونو کې شاروخان او په پښتو فلمونو کې، جهانگیر خان په گوته کوي. د هغوی ټول اگټونه ورته په زړه پورې دي، کله یې چې نوی فلم را شي د دوی په خاطر یې ضرور گوري، خوند او پند ترې اخلي. او کولای شي په ډېر کم وخت حتی په یوځل لیدو هم د دوی تقلید له ځانه سره وکړي. خو اباسین سېلاب وايي. دا یې یو لوی ارمان دی چې په فلمي نړۍ کې د ستوري په شان وځلېږي او مینه وال یې د اوس په پرتله لا زیات شي. ورسره د تمثیل په بارک کې هم ډېره خوشبینې ښيي او وايي: که وشول په را تلونکي وخت کې به د خپل تمثیل کولوو ته لا ډېر وخت وکړي. سېلاب اوس په شمشاد تلویزیون کې ځینې تفرحي خپرونې په مخ وړي، چې زیاته برخه یې په کې د تمثیل کول دي، او دی هم له تمثیل کولو سره ډېره مینه لري، د داسې تمثیل کولو وړتیا لري چې ډېرې تمثیل کونکي او د د تمثیل مینه وال به یې ځای و نه نیسي. په اوس وختو کې یې ډېر وخت تمثیل ته ورکړی. نه یوازې تمثیل او فلم جوړلو سره مینه او علاقه لري، ورسره د گڼ شمېر شاعرانو ښکلي نظمونه غزلې او شعرونه هم دېکلومه کوي، چې په دې سره یې لا د اواز مینه والو په څو برابره زیات شوي، تل هڅه کوي داسې انتخاب وکړي، چې له احساس ډگ پیغام ولري، تر ډېره یې خلک خوښ کړي، لومړۍ پرې د ده او بیا د اورېدونکو او مینه والو زړه پرې اوبه وڅښي. د شعر په ټوله مانا، ترڅ او ډیزاین غور کوي. ښه په شوق یې له ځانه سره په څو څو ځله تکراروي، کله چې په دې وپوهېږي چې اوس یې اورېدونکي د زړه له تله خوښوي، یو ډول اثر پرې کولای شي. مخکې له دې چې انتخاب یې کړي، له هغه شاعرانو ملگرو سره پرې مشورې کوي، چې په شعر او شعارې ډېر ښه پوهېږي، د هغوی په انتخاب او بیا یې د ویلو ترڅ په خپله روته برابروي. په داسې انداز یې وایي لکه څرنگه چې د شعر له لوستلو سره ښايي. ډېر زیات شعرونه یې دېکلومه کړي، چې اختر په نامه دېکولمه کړی شوی شعر یې ډېر زړه جزبونکی، او د دې د ښې وړتیا ثابونکی دی. د دې شعر دېکلومې په مینه والو او اورېدونکو څه بیل شان اغېز وکړ، ډېرېو خلکو د یو رېښتنې صحنې لیدل انځورول، او خورا مینه یې ورسره پېده کړه. دېته ورته ډېر نور شعرونه یې هم په ډېر لوړ او مناسب انداز سره دېکلومه کړي، چې د هر یو خوند یې د یو بل په پرتله زیات دی. د خلکو په زړونو کې یې ځای نیولی او د بیا ځل اورېدو لېوالتیا یې ښيي. همدا دی چې ډېرۍ شاعران د افغانستان جنگ له اباسین نه مشر گڼي او بیا وايي چې: دی په ښه طریقه کولای شي، د تېرو بدامنو وختونو انځورنه مینه والو ته په شه شان سره وړاندې کړي، دا ورته وښایي چې دغې ناخوالو افغانان څنگه کړولي، ورسره دا هم ورته ښايي، چې څنگه کولای شو دې جنگ جگړو ته د پای ټکی کېږدو، او سوله را ولي. لکه مخکې چې مو یادونه وکړه، اباسین سېلاب هم فلمي ستوری دی ورسره یو تکړه سندرغاړی هم دی، د سندرو ویلو پوره وړتیا لري؛ خو دا چې تر اوسه یې ډېرې کمې سندرې ویلي، علت یې د وخت نشتوالو ښي. دا چې په رسنیو کې هم کار کوي ورسره شاعر او دېکلومه کونکی هم دی، نه شي کولای په یو وخت کې ټول کارونه سرته ورسوي، که څه هم محلویشت به یې ورته جوړ کړی وي. خو وايي چې د رسمي چارو د زیاتېدو له امله د سندرو ویلو ته لاس رسی نه شي کولای. ځوان شاعر طاهر شرر صاپی یو ښکلې غزل یې په خپل خوږ اواز او د موسیقۍ په ساز او سرور سره، پوښلې اورېدونکو ته وړاندې کړه. د سندرو په هنر کې همدا سندره وه چې سېلاب یې په یو ستر شهورت ونازاوه. دا غزل چې شرر صیب: په کوم انداز لیکلی او څه انځور یې پکې نغښتې، سېلاب بیا اورېدونکو ته داسې روښان کړی، چې په همغه انداز چې شاعر به غوښتل زمزمه کړی. چې مطله یې داسې ده، ځواني دې رانه وخوړه لونگه ځوانیمرگه غنم غوندې رېبې مې ړنگه بنگه ځوانیمرکه کله چې خلکو دا سندره واورېده په زړه کې یې د اباسین سېلاب د دې هنر د پرمختگ لپاره هېلې زیاتې شوې، داسې را تونکی یې ورته انتخاباوه چې نوموړی به په ډېرو سندر غاړو کې د غوره سندرغاړي لقب خپل کړي. شاعران او هنر مندان یې د دې لومړي قدم په اخیستو ډاډ ورکوي، که دی لا د خپل دې هنر ته وخت ورکړي، بې له شکه چې هنر به یې وده کوي، پرمختگ به کوي. او خپلو اهدافو ته په ډېر ژر ورسېږي، هخه خوب چې ېې د دې هنر په اړه لیدلی وي، تر ډېره به یې تعبیر ته ورنژدې شي، حتی پوره به شي. اباسین سېلاب ۲۵ کاله عمر لري، اوس لا نوی ځوان دی، واده یې نه دی کړی؛ خو بیا هم په خپل هنر کې د پرمختگ لپاره یې پلار او ورونه ډاډگېرنه ورکوي او تشویقوي یې.

نەۋرۇز مىللىتىمىزنىڭ ئېسىل ئەنئەنىۋى بايرىمى، تەبىئەت بايرىمى، ئەمگەك بايرىمى، تەنھەرىكەت بايرىمى، ئەدەبىيات – سەنئەت بايرىمى قاتارلىق كۆپ مەنبەلىك مەدەنىيەت بايرىمى بولۇش سۈپىتى بىلەن، ئەڭ قەدىمىي ئىجتىمائىي مەدەنىيەت ھادىسىسى ھېسابلىنىدۇ. ئۇ دۇنيادا مىلاد، قۇربان ھېيت، چاغان قاتارىدا سانىلىدىغان تۆت چوڭ بايرامنىڭ بىرىدۇر.



نەۋرۇز بايرىمىنىڭ ئىران ۋە تۇران دىيارلىرىدا يېڭى يىل بايرىمى سۈپىتىدە ئۆتكۈزۈلۈۋاتقىنىغا 2600 يىلدىن ئاشقان بولۇپ، بۇ بايرامنى ئۆتكۈزىدىغان مىللەتلەرنىڭ نوپۇسى ھازىر 400 مىليوندىن ئاشىدۇ. نورۇز بايرىمى ئۆتكۈزىدىغان مىللەتلەردە شەكىللەنگەن نورۇز مەدەنىيىتى ئاشۇ مىللەتلەرنىڭ ماددىي مەدەنىيلىك بىلەن مەنىۋى مەدەنىيلىكىنىڭ تەرەققىياتىغا تۈرتكە بولۇپ كەلمەكتە.



ئېلىمىزدە قەدىمىي ۋە ئەنئەنىۋى مىللىي بايراملارنىڭ بىرى بولغان نورۇز بايرىمىنى ئۆتكۈزۈش ئۇيغۇر، قازاق، قىرغىز، ئۆزبېك، تاتار، تاجىك قاتارلىق قەدىمىي مىللەتلەرگە ئورتاقتۇر.



ئەجدادلىرىمىز قۇياش ئېتىقادچىلىقى دەۋرىدە، ‹‹ قۇياش ھارۋىدا ئولتۇرغان ئىلاھ. بىزگە كۆرۈنىدىغىنى ھارۋىنىڭ چاقى. ئىلاھ كۈندۈزى ئاسماندا مېڭىپ يەرنى يورۇتسا، كېچىسى يەرنىڭ تېگى بىلەن يەرنىڭ يەنە بىر چېتىگە ئۆتىدۇ. ئۇ يەر تېگىدىن ئۆتكۈچە جىن – ئالۋاستىلار بىلەن ئېلىشىپ، جەڭ قىلىدۇ ›› دېگەن ئەپسانىگە ئىشەنگەن ۋە كېچىسى گۈلخان يېقىپ، قۇياشنىڭ كۆتۈرۈلۈشىنى تىلەپ گۈلخاننى دائىرە قىلىپ ئۇسسۇل ئويناپ، ئىبادەت قىلغان. كېيىن ئىبادەت ۋاقتىنى تەدرىجىي قىسقارتىپ، 3 – ئاينىڭ 21 – كۈنى، يەنى كېچە بىلەن كۈندۈز تەڭلىشىدىغان كۈنگە توغرىلىغان. شۇنىڭ بىلەن، ئەنئەنىۋى نورۇز بايرىمى شەكىللەنگەن.



نەۋرۇز يېڭى يىل بايرىمىدۇر. ئەجدادلىرىمىز نورۇزنى قۇتلۇق كۈن، دەپ بىلىپ، زور تەنتەنە بىلەن ئۆتكۈزۈشكە ئادەتلەنگەن. نورۇزنىڭ كىرىش ۋاقتى شەمسىيە كالېندارى بويىچە يىل ئاخىرلىشىپ، يېڭى يىل كىرگەن كۈنگە، مىلادى كالېندارى بويىچە 3 – ئاينىڭ 21 – كۈنى، يەنى، كۈن بىلەن تۈن تەڭلەشكەن كۈنگە توغرا كېلىدۇ. شۇڭا، ئەجدادلىرىمىز بۇ كۈننى يىل بېشى قىلىپ تاللاپ، ئۇنىڭغا نورۇز (يېڭى كۈن، يىل يېڭىلانغان كۈن)، دەپ ئات قويغان.



نەۋرۇز ئەنئەنە بايرىمىدۇر. كىشىلەر باھار پەسلىگە ئۇلىشىپ، ئوۋغا چىقىش، مال – ۋارانلىرىنى كۆكلەمگە كۆچۈرۈش، تېرىقچىلىققا كىرىشىشتىن ئاۋۋال ئىش – ئوقىتىگە ئوڭۇشلۇق، بەرىكەت، تۇرمۇشىغا ئاسايىشلىق، بەخت تىلەپ بۇ بايرامنى تەنتەنىلىك ئۆتكۈزۈشكە ئادەتلەنگەن.



نەۋرۇز تەبىئەت ۋە گۈزەللىك بايرىمىدۇر. بۇ تەبىئەت قىشلىق ئۇيقۇسىدىن ئويغىنىپ جانلىنىشقا باشلىغان، ئەگىز سۈيى ئېقىپ تاغ – ئېدىرلار، دەل – دەرەخلەر كۆكىرىشكە باشلىغان، كىشىلەرگە يېڭى ھاياتلىق ئۈمىدى بەخش ئېتىلگەن مەزگىلدۇر. شۇڭا، نورۇز بايرىمى ئۆتكۈزىدىغان مىللەتلەردە تەبىئەتتىن زوقلىنىش، تەبىئەتنى سۆيۈش، تەبىئەتنى گۈزەللەشتۈرۈش ئۈچۈن ئەتىيازلىق كۆچەت تىكىپ ئورمان بىنا قىلىش ئادىتى شەكىللەنگەن.



نەۋرۇز ئەمگەك بايرىمىدۇر. نورۇزدا ئاۋام سەپەرۋەر قىلىنىپ، ئېتىز – ئېرىقلار، باغلار، يوللار، ھويلا – ئاراملار تۈزەشتۈرۈلىدۇ، ئەتىيازلىق تېرىلغۇغا تۇتۇش قىلىنىپ، ئىشلەپچىقىرىش ئەمگىكى ئەۋجىگە چىقىدۇ.



نەۋرۇز تەنھەرىكەت بايرىمىدۇر. نورۇز كۈنى كەڭرى مەيدانلاردا پەلۋانلار چېلىشىشقا چۈشۈپ، ئۆزلىرىنىڭ باتۇرلۇقى ۋە مەردلىكىنى سىنىسا، دالىلاردا چەۋەندازلار بەيگە، ئوغلاق تارتىشىشقا چۈشۈپ، ئۇچقۇر ئاتلىرىنى كۆز – كۆز قىلىشىدۇ؛ دارۋازلار مۇئەللەققە تارتىلغان دارغا چىقىپ، ماھارەت كۆرسەتسە، باقمىچىلار ياكى يۇرت كاتتىلىرى قوچقار، خوراز ۋە كەكلىك سوقۇشتۇرۇش، ئىت تالاشتۇرۇش قاتارلىق جاندارلار ئويۇنلىرىنى ئۇيۇشتۇرۇپ، ئاۋامنىڭ كۆڭلىنى ئاچىدۇ.



نورۇز ئەدەبىيات – سەنئەت بايرىمىدۇر. نورۇز كۈنى يەتتە ياشتىن يەتمىش ياشقىچە بولغانلارنىڭ ھەممىسى بايراملىق كىيىملىرىنى كىيىشىپ، يۇرت – مەھەللە بويىچە قوزغىلىپ، مەنزىرىلىك جايلاردا نورۇزغا ئاتاپ تەييارلىغان ھەرخىل ئويۇنلارنى ئوينايدۇ. شائىرلار نورۇزنامىلەرنى ئوقۇشۇپ، مۇتالىئە قىلىشىدۇ. ئەلنەغمىچىلەر ‹‹ ئون ئىككى مۇقام›› كۈيلىرى، خەلق ناخشىلىرىنى ياڭرىتىپ، سەھرا – دالىنى بايرام خۇشاللىقىغا چۆمدۈرىدۇ. بۇنداق چاغلاردا ئۇسسۇل ئوينىيالايدىغانلىكى ئادەم بەس – بەستە ئۇسسۇلغا چۈشۈپ، سورۇننىڭ كەيپىياتىنى ئەۋجىگە كۆتۈرىدۇ.



بۈگۈنكى كۈندە نورۇز ئامان – ئېسەنلىكنى، سالامەتلىكنى، مەمۇرچىلىق، ئاسايىشلىقنى، ئىناقلىق، ئۆملۈكنى تەشەببۇس قىلىدىغان، ماھىر چەۋەندازلارنى، ئىلھامى قايناپ تۇرىدىغان شائىر – قوشاقچىلارنى تاللايدىغان، باھالايدىغان، ناخشا – مۇزىكا، ئۇسسۇل ۋە قىزىقچىلىقلار بىلەن كۆڭۈل ئاچىدىغان، يېڭىلىقلارنى كۆرەك قىلىدىغان، مۇقىملىقنى قوغداپ، ۋەتەنپەرۋەرلىكنى ئۇرغىتىدىغان، كىشىلەرنى ئىلىم – مەرىپەتكە ئۈندەپ، ئىناق جەمئىيەت بەرپا قىلىشقا يېتەكلەيدىغان، يىل بېشىدا خۇشاللىق بىلەن كۈچ – ئىلھام توپلاپ، ئىشقا پۇختا ئاتلىنىدىغان خاسىيەتلىك كۈن بولۇپ قالدى.



ئەنئەنىلەردىن قارىغاندا، بۇ بايرامنىڭ ۋاقتى ئىككى ھەپتە، يەنى، 15 كۈن قىلىپ بەلگىلەنگەن. ھازىر كۆپرەك رايونلاردا ئۈچ كۈن ھېسابىدا ئۆتكۈزۈلۈۋاتىدۇ.

19 – ئەسىردىلا ياۋروپا ئالىملىرى تەرىپىدىن "يىپەك يولى" دەپ ئاتىلىشقا باشلىغان. ھازىر شۇنداق دەپ ئاتىلىش ئادەتكە ئايلانغان ، قەدىمكى چاغدا ئاسىيا بىلەن ياۋروپا ئارىسىدىكى قۇرۇقلۇق قاتناش يولى ناھايىتى ئۇزۇن تارىخقا ئىگە. مانا شۇ يول توغرىسىدا قەدىمكى چاغدىلا جۇڭگو تارىخچىلىرى ۋە يۇنان روما ئالىملىرى مەلۇمات بەرگەن.



سىماچيەن (مىلادىدىن 145 – يىل بۇرۇن تۇغۇلغان ) "تارىخى خاتىرىلەر" ناملىق ئەسىرىنىڭ "پەرغانە ھەققىدە قىسسە" ۋە سەنگو (مىلادىنىڭ 32 – يىلى تۇغۇلۇپ 92 – يىلى ئۆلگەن) "خەن سۇلالىسى يىلنامىسى غەربىي رايون ھەققىدە قىسسە" بابلىرىنى يازغاندا مىلادىدىن ئىككى ئەسىر بۇرۇن غەربىي رايونغا ئىككى قېتىم كېلىپ كەتكەن (مىلادىدىن 130 يىل بىر قېتىم ، مىلادىدىن 115 يىل بۇرۇن بىر قېتىم)مەشھۇر سەيياھ ۋە دىپلومات جاڭ چيەننىڭ خەن سۇلالىسىنىڭ (مىلادىدىن 207 يىل بۇرۇنقى چاغدىن تارتىپ مىلادىنىڭ 220 – يىلىغىچە ھۆكۈم سۈرگەن )پادىشاھى خەن ۋۇدېغا (مىلادىدىن 140 يىل بۇرۇنقى چاغدىن مىلادىدىن 87 يىل بۇرۇنقى چاغقىچە پادىشاھ بولغان ) غەربىي رايوندىكى دۆلەتلەرنىڭ ئەھۋالى ھەققىدە بەرگەن مەلۇماتىدىن پايدىلانغان. جۇڭگونىڭ ئەنە شۇ قەدىمكى تارىخى يىلنامىلىرىغا ئاساسلانغاندا ، دۇنيانىڭ شەرقىدىن غەربىگە بارىدىغان خەلقئارا قۇرۇقلۇق قاتناش يولى جەنۇبىي ۋە شىمالى يول دەپ ئىككىگە بۆلۈنىدۇ. قەدىمكى چاغدا شەرقى ئاسىيادىكى ئەڭ چوڭ دۆلەتلەرنىڭ بىرى بولغان جۇڭگونىڭ پايتەختى چاڭئەن (ھازىرقى شىئەن )دىن غەرپتىكى روما ئېمپىرىيىسىگە بارىدىغان يول گەنسۇدىكى خېشى كارىدۇرى ئارقىلىق داشاتا (دۇڭخۇاڭ) غا ئېلىپ كېلىدۇ. داشاتادىن غەرپكە قاراپ ماڭغاندا ، يول ئىككى ئاچىغا بۆلۈنىدۇ.بۇ ئاچا يولنىڭ بىرى داشاتانىڭ غەربىي شىمالىدىكى لولان ئارقىلىق لوپنورغا ئۆتۈپ قارا قۇرۇم تىزمىلىرىنى بويلاپ خوتەنگەبارىدۇ. خوتەندىن چىقىپ غەربىي شىمالغا قاراپ ماڭغاندا ياركىندكە بارىدۇ. ئاندىن كېيىن پامىر تاغلىرىدىن ئېشىپ ئافغانىستاندىكى بەدەخشان ئارقىلىق ئامۇ ۋادىسىغا بارىدۇ.بۇ يەردىن غەرپكە قاراپ ماڭغاندا ئىران ، ئىراق ئارقىلىق يۇنانىستانغا بارغىلى بولىدۇ.مانا شۇ يول جەنۇبىي يول دەپ ئاتىلىدۇ.



داشاتادىن غەرپكە قاراپ ماڭغاندا ئىككى ئاچىغا بۆلۈنگەن يولنىڭ يەنە بىرى لولاننىڭ شىمالىدىن ئۆتۈپ قۇمۇلغا ئېلىپ بارىدۇ. ئاندىن كېيىن قۇمۇلدىن تۇرپانغا ئۆتۈپ تەڭرى تاغلىرىنىڭ جەنۇبىي تىزمىسىنى بويلاپ غەرپكە ماڭغاندا قارا شەھەر ، كۇچار ، ئاقسۇ ئارقىلىق قەشقەرگە بارىدۇ. قەشقەردىن چىقىپ غەرپكە ماڭغاندا پامىر تاغلىرىدىن ئېشىپ پەرغانە ، سوغدى ، خارەزمى (ھازىرقى سوۋېت ئىتتىپاقىدىكى ئۆزبېكىستان ئىتتىپاقداش جۇمھۇرىيىتىدە) ئارقىلىق ئالانلار مەملىكىتىگە (كاسپى دېڭىزىنىڭ غەريىي شىمالىدا) بارىدۇ. ئۇنىڭدىن كېيىن ، جەنۇپقا قاراپ مېڭىپ ئىرانغا ئۆتۈپ غەرپكە بۇرۇلۇپ ماڭغاندا مىسۇپوتامىيە (ئىراقتىكى دىجلى ۋە فىرات ۋادىسى) ئارقىلىق يۇنانىستانغا بارىدۇ. بۇ يول – شىمالى يول دەپ ئاتىلىدۇ. مانا شۇ يۇقىرىدا تىلغا ئېلىنغان يوللارنىڭ ئاساسى تۈگۈنلىرى ھونلار دەۋرىدىن تارتىپ (مىلادىدىن بىرنەچچە ئەسىر بۇرۇن) ئۇيغۇرلار ۋە ئۇلارنىڭ قېرىنداشلىرى ياشايدىغان رايونلاردا (تارىم ئويمانلىقى ۋە ئوتتۇرا ئاسىيا ، غەربىي ئاسىيادا) ئىدى. ئوچۇقراق ئېيتقاندا ، ئاسىيانىڭ شەرقىدىكى دۆلەتلەر (چاۋشيەن ، جۇڭگو) بىلەن ياۋرۇپا ئارىسىدىكى مۇناسىۋەتلەرنى (ئىفتىسادى ، مەدىنى ، سىياسىي ئالاقىلارنى) باغلايدىغان خەلقئارا سودا يولىغا ھونلار ، تۈركلەر ، ئۇيغۇرلار ھۆكۈمران بولۇپ كەلگەن. خەلقئارا سودا يولىنىڭ ھۆكۈمرانلىقىنى قولغا كەلتۈرۈش ئۈچۈن جۇڭگو بىلەن ھۇنلار ئارىسىدا ، تۈركلەر بىلەن ئىران ئارىسىدا ،ئۇيغۇرلار بىلەن تىبەت خانلىقى ئارىسىدا ئۇرۇشلار يۈز بەرگەن.



چۈنكى ئاسىيا بىلەن ياۋرۇپا ئارىسىدىكى قاتناش يولىغا قايسى دۆلەت ھۆكۈمران بولسا شۇ دۆلەت خەلقئارا سودىدىن نۇرغۇن پايدىنى قولغا كىرگۈزگەندىن تاشقىرى ، سىياسى جەھەتتىمۇئۆزىنىڭ تەسىرىنى ئاسىيا بىلەن ياۋرۇپاغا ئۆتكۈزۈشتە شۇ خەلقئارا يول ئارقىلىق ئۆز مەقسىتىگە يېتەتتى. يۇنان يازغۇچىلىرى ئاسىيا بىلەن ياۋرۇپا ئارىسىدىكى خەلقئارا قۇرۇقلۇق يولى توغرىسىدا مەلۇمات بېرىشكە ئەھمىيەت بەرگەن. شۇ خەلقئارا يول ئارقىلىق ئەڭ قەدىمكى چاغدىن باشلاپ خوتەننىڭ يىپەك ماللىرى (مىلادىدىن 4 ئەسىر بۇرۇن) ئىچكى ئۆلكىلەردىن چىقىدىغان يىپەك ماللار (مىلادىدىن بىر ئەسىر بۇرۇن) يۇنانىستانغا ، روماغا ئېلىپ بېرىلاتتى. رومالىقلار بارغانسېرى يىپەك ماللارنى ئومۇميۈزلۈك ئىشلىتىدىغان بولدى. شۇ سەۋەپتىن مىلادىنىڭ بىرىنچى ئەسىرىدە ياشىغان يۇنانلىق چوڭ سودىگەر مائىسنىڭ ئوتتۇرا ئاسىيالىقلاردىن ياللاپ ئالغان سودا گۇماشتىلىرى خەلفئارا سودا يولىنى تەكشۈرۈپ چىققان.



يۇنان يازغۇچىسى مارتىيۇس (مىلادىنىڭ بىرىنچى ئەسىرىدە ياشىغان) يازغان ئەسىرىدە مائىس ئىگىلىگەن ماتېرىيالدىن پايدىلىنىپ ئاسىيا بىلەن ياۋرۇپا ئارىسىدىكى خەلقئارا سودا يولىنى تونۇشتۇرغان. يۇناننىڭ مەشھۇر جۇغراپىيە ئالىمى كىلاۋدى پىتولمى (مىلادىنىڭ 2 – ئەسىرىدە ياشىغان) مارىينوس بەرگەن مەلۇماتقا ئاساسلىنىپ ئاسىيا بىلەن ياۋرۇپا ئارىسىدىكى خەلقئارا سودا يولىنى تۆۋەندىكىچە كۆرسەتكەنىدى : پىتولمىنىڭ كۆرسىتىشىچە بۇ يول ئوتتۇرا دېڭىزنىڭ شەرقىي قىرغىقىدىن باشلىنىپ سۈرىيە ، ئىراق (مىسوپوتامىيە ۋادىسى) ئىران ،ئافغانىستانغا ئېلىپ بارىدۇ. ئۇنىڭدىن كېيىن شىمالغا بۇرۇلۇپ تاش مۇنار (بۇ تاش مۇنار تاغلاردىن ئالاي ۋادىسىغا چىقىش جايىدا بولسا كېرەك) دىن ئۆتۈپ سىرىسلەر ئىل (يۇنانچە سۆز بولۇپ ، يىپەك ئېلى دېگەن بولىدۇ) گە (تارىم ئويمانلىقى) ئېلىپ بارىدۇ. يۇنان ئالىمى پىتولمىنىڭ مەلۇماتى جۇڭگو تارىخچىلىرىنىڭ "جەنۇبىي يول" دەپ ئاتىغان يولىغا توغرا كېلىدۇ. يۇقىرىدىكى مەلۇماتلار ھونلار دەۋرىدە (مىلادىدىن 240 يىل بۇرۇنقى چاغدىن تارتىپ مىلادىنىڭ 216 – يىلىغىچە داۋام قىلغان) جۇڭگو ۋە يۇنان يازغۇچىلىرىنىڭ بەرگەن مەلۇماتى.



تۈرك خانلىقى دەۋرىدە (مىلادىنىڭ 552 – يىلىدىن 745 – يىلىغىچە) شەرقتىن غەرپكە بارىدىغان خەلقئارا سودا يولى توغرىسىدا جۇڭگونىڭ "سۈي سۇلالىسى يىلنامىسى ، پېجۈي ھەققىدە قىسسە" دىمۇ مەلۇمات بېرىلگەن. شۇ يىلنامىغا كۆرە دۇنيانىڭ شەرقىدىن غەربىگە بارىدىغان يول ئۈچ تارماقتىن ئىبارەت. بۇ ھەقتە يىلنامىدا مۇنداق مەلۇمات بېرىلگەن :



سۈي سۇلالىسى (مىلادىنىڭ 504 – يىلىدىن 612 – يىلغىچە ھۆكۈم سۈرگەن) نىڭ پادىشاھى ياڭدېنىڭ ۋاقتىدا (مىلادىنىڭ 605 – يىلىدىن 618 – يىلىغىچە) غەربىي رايوندىكى دۆلەتلەرنىڭ سودا ئەلچىلىرىنىڭ كۆپچىلكى ھازىرقى گەنسۇدىكى چاڭيىغا كېلىپ جۇڭگو بىلەن سودا ئىشلىرىنى يۈرگۈزگەن. مانا شۇ مۇھىم ئەھۋالنى ھېساپقا ئالغان ياڭدې پىجۈي ئاتلىق كىشىنى جۇڭگو تەرەپنىڭ سودا ئىشلىرىغا مەسئۇل قىلىپ تەيىن قىلغان. پىجۈي جاڭيىغا كەلگەن چەت ئەل سودىگەرلىرى بىلەن بولىدىغان مۇئامىلىدىن پايدىلىنىپ غەربىي ربيوندىكى (تارمىقىدىن ئالغاندا ئوتتۇرا ئاسىيانى ، كەڭ مەنىدىن ئالغاندا شەرقىي تۈركىستاندىن تارتىپ ياۋروپاغىچە بولغان جايلارنى ئۆز ئىچىگە ئالىدۇ )دۆلەتلەردە ياشايدىغان خەلقلەرنىڭ ئۆرپ – ئادەتلىرى ئۇلارنىڭ دۆلىتىدىكى يوللار ، تاغلار ، دەريالار ، خەتەرلىك ئۆتكەللەر (يول ئۆتكەللىرى) توغرىسىدا مەمۇمات توپلىغان. پىجۈي تتوپلىغان ماتېرىياللىرىغا ئاساسلىنىپ "غەربىي رايوندىكى دۆلەتلەرنىڭ خەرىتىسى" ناملىق ئۈچ جىلىدلىق ئەسەر يازغان. مانا شۇ ئەسەردە مۇنداق دىيىلگەن : "داشاتادىن غەربكە قاراپ مېڭىپ كاسپى دېڭىزىنىڭ بويىغا كىرىشىدە ئۈچ يول بار... بۇ يوللارنىڭ بىرى شامالى يول دەپ ئاتىلىدۇ. بۇ يول قۇمۇل ، بارىكۆل ، تۇرالار (ئۇيغۇرلار كۆزدە تۇتۇلىدۇ. ئا) ياشايدىغان جايلار (قۇمۇلنىڭ غەربىدىن تارتىپ ئېلى ۋادىسىغىچە بولغان جايلار كۆزدە تۇتۇلىدۇ ،ئا) ئارقىلىق تۈرك خانلىقىنىڭ ئوردىسىغا (غەربىي تۈرك خانلىقىنىڭ تالاس دەرياسى بويىدىكى ئوردىسى كۆزدە تۇتۇلىدۇ. ئا) ئېلىپ بارىدۇ. ئۇنىڭدىن كېيىن غەرپكە قاراپ ماڭغاندا چۇ دەرياسىدىن ئۆتۈپ شەرقىي روما ئېمپرىيىسىگە ، كاسپى دېڭىزىنىڭ بويىغا بارىدۇ. يوللارنىڭ يەنە بىرى ئوتتۇرا يول دەپ ئاتىلىدۇ. بۇ يول – تۇرپان ، قاراشەھەر ، كۇچار ، قەشقەر ئارقىلىق ئۆتىدۇ. قەشقەردىن چىقىپ غەرپكە قاراپ ماڭغاندا پامىر تاغلىرىدىن ئېشىپ پەرغانە ، تاشكەنت ، سەمەرقەنت ، كىيبود ، بۇخارا ، مەرۋىدىن ئۆتۈپ ئىران ئارقىلىق كاسپى دېڭىزىنىڭ بويىغا بارىدۇ. يوللارنىڭ يەنە بىرى جەنۇبىي يول دەپ ئاتىلىدۇ. بۇ يول پىچان ، خوتەن ، قاغالىق ، تاشقورغاندىن ئۆتۈپ پامىر تاغلىرىدىن ئېشىپ ۋاخان (ئافغانىستاندا) توخارىستان (ئافغانىستان) ، ئىيپتالىت (ھازىرقى پاكىستان ، شىمالى ھىندىستان) ، بامىييان (ئافغانىستاندا) ، كىبود (سەمەرقەنتنىڭ غەربىي شىمالىدا) ئارقىلىق شىمالىي ھىندىستانغا ئېلىپ بارىدۇ. بۇ يەردىن غەرپكە قاراپ يۈرگەندە كاسپى دېڭىزىنىڭ بويىغا بارغىلى بولىدۇ. بۇ ئۈچ يول بويىدىكى دۆلەتلەرنىڭ ھەربىرىنىڭ ئۆزلىرى ماڭىدىغان يوللىرى بولۇپ ، شىمالدىن جەنۇپقا قاتنايدۇ...... قۇمۇل ، تۇرپان ، پىچان غەربىي رايونغا بارىدىغان يوللارنىڭ دەرۋازىسى بولۇپ ، داشاتا بولسا ناھايىتى مۇھىم ئۆتكەل". پىجۈي تەسۋىرلىگەن يوللار ، تۈرك خانلىقى دەۋرىدىكى (552 – 745) خەلقئارا سودا يوللىرى بولۇپ ، بۇ يوللار غەربىدە شەرقىي روما ئېمپىرىيىسىنىڭ پايتەختى كونىستانتىنپول (ھازىرقى ئىستامبۇل) دىن تارتىپ شەرقتە كۈرىيىگىچە (ھازىرقى چاۋشيەنگىچە) بارىدىغان يول ئىدى. پىجۈينىڭ "غەربىي ربيوندا يېڭى دۆلەتلەرنىڭ خەرىتىسى" ناملىق ئەسىرىدە خاتىرىلەنگەن شىمالىي ، ئوتتۇرا ، جەنۇبىي يول دەپ ئاتالغان ئۈچ يولنىڭ ئاساسى لېنىيىسى ۋە تۈگۈنلىرى بولغان رايونلار (كاسپى دېڭىزىنىڭ شەرقىي ۋە شىمالىي قىرغاقلىرىدىكى جايلار ، ئوتتۇرا ئاسىيا ، ئافغانىستان ، تارىم ئويمانلىقى ، جۇڭغار ئويمانلىقى) تۈرك خاقانلىقىنىڭ تېررىتورىيىسى ئىچىدە بولۇپ ، شۇ چاغدىلا تۈركلەر خەلقئارا سودا يولىنىڭ خوجايىنى ئىدى.



2 . شەرق مەلىكىسى توغرىسىدا[تەھرىر]

يېقىنقى يىللاردىن بېرى بەزى ئالىملار (چەت ئەل ئالىملىرىمۇ بۇنىڭ ئىچىدە) قەدىمكى خوتەننىڭ يىپەك توقۇمىچىلىقى ھەققىدە مۇلاھىزە يۈرگۈزگەندە تارىخى ئەمەلىيەتكە زادى ئۇيغۇن كەلمەيدىغان ئاتالمىش "شەرق مەلىكىسى" دەيدىغان رىۋايەتنى كۆتۈرۈپ چىقىپ ، خوتەنگە پىلە قۇرۇتىنىڭ ئۇرىغىنى شەرق مەلىكىسى (جۇڭگو مەلىكىسى)نىڭ ئېلىپ كەلگەنلىكى ، ئۇنىڭدىن بۇرۇن خوتەندە پىلىنىڭ يوق ئىكەنلىكىنى سۆزلەيدىغان بولۇپ قالدى. بۇ رىۋايەتلەرنى راستقا ئايلاندۇرۇشقا ئۇرۇنىۋاتقانلار رەت قىلغىلى بولمايدىغان تارىخى پاكىتلاردىن كۆز يۇمۇپ ، خوتەنگە پىلە قۇرۇتىنىڭ ئۇرىقىنى مىلادىنىڭ 440 – يىللىرى خوتەن خاقانى ۋيجايا جاۋا (بۇ بۇددا دىنىغا ئېتىقاد قىلىدىغان خوتەن ئۇيغۇرلىرىنىڭ بوددىست ھىندىچە ئىسمى بولسا كېرەك) غا ياتلىق بولغان جۇڭگو مەلىكىسى ئېلىپ كەلگەن ، شۇنىڭدىن كېيىن خوتەندە يىپەك توقۇمىچىلىقى پەيدا بولغان دىيىشىدۇ.



ئۇلارنىڭ بىردىنبىر ئاساسلىنىدىغان دەسمايىسى ، جۇڭگو تارىخىدا ئۆتكەن مەشھۇر بۇددىست شۇەنجاڭ (مىلادىنىڭ 602 – يىلى تۇغۇلۇچ 667 – يىلى ئۆلگەن )نىڭ "ئۇلۇغ تاڭ دەۋرىدىكى غەربكە ساياھەت خاتىرىسى" ناملىق ئەسىرىدىكى شەرق مەلىكىسى توغرىسىدىكى رىۋايەت.



شۇەنجاڭ ھىندو بۇددىزىمىنى ئۆگىنىش ۋە تەتقىق قىلىش ئۈچۈن ھىندىستانغا بېرىپ 19 يىل (مىلادىنىڭ 627 – يىلىدىن 645 – يىلىغىچە) تۇرغان. ئۇ لوياڭغا قايتىپ كەلگەندىن كېيىن ، ئوتتۇرا ئاسىيا ، ھىندىستاندا كۆرگەن – بىلگەن ، ئاڭلىغانلىرىغا ئاساسلىنىپ "ئۇلۇغ تاڭ دەۋرىدىكى غەربكە ساياھەت خاتىرىسى" دېگەن ئەسەرنى بىيەنجى دېگەن كىشىنىڭ ياردىمى بىلەن تۈزۈپ چىققان .



"ئۇلۇغ تاڭ دەۋرىدىكى غەربكە ساياھەت خاتىرىسى" ناملىق ئەسەرگە شۇەنجاڭنىڭ ئېيتىپ بېرىشى بويىچە (ھىندى ، تىبەت رىۋايەتلىرى ئاساسىدا) خوتەن خانى ۋيجايا جاۋاغا ياتلىق بولغان شەرق مەلىكىسى (جۇڭگو مەلىكىسى) خوتەنگە كېلىدىغان چاغدا باش كىيىمىنىڭ ئىچىگە پىلە قۇرتىنىڭ ئۇرۇقىنى يوشۇرۇپ چېگرادىن ئۆتكەنمىش. چېگرادىكىلەر مەلىكىنىڭ باش كىيىمىنى تەكشۈرۈشكە جۈرئەت قىلالمىغان. مەلىكە خوتەنگە كەلگەندىن كېيىن پىلىنى ئۆستۈرگەن. بۇنى كۆرگەن ۋەزىرلەر خوتەن خانىغا: بۇ قۇرۇتلارنى كۆيدۈرۈۋېتەيلى ، بولمىسا ئۇ ئەجدىرھاغا ئايلىنىشى مۇمكىن ، دەيدۇ. ئەمما مەلىكە پىلە قۇرۇتىنى يوشۇرۇن ھالدا بېقىپ ئۆستۈرۈپ ئۇنىڭدىن چىققان يىپەكتىن شايى – دۇردۇن توقۇپ ، كىيىم – كېچەك تىكىپ كىيىپتۇ. بۇنى كۆرگەن خوتەن خانى پىلە قۇرۇتنى ئۆستۈرۈشكە يول قويۇپتۇ.



بۇ رىۋايەتنىڭ تارىخى ئەمەلىيەتكە زادى ئۇيغۇن كەلمەيدىغانلقىنى ئىسپاتلاش ئۈچۈن مىلادىنىڭ 440 – يىلى خوتەن خانىغا ياتلىق قىلىنغان (بولغان) شەرق مەلىكىسى (جۇڭگو مەلىكىسى) پىلە قۇرۇتىنى خوتەنگە ئېلىپ كېلىشتىن ئاز دېگەندە 800 يىل بۇرۇن (مىلادىدىن 4 ئەسىر بۇرۇن) خوتەندە يىپەك توقۇمىچىلىقىنىڭ تەرەققىي قىلغانلىقىنى ، ياۋرۇپادىكى رومالىقلارنىڭ خوتەندە توقۇلغان يىپەك ماللارنى ئەتىۋالاپ ، ئۇنىڭدىن كىيىم – كېچەك كىيگەنلىكىنى ، شۇ چاغلاردا تېخى جۇڭگودىن چىقىدىغان يىپەك ماللارنى بىلمەيدىغانلىقىنى ئىسپاتلايدىغان پاكىتلارنى كەلتۈرۈش ئارقىلىق "خوتەنگە پىلە قۇرۇتىنىڭ ئۇرۇقىنى شەرق مەلىكىسى ئېلىپ كەلگەن" دېگەن رىۋايەتنى رەت قىلىش لازىم.



ئاتالمىش "شەرق مەلىكىسى" توغرىسىدىكى يالغان رىۋايەتنى رەت قىلىش ئۈچۈن قەدىمكى چاغلاردا رومالىقلارنىڭ جۇڭگونىڭ يىپەك ماللىرىنى تېخى بىلمىگەن ۋە ئۇنى كۆرمىگەن چاغدا ، ئەڭ دەسلەپ قايسى خەلقلەر توقىغان يىپەك ماللارنى بىلىدىغانلىقى ۋە ئۇ ماللاردىن كىيىم – كېچەك كىيگەنلىكى توغرىسىدا مەلۇمات بېرىش ناھايىتى مۇھىم مەسىلە.



قەدىمكى چاغدىكى يۇنان ۋە روما ئالىملىرىدىن كىتىئاس (مىلادىدىن 400 يىل بۇرۇن ئۆتكەن) ، سىترا بون (مىلادىدىن 54 يىل بۇرۇن تۇغۇلۇپ مىلادىنىڭ 24 – يىلى ئالەمدىن ئۆتكەن) ، خوبلىئوس ۋىيگىلى مارۇ (مىلادىنىڭ 50 – يىللىرى ئۆتكەن) ، روما تارىخچىسى فىلوروس (مىلادىنىڭ 50 – يىللىرى ئۆتكەن) ، فىلىينى (مىلادىنىڭ 23 – يىلى تۇغۇلۇپ 79 – يىلى ئالەمدىن ئۆتكەن) ، كىلاۋدى پىتولمى (مىلادىنىڭ 2 – ئەسىرلىرىدە ئۆتكەن) قاتارلىقلار ئۆزلىرىنىڭ يازغان ئەسەرلىرىدە تارىم ئويمانلىقىدىكى ئۇيغۇرلارنىڭ يۇرتىنى (خوتەننى ئاساس قىلغان) يۇنان تىلىدا "سىرىسلار دۆلىتى""(يىپەك دۆلىتى") دەپ يېزىشقان.



ياۋرۇپالىقلاردىن تۇنجى قېتىم يىپەك دۆلىتى ("سىرىسلار دۆلىتى") توغرىسىدا مەلۇمات بەرگەن. يۇنان ئالىمى كىتىئاس بولۇپ ، ئۇ مىلادىدىن 400 يىل بۇرۇن "سىرىسلار دۆلىتى" دېگەن نامنى تىلغا ئالغان.



يۇنان ئالىملىرىدىن سىتىرابون "ساياھەت خاتىرىسى" ناملىق ئەسىرىنى يازغاندا (مىلادىدىن بىر ئەسىر بۇرۇن) ئۇلۇغ ئىستىلاچى ئالىكاندىر ماكىدونىكى (ئىسكەندەر زۇلقەرنەيىن)نىڭ مىلادىدىن 328 يىل بۇرۇن ئالەمدىن ئۆتكەن مەشھۇر سەركەردىسى ئۇنىسكىر يىتوسنىڭ خاتىرىسىدىن پايدىلانغان. ھىندىستانغا كەلگەن ئۇنىسكىر يىتوس ئۆزى يازغان خاتىرىدە ئۇيغۇرلارنىڭ ئانا يۇرتى بولغان تارىم ئويمانلىقىنى "سىرىسلار دۆلىتى" ("يىپەك دۆلىتى") دەپ يازغان. "سىرىسلار دۆلىتى" (يىپەك دۆلىتى") توغرىسىدا يۇنان ۋە روما يازغۇچىلىرى ئىچىدىن مىلا ، پىلىينى ، مارىسىللىنوس ، پىتولمى قاتارلىقلارنىڭ بەرگەن مەلۇماتى بىرقەدەر تەپسىلىراق ۋە ئېنىقراق.



روما يازغۇچىسى مىلا "مىلادىنىڭ 50 – يىللىرى ئۆتكەن) ، "ئاسىيانىڭ شەرقىدە ھىندىلار ، سىرىسلىقلار ، سىتىسىلىكلەر3 ياشايدۇ. ھىندىلار بولسا جەنۇبتا ، سىتسىلىكلەر بولسا ئەڭ شىمالدا ، سىرىسلار بولسا مانا شۇ ئىككىسىنىڭ (سىتىسلىكلەر بىلەن ھىندىلار – ئا) ئوتتۇرا قىسمىدا ياشايدۇ... سىرىسلىقلار دۇنيادا تەڭدىشى تېپىلمايدىغان سەمىمى ، سادىق ئادەملەر ، ئۇلار سودا ئىشلىرىغا ئۇستا بولۇپ ، سودا قىلغاندا يۈزمۇيۈز تۇرۇپ سۆزلەشمەيدۇ. ماللىرىنى قۇملۇققا قويۇپ – قويۇپ كەينىنى قىلىپ تۇرىدۇ " [1]دەپ يازىدۇ. مىلاننىڭ سىرىسلىقلار توغرىسىدا بەرگەن مەلۇماتى. ئەمەلىيەتكە دەلمۇدەل ئۇيغۇن كېلىدۇ. بۇنىڭدا ، دىققەت قىلىشقا ئەرزىيدىغان ئۈچ مۇھىم نۇقتا بار. (1)مىلا ئېيتقاندەك ، ىندىستان سىرىسلىقلار (ئۇيغۇرلار – ئا)نىڭ يۇرتى تارىم ئويمانلىقىنىڭ جەنۇبىغا جايلاشقان. [2] مىلا "سىرىسلىقلار سودا قىلغاندا ، يۈزمۇيۈز تۇرۇپ سۆزلەشمەيدۇ. مالنى قۇملۇققا قويۇپ – قويۇپ ، كەينىنى قىلىپ تۇرىدۇ" دەيدۇ. بۇنداق ئەھۋال قەدىمكى چاغلاردا تۈركى خەلقلىرىدە ئادەتكە ئايلىنىپ كەتكەن "تىلسىز سۇدا"نى كۆرسىتىدۇ.[3] مىلا "سىرىسلىقلار مالنى قۇملۇققا قويىدۇ " دەيدۇ. مىلا ئېيتقان قۇملۇق تارىم ئورمانلىقىدىكى قۇملۇق – چۆلنى كۆرسىتىدۇ. جۇڭگونىڭ ئاساسى تېررىتورىيىسى بولغان ئوتتۇرا تۈزلەڭلىكتە قۇملۇق چۆل يوق. بۇنى جۇغراپىيىدىن ئاددى ساۋادى بار ئادەملەرنىڭ ھەممىسى بىلىدۇ.



تەبىئەتشۇناس رىم يازغۇچىسى پىلىينى "تەبىئەت تارىخى" ناملىق ئەسىرىدە "سىرىس دۆلىتى" ("يىپەك دۆلىتى") توغرىسىدا مەزمۇنى مىلا بەرگەن مەلۇماتقا ئوخشايدىغان مەلۇمات بېرىدۇ. پىلىينى مۇنداق دەپ يازىدۇ : "سىرىسلىقلارنىڭ ئورمانلىرىدىن يىپەك چىقىدۇ. دۇنياغا مەشھۇر.... ئۇلار كىمخاپ ، تاۋاق – دۇردۇن توقۇپ روماغا ئېلىپ بېرىپ ساتىدۇ..... سىرىسلىقلار مۇلاھىم ، تارتىنچاق كېلىدۇ."[4]پىلىينى يەنە مۇنداق دەپ يازىدۇ : "سىرىسلىقلار ، بوي – تۇرقى جەھەتتە ئادەتتىكى ئادەملەردىن ئېگىز ، چاچلىرى قىزىل ، كۆزلىرى كۆك ، ئاۋازلىرى جاراڭلىق كېلىدۇ. چەتتىن بارغانلار ، ئۇلارنىڭ تىلىنى بىلمەيدۇ. شۇنىڭ ئۈچۈن ئۇلار بىلەن سۆھبەتلىشەلمەيدۇ. چەتتىن بارغانلار ، ماللىرىنى مەلۇم دەريانىڭ شەرقى قىرغىقىغا ئېلىپ بېرىپ ، سىرىسلىقلارنىڭ ماللىرىنىڭ يېنىغا قويىدۇ. ئەسلىدە بېكىتىلگەن باھا بويىچە مال ئالماشتۇرۇلىدۇ." [5]پىلىيىنىڭ مەلۇماتىدىن قارىغاندا ، سىرىسلىقلارنىڭ يىپەك ماللىرى روماغا ئېلىپ بېرىلىپ سېتىلىدىكەن. ئۇنىڭ ئۈستىگە پىلىيىنىنىڭ ، سىرىسلىقلارنىڭ ئېرقى خۇسۇسىيەتلىرى توغرىسىدا بەرگەن مەلۇماتى ناھايىتى يۇقىرى قىممەتكە ئىگە. بۇ ، ئۇيغۇرلارنىڭ ئاق تەنلىك ئېرققا مەنسۇپ ئىكەنلىكىنى ئىسپاتلايدۇ. جۇڭگونىڭ قەدىمكى تارىخچىلىرىمۇ ، ئېرقى جەھەتتىن ئۇيغۇرلارغا بەكمۇ يېقىن بولغان ئويسۇنلارنى "كۆك كۆز ، قىزىل چاچ" [6]دەپ يازغان بولسا ، قىرغىزلارنى "چىرايى ئاق سۈزۈك ، قىزىل چاچ ، كۆزلىرى كۆك" [7]دەپ يازغان. پىلىيىنى ، رىم ئات سۆڭەكلىرىنىڭ ئەيشى – ئىشرەت ، كەيپ – ساپاغا بېرىلىپ تۇرمۇشتا چىرىكلىشىپ ، بايلىقلارنى ھەددىدىن ئارتۇق بزۇپ – چېچىپ ، ئەخلاقى جەھەتتە بەكمۇ بۇزۇلۇپ كەتكەنلىكى ، ئۇلارنىڭ كىيىدىغان كىيىملىرىنىڭ سىرىسلەر دۆلىتىدىن كېلىدىغان كىمخاپ تاۋار – دۇردۇنلاردىن تىكىلىدىغانلىقى ، سىرىسلار دۆلىتىدىن سېتىپ ئالىدىغان يىپەك ماللار ئۈچۈن رىم تىللالىرىنىڭ (ئالتۇن پۇللىرىنىڭ – ئا) سىرىسلىقلارنىڭ قولىغا چۈشۈپ كېتىدىغانلىقى توغرىسىدا ھەسرەتلىنىپ مۇنداق دەپ يازغان ئىدى : "دۆلىتىمنىڭ تىللالىرىدىن ھىندىستان سىرىسلار دۆلىتى ۋە ئەرەپ يېرىم ئارىلى قاتارلىق ئۈچ دۆلەتكە ھەر يىلى ئاز دېگەندە يۈز مىليون سىتىر كىيىس (رىم تىللاسى) ئېقىپ كېتىدۇ. مانا مۇشۇ پۇللار ، دۆلىتىمدىكى ئەرلەر بىلەن ئاياللار (ئاق سۆڭەكلەرنى دىمەكچى)نىڭ بۇزۇپ – چېچىپ خەجلىشى ئۈچۈن كېتىدۇ." [8]پىلىيىنىنىڭ "تەبىئەت تارىخى"دىكى مەلۇماتقا ئاساسلانغاندا ، روما ئاق سۆڭەكلىرى ئۆزلىرىنىڭ ئەيشى – ئىشرەت ، كەيپ – ساپالىق تۇرمۇشىنىڭ ئېھتىياجىغا لازىم بولىدىغان يىپەك ماللارنى سىرىسلىقلاردىن ئالغان بولسا ، دورا – دەرمەك ، ئۈنچە – ياقۇتلارنى ھىندىستاندىن ، مەرۋايىتلارنى ئەرەبىستاندىن ئالغان.



يۇقىرىدا ئېيتىلغاندەك ھەر يىلى روما پۇلىدىن يۈز مىليون سىىيسىتىركىيس (رىم تىللاسى) روما ئاق سۆڭەكلىرىنىڭ ئېھتىياجى ئۈچۈن سىرىسلىقلاردىن سېتىپ ئالىدىغان يىپەك ماللار ، ھىندىستاندىن ۋە ئەرەبىستاندىن ئالىدىغان دورا – دەرمەك ، ئۈنچە – ياقۇت ، مەرۋايىتلار ئۈچۈن سەرپ بولغان. يونان ئالىمى كىلاۋدى پىيتۇلىيمى "جۇغراپىيە" ناملىق ئەسىرىدە ، سىرىسلار دۆلىتى (يىپەك دۆلىتى) نىڭ تەبىئى شارائىتى ، قانداق خەلقلەر ياشايدىغانلىقى توغرىسىدا مۇنداق دەپ يازىدۇ : " سىرىسلار دۆلىتىنىڭ غەربىي چېگرىسى سىستىيە بولۇپ ، ئۇ ئىمماۋىس [9]تېغىنىڭ سىرتىدا ، شىمالى چېگراسى نامسىز يەر....... شەرقى چېگراسىمۇ نامسىز يەر. .... جەنۇبىي بولسا ، ھىندىستاندىكى گانگې دەرياسىنىڭ شەرقىي قىرغىقى بىلەن چېگرالىنىدۇ." [10]پىتۇلىيمى ، سىرىسلار دۆلىتىنىڭ ئەتراپىنى ئوراپ تۇرىدىغان تاغلار توغرىسىدا مۇنداق دەپ يازىدۇ : "سىرىسلار دۆلىتىنىڭ تۆت تەرىپىنى.... تاغلار ئوراپ تۇرىدۇ. ئۇنىڭ تېررىتورىيىسى ئىچىدە ئىككى چوڭ دەريا ئاقىدۇ. ئۇ دەريالارنىڭ بىرىنچىسى ئۇيغۇرداس ( ) دەرياسى بولۇپ ، ئۇ ئاۋشاتىسئان ۋە ئاسمىران( ) تېغىدىن ئىبارەت ئىككى مەنبەدىن باشلىنىدۇ. دەريالارنىڭ ئىككىنچىسى باۋتىس دەرياسى بولۇپ ، ئۇكاسىيان ( ) ۋە ئوتتۇرۇ كورخۇس ( ) تېغىدىن ئىبارەت ئىككى مەنبەدىن باشلىنىدۇ.[11] يۇقىرىدا تىلغا ئېلىنغان ئىككى دەريانىڭ ئالدىنقىسى ئاقسۇ ياكى قەشقەر دەرياسى ، كېيىنكىسى ياركەنت (زەرەپشان) ياكى قارىقاش دەرياسىدىن باشقا دەريا ئەمەس.



پىتولىمى ئۇيغۇرداس دەرياسىنىڭ بويىدا ئۇيغۇرداس دۆلىتى بار دېگەن. پىتولىمىنىڭ ئۇيغۇرداس دېگىنى ، دەل ئۇيغۇرلارنىڭ يۇنان تەلەپپۇزىدا ئېيتىلىشى خالاس. يۇنان ئالىمى مارىسىللىنوس (مىلادىنىڭ 380 – يىلى ياشىغان) "تارىخنامە" ئاتلىق كىتابىدا : "شەرقتە سىرىسلەر ئېلى بار ، ئۇنىڭ ئەتراپىنى ئېگىز تاغلار ئوراپ تۇرىدۇ. بۇ تاغلار بىر تۇتاش سوزۇلۇپ تەبىئى توسۇقنى شەكىللەندۈرىدۇ. سىرىلىقلار شۇنىڭ ئىچىدە ياشايدۇ. ئۇلارنىڭ يېرى تەكشى ، كەڭتاشا ۋە باي. غەربتە سېستىيە بىلەن چېگرالىنىدۇ. شەرقىي بىلەن شىمالدىن ئىبارەت ئىككى تەرىپى چۆللۈك ، تاغلىرىنىڭ ئۈستى (تەڭرى تاغلىرى بىلەن پامىر تاغلىرى كۈزدە تۇتۇلسا كېرەك) يىل بويى قار بىلەن قاپلىنىپ تۇرىدۇ. جەنۇبىي چېگراسى ھىندى ۋە گانگې دەرياسىغىچە بارىدۇ.... تاغلىرىنىڭ ھەممىسى ئېگىز ، يوللىرى تىك قىيالىق ئەگرى – بۈگرى كېلىدۇ. تاغلارنىڭ ئارىسى تۈزلەڭلىك ، ئۇنىڭ تېررىتورىيىسى ئىچىدە ئۇيغۇرداس ۋە باۋتىس دەرياسىدىن ئىبارەت ئىككى دەريا ئاقىدۇ... سىرىسلەر تىچ ياشايدۇ. ھەربىي قورال – ياراق تۇتمايدۇ. زادى ئۇرۇش قىلمايدۇ. مۇلاھىم ، ياۋاش ، قوشنا دۆلەتلەرنى پاراكەندە قىلمايدۇ. ئىقلىمى مۆتىدىل ، ھاۋاسى ساپ ، پاكىز ، ئاسماندا بۇلۇت كۆپ بولمايدۇ. قاتتىق بوران چىقمايدۇ. ئورمانلىق ئىنتايىن كۆپ ، ئورمانلىقتا ماڭسا ئاسماننى كۆرگىلى بولمايدۇ."[12]مارسىللىنوس يەنە مۇنداق دەپ يازىدۇ : "سىرىسلىقلار ئاددى – ساددا ياشاشقا ئادەتلەنگەن. ئۇلار خالى جايدا ئولتۇرۇپ كىتاب ئوقۇپ كۈن ئۆتكۈزۈشنى ياخشى كۆرىدۇ. كىشىلەر بىلەن بېرىش – كېلىش قىلىشنى ئانچە ياقتۇرمايدۇ. چەت ئەللىكلەر چېگرادىكى دەريادىن ئۆتۈپ ، ئۇ يەرگە يىپەك ياكى باشقا مال ئالغىلى بارسا ، كۆزلىرى بىلەن بىر – بىرىگە بېقىشىپلا باھاسىنى توختىتىدۇ. گەپلەشمەيدۇ ، ئۇلارنىڭ يەر بايلىقى مول ، باشقىلارغا ئېھتىياجى چۈشمەيدۇ ".[13]يۇقىرىدا ، يۇنان ۋە رىم ئالىملىرىنىڭ خاتىرىلىرىدىن كەلتۈرگەن پاكىتلاردىن تۆۋەندىكى خۇلاسىنى چىقىرىش مۇمكىن :







مىلادىدىن 400 يىل بۇرۇنقى ۋاقتىدىن تارتىپ ياۋرۇپالىقلار (يۇنان ۋە رىم) تارىم ۋادىسىدىكى ئۇيغۇرلار ئېلىنى، يىپەك ئىلى (سىرىسلەر ئېلى) دەپ تونىغان.

يۇنان ۋە رىم ئالىملىرى تارىم ئورمانلىقىنىڭ تەبىئى شارائىتى (تاغلىرى ، دەريالىرى ، ئورمانلىرى ، چۆللىرى ، ئېقلىمى) ئۇنىڭغا چېگرالىنىدىغان دۆلەتلەر ھەققىدە ئاساسەن توغرا مەلۇمات بەرگەن.

يۇنان ۋە رىم ئالىملىرى ئەڭ قەدىمكى چاغلىرىدىن تارتىپ ياۋرۇپالىقلارنىڭ تارىم ئويمانلىقىدىكى ئۇيغۇرلارنىڭ يىپەك ماللىرىنى ئىستىمال قىلغانلىقىنى كۈچلۈك پاكىتلار ئارقىلىق ئېنىق ۋە تەپسىلىي چۈشەندۈرگەن.

يۇنان ۋە رىم ئالىملىرى تارىم ئويمانلىقىدا ياشىغان خەلقنىڭ ئۇيغۇر ئىكەنلىكىنى ، ئۇلارنى ئۆز نامى بىلەن (ئۇيغۇرداس) ئاتىغاندىن تاشقىرى ، ئۇلارنىڭ ئىرقى جەھەتتىن قايسى ئىرققا مەنسۇپ ئىكەنلىكىنىمۇ ئىسپاتلاپ بەرگەن.

ئەگەر ، سىرىس ئېلى دەپ ئاتالغان تارىم ئويمانلىقىدا يىپەكچىلىك سانائىتى ئەڭ قەدىمكى چاغلاردىن تارتىپ (مىلادىدىن نەچچە ئەسىر بۇرۇن) تەرەققى قىلمىغان بولسا ، "جاھاننىڭ مەر – مەر شەھرى" دەپ ئاتالغان رىمدىكى روما ئاق سۆڭەكلىرىنىڭ يۇقىرى سۈپەتلىك يىپەك ماللارغا بولغان ئېھتىياجىنى قامدىيالمىغان بولاتتى.

ئۇيغۇرلارنىڭ قەدىمكى چاغلاردا ، يىپەك توقۇمىچىلىقىدا يۇقىرى سەۋىيىگە كۆتۈرۈلگەنلىكىنى ئىسپاتلايدىغان پاكىتلار ئاز ئەمەس. ئىچكى ئۆلكىلەردىكى خەنزۇلار يىپەك توقۇمىچىلىقىدا ئۇيغۇرلارنىڭ توقۇمىچىلىقتىكى ئارتۇقچىلىقىدىن ئۆگەنگەنلىكى بۇنىڭغا مىسال بولىدۇ. يېقىنقى يىللاردا ، شىنجاڭنىڭ ھەرقايسى جايلىرىدىن 6 – ئەسىردە ئىچكى ئۆلكىلەردە توقۇلغان يىپەك توقۇلمىلار قېزىۋېلىندى. بۇنداق يىپەك توقۇلمىلاردىكى "نىسبەتەن كۆزگە كۆرۈنەرلىك تەرەققىيات شۇ بولغانكى ، شۇ چاغلاردا تۈز يوللۇق كىمخاپلاردىن تاشقىرى يەنە قىيپاش گۈللۈك ئارقىغىدىن گۈل چىقىرىلغان كىمخاپلارمۇ مەيدانغا كەلگەن ، ئارقاقتىن بۇ خىل گۈل چىقىرىش ئۇسۇلىنى خەن سۇلالىسى ۋاقتىدىلا (مىلادىدىن بىرنەچچە ئەسىر بۇرۇن شىنجاڭدىكى قېرىنداش مىللەتلەر يۇڭ توقۇمىچىلىق ھۈنىرىدە قوللانغان. ئوتتۇرا تۈزلەڭلىك رايونلىرى يىپەك توقۇمىچىلىقتا قوللانغان بۇ خىل توقۇش ئۇسۇلىنى (كىمخاپلارنىڭ ئارقىغىدىن گۈل چىقىرىش ئۇسۇلىنى – ئا) شىنجاڭدىن ئۆگەنگەن بولسا كېرەك." [14]مانا شۇ پاكىت قەدىمكى چاغلاردا ئۇيغۇرلارنىڭ يىپەك توقۇمىچىلىقى ياكى يۇڭ توقۇمىچىلىقىدا بولسۇن يۇقىرى سەۋىيىگە كۆتۈرۈلگەنلىكىنى كۆرسىتىدۇ.



بۇنىڭدىن تاشقىرى ، مىلادىنىڭ 440 – يىللىرى ، شەرق مەلىكىسى پىلە قۇرۇتىنىڭ ئۇرۇقىنى خوتەنگە ئېلىپ كەلگەن ، ئۇنىڭدىن بۇرۇن شىنجاڭ رايونىدا "يىپەك يوق ئىدى "دېگەن رىۋايەتنى رەت قىلىدىغان مۇنداق بىر پاكىتىمۇ بار : يېقىنقى يىللاردا تۇرپان رايونىدا ئېلىپ بېرىلغان ئارخىئولوگىك قېزىشلەر ئارقىلىق "قېزىۋېلىنغان قول يازمىلار ئىچىدە.... مىلادىنىڭ 418 – يىلىدا پىلە قۇرۇتىنى ۋە ئۈژمە دەرىخىنى ئىجارىگە ئېلىش توغرىسىدا يېزىلغان ھۆججەت ئۇچرىدى." (يادىكارلىق 40 – بەت). مانا بۇ پاكىت ، يىپەك مەلىكىسى ("شەرق مەلىكىسى") مىلادىنىڭ 440 – يىللىرى پىلە قۇرۇتىنىڭ ئۇرۇقىنى خوتەنگە ئېلىپ كەلگەن دېگەن رىۋايەتنى رەت قىلىدۇ. قەدىمكى چاغدا خوتەندىلا ئەمەس ، ھەتتا تۇرپان رايوندىمۇ يىپەكچىلىك تەرەققىي قىلغانىدى.



يۇقىرىدىكى پاكىتلاردىن چىقىرىلغان ئەڭ ئاساسلىق ۋە مۇھىم خۇلاسە ئاتالمىش "يىپەك مەلىكىسى" رىۋايىتىدىكى مىلادىنىڭ 440 – يىلى جۇڭگودىن خوتەن خانىغا ياتلىق بولغان مەلىكىنىڭ پىلە قۇرۇتى ئۇرۇقىنى خوتەنگە ئېلىپ كېلىشىدىن بۇرۇن خوتەندە يىپەك توقۇمىچىلىق سانائىتى يوق ئىدى دېگەن يالغان رىۋايەتنىڭ ئۈزۈل – كېسىل رەت قىلىنىشى ، خالاس.

شۇنىمۇ ئېيتىش كېرەككى، قەدىمكى چاغلاردا جۇڭگودىن باشقا دۆلەتلەرنىڭ ھۆكۈمرانلىرىغا ياتلىق بولغان مەلىكىلەرنىڭ ئىسىملىرى جۇڭگونىڭ تارىخى يىلنامىلىرىغا ، ئۇيسۇن خانلىرىغا ياتلىق بولغان جۇڭگو مەلىكىلىرىنىڭ ئىسىملىرى جۇڭگو تارىخى يىلنامىلىرىغا يېزىلغان. ئەپسۇسكى مىلادىنىڭ 5 – ئەسىرىدە خوتەن خانىغا ياتلىق بولغان مەلىكىنىڭ ئىسمى جۇڭگونىڭ تارىخى يىلنامىلىرىدا ئۇچرىمايدۇ.



تارىخىي پاكىتلارنىڭ راستلىقىغا ھۆرمەت قىلىنىدىغان بولسا ، تارىخىي پاكىتلار بويىچە ئىسپاتلانمىغان رىۋايەتنىڭ يالغانلىقى ئېتىراپ قىلىنسا ، خوتەنگە پىلە قۇرۇتىنىڭ ئۇرۇقىنى ، خوتەن خانىغا ياتلىق بولغان جۇڭگو مەلىكىسى ئېلىپ كەلگەن دېگەن رىۋايەتنىڭ تامامەن يالغانلىقىنى ئېتىراپ قىلماسلىققا ھېچقانداق بانا – سەۋەب يوق. ھونلارنىڭ ئاسىيا بىلەن ياۋرۇپا بارىدىغان خەلقئارا قاتناش يولىغا ھۆكۈمرانلىق قىلىشى ، جۇڭگو بىلەن غەرب ئەللىرى ئارىسىدىكى تۈرلۈك مۇناسىۋەتلەرنىڭ ئوڭۇشلۇق ئېلىپ بېرىلىشىغا توسقۇن بولغان. شۇنىڭ ئۈچۈن شەرق بىلەن غەرب ئارىسىدىكى خەلقئارا قاتناش يولى جۇڭگوغا نىسبەتەن گاھ تاقىلىپ ، گاھ ئېچىلىپ تۇراتتى.



مىلادىنىڭ 87 – يىلى ۋە مىلادىنىڭ 101 – يىلى ، ئىران ئەلچىلىرى جۇڭگوغا كەلگەن بولسا ، مىلادىنىڭ 97 – يىلى جۇڭگو ئەلچىسى گەن يىڭ (بۇنى بەنچاۋ ئەۋەتكەن) ئىرانغا بارغان. جۇڭگو بىلەن ئىران ئارىسىدا سودا مۇئامىلىسى توغرىسىدا بېتىم تۈزۈلگەن. تارىخى مەلۇماتلارغا ئاساسلانغاندا مىلادىنىڭ 120 – يىلى رومالىق بىر سېھرىگەر دېڭىز يولى بىلەن ھىندىچىنىغا كېلىپ ، بېرما ئارقىلىق جۇڭگوغا كەلگەن. مىلادىنىڭ 1660 – يىلى روما ئېمپىراتۇرى ماركۇس ئاۋرىل (مىلادىنىڭ 161 – يىلىدىن 166 – يىلىغىچە ئېمپىراتۇر بولغان) دېڭىز يولى ئارقىلىق جۇڭگوغا ئەلچى ئەۋەتكەن. روما ئەلچىسى دېڭىزدىن ھىندىچىنىغا چىقىپ ۋيېتنام ئارقىلىق جۇڭگوغا كەلگەن. ئەلچى جۇڭگو پادىشاھى خەن خۇئاندى (مىلادىنىڭ 147 – يىلىدىن 167 – يىلىغىچە پادىشاھ بولغان )غا پىل چىشى ، كەركە مۈڭگۈزى ، ياقۇت قاتارلىق سوۋغاتلارنى ئېلىپ كەلگەن. يۇقىرىدىكى تارىخى پاكىتلارغا ئاساسلانغاندا ئىراننىڭ جۇڭگو بىلەن ئورناتقان دىپلوماتىك مۇناسىۋىتى رومانلىقلارنىڭ سىرىسلەر ئېلى (يىپەك ئېلى) دەپ ئاتىغان ئۇيغۇرلار بىلەن ئورناتقان يىپەك سودىسى مۇناسىۋىتى (بۇنداق مۇناسىۋەت مىلادىدىن 400 يىل بۇرۇن باشلانغان)دىن ئاز دېگەندە ئۈچ ئەسىر (مىلادىدىن 105 يىل بۇرۇن) كېيىن باشلانغان. رومالىقلارنىڭ جۇڭگو بىلەن ئورناتقان دىپلوماتىك مۇناسىۋىتى (ئەگەر شۇنداق مۇناسىۋەت مىلادىنىڭ 166 – يىلىدىن باشلانغان بولسا) رومالىقلارنىڭ سىرىسلەر ئېلى بىلەن ئورناتقان يىپەك سودىسى مۇناسىۋىتىدىن تەخمىنەن 600 يىل كېيىن باشلانغان.



پىلە قۇرۇتى ئۇرۇقىنىڭ ياۋرۇپاغا تارقىلىشى[تەھرىر]

قەدىمكى رىم ئېمپىرىيىسى دەۋرىدە (مىلادىدىن بۇرۇن ۋە كېيىن) يىپەكتىن توقۇلغان كىمخاپ ، شايى تاۋار ، دۇردۇنلارنىڭ ياۋرۇپاغا (ئاساسەن روماغا) تارىم ئويمانلىقىدىكى يىپەك ئېلى (سىرىسلەر ئىلى)دىن بارغانلىقى ، ئاتالمىش "يىپەك مەلىكىسى" رىۋايىتىنىڭ يالغانلىقى توغرىسىدا سۆزلەپ ئۆتتۈم. تۈرك خاقانلىقى دەۋرىدە ئىچكى ۋە تاشقى (خەلقئارا) سودا ئىشلىرى ئەھۋالىنىڭ ئاساسى مەشغۇلاتلىرىدىن بىرى ئىدى. بولۇپمۇ بۇ دەۋردە خەلقئارا سودا ناھايىتى كەڭ مىقياسىدا قانات يايغان ئىدى. مىلادىنىڭ 624 – يىلى تۇرپان خانلىقى (مىلادىنىڭ 460 – يىلىدىن 640 – يىلىغىچە ھۆكۈم سۈرگەن)نىڭ خانى كۈي ۋىنتاي غەربىي تۈرك خانلىقىنىڭ خاقانى تۇن يابغۇ خاقاننىڭ رۇقسىتىنى ئېلىپ جۇڭگودىن تاڭ سۇلالىسىنىڭ پادىشاھى لى يەن (مىلادىنىڭ 618 – يىلىدىن 627 – يىلىغىچە پادىشاھ بولغان) گە ئەۋەتكەن سوۋغا – سالاملىرى ئىچىدە شەرقىي رومادىن كەلتۈرۈلگەن كۇچۇكلار ، قارا تۈلكە موينىسى قاتارلىق نەرسىلەر بار ئىدى. شۇنىڭغا قارىغاندا تۈركلەر روما بىلەن قىلغان سودىلىرىدا رومانىڭ نەسىللىك ئىتلىرىنىمۇ ئېلىپ كەلگەن. تۈرك خاقانلىرى دەۋرىدىمۇ ، بۇرۇنقى چاغلارغا ئوخشاشلا ئاسىيادىن ياۋرۇپاغا ئېلىپ بېرىلىدىغان سودا ماللىرىدىن يىپەك رەختلەر ، دورا – دەرمەكلەر ، زىبۇ – زىننەت بۇيۇملىرى ، تېردىن ئىشلەنگەن ماللار ئاساسى سالماقنى ئىگىلەيتتى.



تۈركلەر يالغۇزلا ياۋرۇپا بىلەن سودا قىلىپ قالماستىن ، جۇڭگو بىلەنمۇ ناھايىتى كەڭ ھالدا سودا ئىشلىرىنى ئېلىپ بارغان. تۈرك خانلىقى تەركىبىدىكى ئوتتۇرا ئاسىيانىڭ مۇھىم خانلىقلىرىدىن سەمەرقەنت خانلىقى ، بۇخارا خانلىقى قاتارلىق خانلىقلار مىلادىنىڭ 627 – يىلىدىن 647 – يىلىغىچە بولغان 20 يىل ئىچىدە جۇڭگوغا توققۇز قېتىم سودا ئەلچىلىرى ئەۋەتكەن . تۈرك خاقانلىقى تەركىبىگە كىرگەن خانلىقلار يېرىم مۇستەقىل ھالدا بولۇپ بەزى خەلقئارا مۇناسىۋەتلەردىكى چوڭ ئىشلارنى قىلىشتا تۈرك خاقانىنىڭ رۇخسىتىنى ئالاتتى.



يۇقىرىدا ئېيتىپ ئۆتكىنىمىزدەك ، تۈركلەر بىلەن ئىران ساسانىلار سۇلالىسى ئارىسىدا بىرنەچچە قېتىم ئۇرۇشنىڭ يۈز بېرىشى خەلقئارا سودا يولىغا بولغان ھۆكۈمرانلىقنى تالىشىش ۋە خەلقئارا يىپەك سودىسىدىن كېلىدىغان پايدىنى قولغا كەلتۈرۈش ئۈچۈن بولغان بۇ كۆرەشتە شەرقىي روما بىلەن ئىتتىپاقداشلىق مۇناسىۋىتىنى ئورناتقان تۈركلەر ئۈستۈن چىققان.



ھەر يىلى دېگۈدەك نەچچە ئون مىليونلىغان رىم تىللاسىغا سىرىسلەر ئېلىدىن يىپەك ماللارنى سېتىپ ئېلىشقا خېلىدىن بېرى چىدىمىغان رىم ھۆكۈمرانلىرى (ئېمپىراتۇرلىرى) پىلىدىن يىپەك چىقىرىپ ئۆزلىرىنىڭ ئېھتىياجىغا لازىم بولىدىغان كىمخاپ ، شايى ، تاۋار ، دوردۇن قاتارلىق يىپەك ماللارنى ئۆزلىرى ئىشلەپچىقىرىش ئۈچۈن باش قاتۇرۇشقان ئىدى.



ئۇلارنىڭ بۇ ئارزۇ – ئۈمىدلىرى تۈرك خانلىقى دەۋرىدە ئەمەلگە ئاشتى. بۇ توغرىدا يۇنان ئالىملىرى مۇنداق مەلۇمات بېرىدۇ : 6 – ئەسىرنىڭ ئاخىرىدا ياشىغان يۇنان ئالىمى زوناراس ( ) "تارىخنامە" ناملىق ئەسىرىدە مۇنداق دەپ يازغان ئىدى : "بۇرۇن رومانلىقلار يىپەك ئىشلەپچىقىرىش ئۇسۇلىنى بىلمەيتتى. يىپەكنى پىلە قۇرۇتىنىڭ قۇسىدىغانلىقىنى بىلمەيتتى ".



6 – ئەسىرنىڭ ئاخىرىدا ياشىغان يۇنان ئالىمى تىئوخانۇس ئىسمى نامەلۇم بىر ئىرانلىقنىڭ پىلە قۇرۇتى ئۇرۇقىنى قاچان ، نەدىن ، قانداق ھىلە بىلەن ئوغۇرلۇقچە شەرقىي روما ئېمپىرىيىسىنىڭ (مىلادىنىڭ 395 – يىلىدىن 1451 – يىلىغىچە ھۆكۈم سۈرگەن) ئېمپىراتۇرى ئۇلۇغ پوستىئاننىڭ ۋاقتىدا (مىلادىنىڭ 527 – يىلىدىن 565 – يىلىغىچە ئېمپىراتۇر بولغان) شەرقىي روماغا ئېلىپ بارغانلىقى ۋە شۇنىڭدىن كېيىن شەرقىي رومالىقلار پىلە قۇرۇتىنى بېقىپ ، يىپەك ماللارنى ئىشلەپچىقىرىشنى يولغا قويغانلىقى توغرىسىدا مەلۇمات بېرىدۇ. تىئوفانىس مۇنداق دەپ يازىدۇ : "ئېمپىراتۇر پوستىئاننىڭ ۋاقتىدا ئىرانلىق مەلۇم بىر ئادەم شەرقىي روماغا پىلە بېقىپ يىپەك ئىشلەپچىقىرىش ئۇسۇلىنى ئۆگەتكەن. ئۇنىڭدىن بۇرۇن شەرقىي رومالىقلار پىلە بېقىشنى بىلمەيتتى. ھىلىقى ئىسمى نامەلۇم ئىرانلىق سىرىسلەر ئېلىدە (يىپەك ئېلىدە) ئۇزۇن تۇرغان. ئۇ سىرىسلەر ئېلىدىن قايتىپ كېتىدىغان چاغدا پىلە قۇرۇتىنىڭ ئۇرۇقىنى يول ماڭغاندا ئىچى كاۋاك ھاسىسىنىڭ ئىچىگە يوشۇرۇن جايلاشتۇرغان. ئاندىن كېيىن پىلە قۇرۇتىنىڭ ئۇرۇقىنى ئاشۇنداق ئۇستىلىق بىلەن شەرقىي روماغا ئېلىپ كەلگەن. بۇ ئىرانلىق باھار كېلىش بىلەنلا پىلە قۇرۇتىنىڭ ئۇرۇقىنى ئۈژمە يوپۇرمىغىنىڭ ئۈستىگە قويغان ، بىرنەچچە كۈندىن كېيىن پىلە قۇرۇتى يوپۇرماقنى يەپ چوڭ بولۇپ ، بىر جۈپ قانات چىقارغان ، ئۇچىدىغان بولغان ، رومالىقلار پىلە بېقىپ ،يىپەك چىقىرىشنى ئۆگەنگەندىن كېيىن پوستىئان تۈركلەرگە (كونىستانتىنپۇلغا – ئىستانبۇلغا كەلگەن تۈرك ئەلچىلىرىگە ئاپتور) رومالىقلارنىڭ پىلە بېقىپ يىپەك چىقىرىش ئۇسۇلىنى كۆرسەتكەندە تۈركلەر ھاڭ – تاڭ قېلىپ چۇچۇپ كەتكەن ".



تۈركلەر بىلەن شەرقىي روما ئارىسىدىن سودىدا تۈركلەر ئارقىلىق شەرقىي روماغا ئېلىپ بېرىلىدىغان ماللارنىڭ ئىچىدە ئاساسى سالماقنى يىپەك توقۇلمىلىرى ئىگەللەيدىغانلىقى ، خوتەننى ئاساس قىلغان ئۇيغۇر يىپەكچىلىكىنىڭ ئەڭ قەدىمكى چاغلاردىن تارتىپ دۇنياغا مەشھۇرلىقى ، پىلە قۇرۇتى ئۇرۇقىنىڭ "يىپەك مەلىكىسى" توغرىسىدىكى رىۋايەت بويىچە مىلادىنىڭ 440 – يىلى جۇڭگودىن خوتەنگە كەلتۈرۈلگەنلىكىنىڭ يالغانلىقى ، پىلە قۇرۇتى ئۇرۇقىنىڭ سىرىسلەر ئېلى (ئاساسەن خوتەن) دىن شەرقىي روماغا ئىسمى نامەلۇم بىر ئىرانلىق راھىپ تەرىپىدىن مىلادىنىڭ 550 – يىللىرى ئوغۇرلۇقچە ئېلىپ بېرىلغانلىقى توغرىسىدىكى پاكىتلارنى يۇقىرىدا بىر – بىرلەپ سۆزلەپ ئۆتتۈم. مېنىڭچە شۇ پاكىتلار يىتەرلىك بولسا كېرەك.



ئاخىرقى سۆز قەدىمكى چاغدىكى ئاسىيادىن ياۋرۇپاغا بارىدىغان خەلقئارا سودا يولى يالغۇزلا ئاسىيادىكى خەلقلەر بىلەن ياۋرۇپادىكى خەلقلەر ئارىسىدا ئىقتىسادى جەھەتتە قىممەتلىك تاۋارلارنى ئۆزئارا ئالماشتۇرۇش رولىنى ئويناپلا قالماستىن، بەلكى مەدەنىيەت جەھەتتە ئاسىيادىكى خەلقلەر بىلەن ياۋرۇپادىكى خەلقلەرنىڭ ئۆزئارا بىر – بىرىنىڭ ئارتۇقچىلىقىنى ئۆگىنىپ شانلىق مەدەنىيەت يارىتىشىمۇ ناھايىتى كۈچلۈك تۈرتكىلىك روللارنى ئوينىغان. خەلقئارا سودا يولىدا سودا كارۋانلىرى ، ئەلچىلەر ، سەيياھلار غەربتىن شەرققە ، شەرقتىن غەربكە موكىدەك ئۆتۈشۈپ تۇراتتى. بەزىدە بولسا تاجاۋۇزچىلىق مەقسىتى بىلەن سەپەرۋەر قىلىنغان تۈمەنلىگەن قوشۇنلارمۇ سودا ئەلچىسى ياكى ئاددى سودىگەر قىياپىتىگە كىرىۋالغان جاسۇسلارمۇ قاتناپ تۇراتتى. غەرب بىلەن شەرق ئوتتۇرا ئاسىيا ئارقىلىق توۋار ئالماشتۇرۇش ئىشلىرىنى يۈرگۈزگەندىن تاشقىرى ، مەخپىي ئاخبارات يىغىش ئىشلىرىنى قىلاتتى. باشتا ئېيتىپ ئۆتكۈنىمىزدەك مىلادىنىڭ 550 – يىللىرى ئىسمى نامەلۇم بىر ئىرانلىقنىڭ ئۇيغۇرلار يۇرتىدىن پىلە قۇرۇتىنىڭ ئۇرۇقىنى ئوغرىلاپ مەخپى ھالدا شەرقىي روماغا يەتكۈزۈپ بېرىپ يىپەك ئىشلەپچىقىرىشىدا ھۆكۈم سۈرگەن نەچچە مىڭ يىللىق سىرنىڭ ئىشىكىنى ئېچىپ تاشلىغان. شۇ ئىرانلىق پىلە قۇرۇتىنىڭ ئۇرۇقىنى شەرقىي روماغا يەتكۈزۈپ بېرىشتىن بۇرۇن شەرقىي روما ئېمپىراتورى ئۇلۇغ پوستىئان (مىلادىنىڭ 527 – يىلىدىن 565 – يىلىغىچە ئېمپىراتور بولغان) بىلەن شۇ مەسىلە توغرىسىدا سۆزلەشكەن. پوستىئان ئەگەر ئىرانلىق راھىپ پىلە قۇرۇتىنىڭ ئۇرۇقىنى سىرىسلەر ئېلىدىن ئوغرىلاپ ئېلىپ كېلىدىغان بولسا ئۇنىڭغا ناھايىتى قىممەت باھالىق ئىنئام (مۇكاپات) بېرىدىغان بولغان.



ئېھتىمال ئىرانلىق راھىپ خىرىستىئان مۇرىدى بولغانلىقى ، پوستىئاندىن ئالىدىغان قىممەت باھالىق ئىنئامغا قىزىققانلىقى ئۈچۈن ، سىرىسلەر ئېلىدىن پىلە قۇرۇتىنىڭ ئۇرۇقىنى ئوغرىلاپ شەرقىي روماغا يەتكۈزۈپ بەرگەن. مانا مۇشۇ ۋەقەدىن تەخمىنەن 140 يىل بۇرۇن (مىلادىنىڭ 424 – يىلى) غەربنىڭ سۈزۈك ، رەڭلىك شىشە ياساشتىكى مەخپىي سىرىنى ئاق ھون سودىگەرلىرى (ئۇلۇغ ياۋچىلار) جۇڭگوغا بىلدۈرۈپ قويغان ، ئۇنىڭدىن بۇرۇن خەنزۇلار رەڭلىك شىشە ياساشنى بىلمەيتتى. ئۇلار رەڭلىك شىشىنى ناھايىتى يۇقىرى باھادا غەربتىن (سۈرىيىدىن) سېتىپ ئېلىشقا مەجبۇر ئىدى. شۇنىڭ ئۈچۈن خەنزۇلار ئۇزۇندىن بېرى رەڭلىك شىشە ياساشنىڭ سىرىنى غەربتىن بىلىپ ئېلىشقا ئۇرۇنۇپ كەلگەنىدى. بۇنداق ئۇرۇش مىلادىنىڭ 424 – يىلى ياۋچىلارنىڭ ياردىمى بىلەن ئەمەلگە ئاشتى.



بۇ ئىشقا مۇناسىۋەتلىك تارىخى پاكىتلار مۇنداق : "شىمالى سۇلالىلەر تارىخى"دىكى خاتىرىگە ئاساسلانغاندا شىمالى ۋېي خانلىقى (مىلادىنىڭ 386 – يىلىدىن 534 – يىلىغىچە ھۆكۈم سۈرگەن)نىڭ خانى تۇباتاۋنىڭ ۋاقتىدا (مىلادىنىڭ 424 – يىلىدىن 452 – يىلىغىچە خان بولغان) مىلادىنىڭ 424 – يىلى ئۇنىڭ پايتەختىگە كەلگەن ئاق ھون سودىگەرلىرى (ئۇلۇغ ياۋچىلار) تاغدىن دورا ئېلىپ كېلىپ ئۇنى ئېرىتىپ رەڭلىك شىشە ياسىغان. بۇ كىشىلەر غەربنىڭ رەڭلىك شىشىلىرىدىنمۇ پاقىراق ، چىرايلىق چىققان. بۇنداق رەڭلىك شىشىنى كۆرگەنلەر ھەيران قېلىشىپ ئىلاھى شىشىلەر دىيىشكەن. ئاق ھون سودىگەرلىرى رەڭلىك شىشە ياساشنى خەنزۇلارغا ئۆگەتكەن. شۇنىڭدىن باشلاپ سۈرىيە ، يۇنانلىقلارنىڭ رەڭلىك شىشىنى مونوپولىيە قىلىۋېلىش ھوقۇقى پاچاقلىنىپ كەتكەن. قەدىمكى تارىخىي پاكىتلار بويىچە ھۆكۈم قىلغاندا بىرەر مىللەت ياكى خەلقنىڭ ئىجاد قىلغان بىر خىل تېخنىكىسى غەرب ئەللىرىدە پۈتۈنلەي قەتئى مەخپىي ئىدى. شۇنىڭغا قارىغاندا تۇباتاۋ شىمالى ۋېي خاندانلىقىنىڭ پايتەختى داتۇڭغا كەلگەن سودىگەرلەرنىڭ رەڭلىك شىشە ياساش تېخنىكىسىنى بىلىدىغانلىقىنى بىلىپ قالغان بولسا كېرەك. ئەگەر ئەھۋال بىزنىڭ قىياسىمىزدىكىدەك بولىدىغان بولسا ، ئۇ چاغدا تۇباتاۋ ئاق ھون سودىگەرلىرىگە نىسبەتەن بېسىم كۈچ ئىشلىتىپ ئۇلارنى رەڭلىك شىشە ياساش تېخنىكىسىنى ئۆگىتىشكە مەجبۇرلىغان ياكى ناھايىتى يۇقىرى باھالىق بەدەل بېرىشكە ۋەدە قىلىپ ئالىدىغان بولۇشى مۇمكىن. يۇقىرىدا تىلغا ئېلىنغان ئىرانلىق راھىپ بىلەن ئاق ھون (مىلادىنىڭ 420 – يىلىدىن 565 – يىلىغىچە ھۆكۈم سۈرگەن) سودىگەرلىرى ماددى مەدەنىيەت ئىچىدە ناھايىتى ئېسىل قىممەتلىك بولغان يىپەك بىلەن رەڭلىك شىشىنى ئىشلەپچىقىرىشنىڭ سىرلىرىنى بىلمەيدىغانلارغا مەلۇم قىلىپ قويغان مەدەنىيەت ئوغرىلىرى ، جاسۇسلىرى بولسىمۇ ، مەلۇم نۇقتىدىن قارىغاندا ئۇلار شەرقىي روما بىلەن جۇڭگونىڭ ماددى مەدەنىيىتىنى يۈكسەلدۈرۈشتە ناھايىتى چوڭ رول ئوينىغان.



خەنزۇلار ئۇيغۇرلاردىن ۋە ئۇيغۇرلار ئارقىلىق ماددى ۋە مەنىۋى مەدەنىيەتكە دائىر نۇرغۇن ئېسىل قىممەتلىك بايلىقلارغا ئىگە بولغان. جاڭ چيەن مىلادىدىن 115 يىل بۇرۇن ئوتتۇرا ئاسىياغا ئىككىنچى قېتىم كېلىپ كەتكەندىن كېيىن ، ئۇ ئۇيغۇرلار يۇرتىدىن ئۈزۈم ۋە بىدە ئېلىپ كەتكەن. ئۈزۈم بىلەن بىدە پادىشاھ خەنۋۇدىنىڭ ناھايىتى قەدىرلەپ ، ئېتىبار بېرىشكە ئېرىشكەن. پادىشاھ ئوردىسىنىڭ ئەتراپىدا ئۈزۈملۇك باغلار ، بىدىلىكلەر پەيدا بولغان. بولۇپمۇ بىدە ئاتنىڭ ئاساسلىق يەم – خەشەكلىرىدىن بىرى بولۇپ ، ئات بېقىشقا ناھايىتى مۇھىم ئورۇندا تۇراتتى. شۇ چاغلاردا خەنۋۇدى ھونلار بىلەن بولىدىغان ئۇرۇشلاردا ئاتلىق قوشۇننىڭ ناھايىتى مۇھىملىقىنى چوڭقۇر ھېس قىلغان. ئۇ جەڭ ئاتلىرىنى كۆپلەپ ئۆستۈرۈشكە ئالاھىدە كۆڭۈل بۆلۈۋاتقان ئىدى. جاڭ چيەن ئوتتۇرا ئاسىيادىن قايتىشىدا ئۈزۈم بىلەن بىدىدىن باشقا يەنە زىغىر ، قارامۇچ ، پىياز ، تۇرۇپ ، قوغۇن ، تاۋۇز ، كاۋا قاتارلىقلارنىڭ ئۇرۇقىنىمۇ ئېلىپ كەتكەن. شۇنىڭدىن باشلاپ ئاشۇ نەرسىلەر جۇڭگودا تېرىلىشقا باشلىغان. دەل شۇ چاغدا ياڭاق بىلەن ئانار كۆچىتىمۇ ئۇيغۇرلار يۇرتىدىن ئىچكىرىگە كەلتۈرۈلگەن. خەنزۇلار مىلادىنىڭ 5 – ئەسىرىگىچە پاختىدىن رەخ ئىشلەپچىقىرىشنى بىلمەيتتى. پەقەت 5 – ئەسىردىلا كېۋەز ئۇرۇقى (چىگىت) تۇرپاندىن ئىچكىرىگە كەلتۈرۈلگەندىن كېيىنلا پاختا رەخت ئىشلەپچىقىرىش باشلانغان. ئۇنىڭغىچە خەنزۇلار پاختا رەختلەرنى ئوتتۇرا ئاسىيادىن ۋە باشقا مەملىكەتلەردىن ئېلىپ كېلەتتى. قەدىمكى چاغلاردا قۇنقاۋ ، پىپا ، بالابار قاتارلىق مۇزىكا ئەسۋابلىرى ئىراندىن جۇڭگوغا كىرگەن بولسا ، داپ ، نەي ، سۇناي قاتارلىق مۇزىكا ئەسۋابلىرى ، ئوتتۇرا ئاسىيادىن (ئۇيغۇرلاردىن) كىرگەن. بولۇپمۇ سۈي سۇلالىسى (581 – 618) ، تاڭ سۇلالىسى (618 – 907) دەۋرىدە ئۇيغۇرلار جۇڭگو مۇزىكا سەنئىتىنىڭ تەرەققىياتىغا ئۆچمەس تۆھپىلەرنى قوشقان. مىلادىنىڭ 579 – يىلى كۆك تۈرك خانلىقىنىڭ بىكە (مەلىكە) لىرىدىن ئاسىنا بىكە ، شىمالى جۇ سۇلالىسى (557 – 581) نىڭ خانى جۇشۇەندىغا ياتلىق قىلىنغاندا ، تۈرك خاقانلىقىنىڭ پايتەختى ئۇتۇ كۈندىن چاڭئەنگە (شىئەنگە) ئاسىنا بىكە بىلەن بىللە سۇ جاۋا ئاتلىق مەشھۇر مۇزىكا ئۇستازىمۇ كەلگەن. ئۇ كۇچارلىق بولۇپ ، سۈي سۇلالىسى ۋاقتىدا (581 – 618) چاڭئەندە ئۇزۇن مۇددەت تۇرغان. سۇجاۋا شۇ مەزگىلدە ، جۇڭگو مۇزىكا سەنئىتىنىڭ بەكمۇ تۆۋەن ، نامراتلىقىنى نەزەردە تۇتۇپ جۇڭگو مۇزىكىچىلىقىنىڭ مەزمۇنىنى بېيىتىش ، ئۇنىڭ سەنئىتىنى ناھايىتى يۇقىرى كۆتۈرۈش ئۈچۈن مۇكەممەل بولغان توققۇز يۈرۈش مۇزىكا سېستىمىسىنى بەرپا قىلغان. ئۇلار تۆۋەندىكىچە :



كۇچار مۇزىكىسى

قەشقەر مۇزىكىسى ،

بۇخارا مۇزىكىسى

سەمەرقەنت مۇزىكىسى

غەربىي لياۋ مۇزىكىسى

ھىندى مۇزىكىسى ،

كۈرىيە مۇزىكىسى ،

چىڭ مۇزىكىسى ،

لى خۇا مۇزىكىسى.

بۇنىڭ ئىچىدە چىڭ مۇزىكىسى بىلەن لى خۇا مۇزىكىسى جۇڭگونىڭ ئەنئەنىۋى مۇزىكىسى ئىدى. بۇنىڭدىن باشقا يەتتە خىل مۇزىكىنىڭ ئىچىدىن ھىندى مۇزىكىسى بىلەن كۈرىيە مۇزىكىسىنى ھېسابقا ئالمىغاندا ، قالغان بەش خىل مۇزىكا كۇچار مۇزىكىسىنى ئاساس قىلغان ئۇيغۇر مۇزىكىسى ئىدى.



سۇجاۋا پىيبا چېلىشقا ناھايىتى ئۇستا بولۇپ ، شۇ چاغدا جۇڭگودىن ئۇنىڭغا تەڭ كەلگەندەك بىرەر مۇزىكانت چىققان ئەمەس. شۇڭا ئۇ مۇزىكا ئۇستازى دېگەن ھۆرمەتلىك نامغا ئىگە بولغان.



جۇڭگو تاڭ سۇلالىسى دەۋرىگە كەلگەندە (618 – 907) سۇجاۋا يارىتىپ بەرگەن توققۇز يۈرۈش مۇزىكا سېستىمىسى ئاساسىدا 640 – يىلى 10 يۈرۈش مۇزىكا سېستىمىسىنى بەرپا قىلدى. بۇلار تۆۋەندىكىچە : 1. كۇچار مۇزىكىسى ، 2. قەشقەر مۇزىكىسى ، 3. تۇرپان مۇزىكىسى ، 4. غەربىي لياۋ مۇزىكىسى ، 5. بۇخارا مۇزىكىسى ، 6. سەمەرقەنت مۇزىكىسى ، 7. كامبودژا مۇزىكىسى ، 8. كۈرىيە مۇزىكىسى ، 9. يەن مۇزىكىسى ، 10. چىڭ شاڭ مۇزىكىسى. مانا شۇ ئون يۈرۈش مۇزىكا سېستىمىسى ئىچىدە ئالتە خىل مۇزىكا (بۇنىڭغا تۇرپان مۇزىكىسى قوشۇلغان) كۇچار مۇزىكىسىنى ئاساس قىلغان ئۇيغۇر مۇزىكىسى ئىدى. بۇنىڭ ئىچىدىكى غەربىي لياۋ مۇزىكىسىمۇ مەلۇم دەرىجىدە خەنزۇ مۇزىكىسىنىڭ تەسىرىگە ئۇچرىغان ھون مۇزىكىسى بولۇپ ، قەدىمكى ئۇيغۇر مۇزىكىسى ئىدى. بۇنىڭدىن تاشقىرى كۇچار ئۇسۇلىنى ئاساس قىلغان ئۇيغۇر ئۇسۇل سەنئىتىمۇ ، ئۇيغۇر مىللى ھەيكەلتاراشلىقى بىلەن رەسسامچىلىقنىڭ ئەنئەنىۋى ئالاھىدىلىكلىرىنى ئاساس قىلغان بۇددا سەنئىتىمۇ (ئاساسەن ھەيكەلتاراشلىق ۋە رەسساملىق) خەنزۇلارنىڭ ھەيكەلتاراشلىق ۋە رەسساملىق سەنئىتىگىمۇ كۈچلۈك تەسىر قىلغان. بۇنىڭغا دۇڭخۇاڭ مىڭ ئۆيىدىكى ھەيكەللەر ، تام سىزما رەسىملىرى ئىسپات بولىدۇ. شۇنىمۇ ئېيتىپ ئۆتۈش كېرەككى ، 7 – ئەسىرنىڭ باشلىرىدا خوتەنلىك ئۇيغۇر رەسسامى يۇچىيزىڭ تاڭ سۇلالىسى پادىشاھى لى شىمىننىڭ (627 – 649) تەكلىۋىگە بىنائەن شىئەنگە كېلىپ ساكيامۇننىڭ تىپىك سۈرئىتىنى سىزىپ بەرگەن. جۇڭگو رەسساملىرى بۇددا رەسسامچىلىقىدىكى رەسىم سىزىش سەنئىتىنى ۋە غەرب ئەللىرىدىكى كىشىلەرنىڭ رەسىمىنى سىزىش سەنئىتىنى ئۇنىڭدىن ئۆگەنگەن. چۈنكى يۇچىيزىڭ رەسسامچىلىقتا (غەرب ئۇسلۇبىدىكى رەسسامچىلىقتا) تەڭدىشى يوق رەسسام ئىدى. جۇڭگونىڭ قەدىمكى چاغدىكى مەنىۋى مەدەنىيەتنىڭ (پەلسەپە بىلەن ئەدەبىياتنىڭ) بېيىشى ۋە يۈكسۈلۈشىنى ئۇيغۇرلارنىڭ قوشقان ئۆچمەس تۆھپىسىدىن ئاجرىتىپ قاراش تولىمۇ بىمەنىلىك بولىدۇ. بۇددا دىنى ئەڭ دەسلەپ جۇڭگوغا تارقىلىشتىن بۇرۇن ئۇيغۇرلار ئارىسىدا يىلتىز تارتقان ئىدى. ئۇيغۇلار ئىچىدىن چىققان مەشھۇر بۇددىست ، راھىبلار 3 – ئەسىردىن باشلاپ بۇددا دىنىنى جۇڭگوغا تونۇشتۇرۇشقا كىرىشكەن. بۇنىڭغا كۇچارلىق ئۇلۇغ بۇددىست پەيلاسۇپ ۋە مەشھۇر تەرجىمان ، شائىر كوماراجىۋانى مىسال كەلتۈرۈش يېتەرلىك. كوماراجىۋا (مىلادىنىڭ 344 – يىلى تۇغۇلۇپ 431 – يىلى ئالەمدىن ئۆتكەن) كىچىك ۋاقتىدا دادىسىغا ئەگىشىپ ھىندىستانغا بارغان. ئۇ ھىندىستاندا كۆپ يىللار تۇرۇپ بۇددىزىم ئەقىدىلىرىنى تىرىشىپ ئۆگەنگەن. شۇ چاغلاردا بۇددىزىم تەلىماتىدا ماخىيانا ۋە خىنيايانا دەپ ئاتىلىدىغان چوڭ ئىككى مەزھەپ ، ئىككى ئېقىم بار ئىدى. كوماراجىۋا بۇ ئىككى چوڭ مەزھەپ ئىچىدىن خىنيايانا مەزھىپىنى قوبۇل قىلغان. ئۇ ھىندىستاندىن قايتىپ كېلىۋاتقان چاغدا قەشقەردە شورى ئاساما ئاتلىق بۇددا راھىپىغا يولۇققان ۋە ئۇنىڭدىن ماخىيانا مەزھىپىنى ئۆگىنىپ بۇرۇن قوبۇل قىلغان خىنيانا مەزھىپىدىن ۋاز كەچكەن. كوماراجىۋا قەشقەردىن ئۆتۈپ ئاقسۇغا كەلگەندە ئاجايىپ بىر راھىپقا يولۇققان ، ئۇ راھىپ ئۆزىنى دۇنيادا تەڭداشىسز راھىپ ھېسابلاب ھەمىشە ماختىنىپ : "ئەگەر بۇددا ئەقىدىلىرى بويىچە بىرەر كىشى مەن بىلەن مۇنازىرىلىشىپ مېنى يېڭىدىغان بولسا ئۇنداق كىشىگە بېشىمنى كېسىپ بېرىمەن" دەيدىكەن. كوماراجىۋا شۇ راھىپ بىلەن مۇنازىرىلىشىپ ئۇنى يېڭىپ قويىدۇ. بۇ ۋەقە غەربتىكى بۇددا دۇنياسىنى زىلزىلىگە كەلتۈرىدۇ. كوماراجىۋانىڭ داڭقى غەرب ئەللىرىگە تارقىلىدۇ. بۇ ۋەقەدىن كۇچار خانى ناھايىتى چوڭ پەخىرلىنىش ھېس قىلىپ كوماراجىۋانىڭ ئالدىغا – ئاقسۇغا كېلىپ ئۇنى ئىززەت – ھۆرمەت بىلەن كۇچارغا ئېلىپ كېتىدۇ. كوماراجىۋانى كۇچار دۆلىتىنىڭ پىرى ئۇستازى دەپ جاكالايدۇ. تارىختىن خەۋرى بار كىشىلەرگە مەلۇم بولغاندەك كوماراجىۋا مەلۇم سەۋەبلەر بىلەن مىلادىنىڭ 385 – يىلى ھازىرقى گەنسۇنىڭ ئوۋىيغا كەلگەن. ئۇ شۇ جايدا 15 يىل تۇرغان. كوماراجىۋا شۇ مەزگىلدە خەنزۇ يېزىقىنى (تىلىنى) پۇختا ، مۇكەممەل ، چوڭقۇر ئۆگەنگەن. شۇ چاغدىكى جۇڭگو تارىخىدا "ئىككى چىن سۇلالىسى" بىلەن "شىمالى ۋە جەنۇبى سۇلالىلەر دەۋرى" دەپ ئاتالغان ناھايىتى قالايمىقان (ئاساسەن ئۇرۇش قالايمىقانچىلىقى) بىر دەۋر ئىدى. جۇڭگودىكى بەزى خانلىقلارنىڭ ئالىي ھۆكۈمرانلىرى بۇددا دىنىنىڭ نادان ، ساددا خەلقلەرنى ئاسانلا ئالداپ كېتەلەيدىغان پاسسىپ ئىدىيىسىدىن پايدىلىنىپ خەلقنى بىخۇتلاشتۇرۇپ ئۆزلىرىگە قارشى چىقمايدىغان يۇۋاش پۇقرالارغا ئايلاندۇرۇش ئۈچۈن ھىچنەرسىگە قارىماي بارلىق تىرىشچانلىقلارنى كۆرسىتىپ ، بۇددا دىنىنى تەشەببۇس قىلىۋاتقان ئىدى. بۇددا دىنىغا كۈچلۈك تەرپدار بولغان كېيىنكى چىن خانلىقى (384 – 417) نىڭ خانى ياۋشىيىن (مىلادىنىڭ 394 - يىلىدىن 415 – يىلىغىچە خان بولغان) مىلادىنىڭ 401 – يىلى كوماراجىۋانى تەكلىپ قىلىپ شىئەنگە ئېلىپ كەلگەن ۋە ئۇنى دۆلەتنىڭ پىرى ئۇستازى دەپ جاكالىغان. ياۋشىيىننىڭ يارىتىپ بەرگەن كەڭ ئىمكانىيىتىدىن پايدىلانغان كوماراجىۋا شىئەندىكى مەشھۇر بۇددا ئىبادەتخانىلىرىنىڭ بىرىدە 3000 شاگىرتقا بۇددا ئەقىدىلىرىدىن دەرس بەرگەن. ياۋشىيىن دائىم ئىبادەتخانىغا بېرىپ ئۇنىڭ دەرسىنى ئاڭلىغان. كېيىنكى چاغلاردا ئۇنىڭ شاگىرتلىرىدىن بىرقانچىسى "دانىشمەن – ئەۋلىيا" دېگەن نامغا ئىگە بولغان. كوماراجىۋا شىئەندە بۇددا ئەقىدىلىرىدىن دەرس ئۆتكەندىن تاشقىرى بۇددا نامىلاردىن 384 جىلىدلىق 74 پارچە كىتابنى قەدىمكى ھىندى تىلىدىن خەنزۇچىغا تەرجىمە قىلغان. بۇنىڭدىن باشقا بەدىئىلىگى ناھايىتى كۈچلۈك بولغان بۇددا رىۋايەتلىرىدىن ناھايىتى نۇرغۇن ئەسەرلەرنى قەدىمكى ھىندى تىلىدىن خەنزۇچىغا تەرجىمە قىلغان. ئۇنىڭ قولىدىن چىققان تەرجىمە ئەسەرلەرنىڭ تىلى بەكمۇ راۋان ، مەزمۇنى چوڭقۇر بولسىمۇ ، چۈشىنىشلىك بولغان. كوماراجىۋانىڭ دىن ، پەلسەپە ، ئەدەبىيات ساھەسىدىكى پائالىيىتى جۇڭگو بۇددىزىم پەلسەپىسىنىڭ ۋە ئەدەبىياتنىڭ تەرەققىياتىغا ناھايىتى كۈچلۈك تەسىر كۆرسەتكەن. شۇنىڭ ئۈچۈن ، ئۇ ، ئۇلۇغ بۇددىست ، پەيلاسوپ ، داڭلىق تەرجىمان ، تالانتلىق شائىر دېگەن نامغا ئىگە بولغان. خۇلاسە قىلىپ ئېيتقاندا ، قەدىمكى چاغدىن تارتىپ 15 – ئەسىرگىچە (ئۇلۇغ ساياھەتچى ماگېلان غەرب بىلەن شەرق ۋە شەرق بىلەن غەرب ئارىسىدىكى دېڭىز ، ئوكيان يولىنى ئاچقۇچى) ئاسىيا بىلەن ياۋرۇپا ئارىسىدىكى خەلقئارا قاتناش يولى (ئاتالمىش يىپەك يولى) يالغۇزلا غەرب بىلەن شەرق ئەللىرى ئارا سودا ئارقىلىق ئىقتىساد ئالماشتۇرۇش يولى بولۇپلا قالماستىن ، مەدەنىيەت ئالماشتۇرۇش ئارقىلىق شەرق ئەللىرى بىلەن غەرب ئەللىرىنىڭ قەدىمكى ئەنئەنىۋى مىللى مەدەنىيىتى ئاساسىدا شانلىق ، يۈكسەك مەدەنىيەت يارىتىشىدا ئالتۇن كۆۋرۈكلۈك رول ئوينىغان. بولۇپمۇ بۇ ئۇلۇغ تارىخىي مۆجىزىدە ئوتتۇرا ئاسىيا خەلقلىرى ئاساسى (ئۆگىتىش ۋە تونۇشتۇرۇش جەھەتتە) رولىنى ئوينىغان.



ئوتتۇرا ئاسىيا خەلقلىرى ئۆز تارىخىنىڭ ئۇزاق داۋام قىلغان قەدىمكى باسقۇچلىرىدا ئىران ، ئەھمانلار سۇلالىسى (مىلادىدىن 700 يىل بۇرۇنقى چاغدىن تارتىپ مىلادىدىن 328 يىل بۇرۇنقى چاغقىچە ھۆكۈم سۈرگەن) ، گىرىك باكتىرىيە پادىشاھلىقى (مىلادىدىن 250 يىل بۇرۇنقى چاغدىن تارتىپ مىلادىدىن 150 يىل بۇرۇنقى چاغقىچە ھۆكۈم سۈرگەن) ، ياۋچىلار دۆلىتى بولغان توخارىستان (مىلادىدىن 150 يىل بۇرۇنقى چاغدىن تارتىپ مىلادىدىن 50 يىل بۇرۇنقى چاغقىچە ھۆكۈم سۈرگەن) ، كوشات ئېمپىرىيىسى (مىلادىدىن 50 يىل بۇرۇنقى چاغدىن تارتىپ مىلادىنىڭ 420 – يىلىغىچە ھۆكۈم سۈرگەن) ،ئاق ھون ئېمپىرىيىسى (مىلادىنىڭ 420 – يىلىدىن 565 – يىلىغىچە ھۆكۈم سۈرگەن) قاتارلىق دۆلەتلەرنىڭ قەدىمكى چاغدىكى شانلىق مەدەنىيىتىنى يارىتىشىدا قاتناشقان بولسا ، تۈرك خانلىقى دەۋرىدىكى (552 – 745) ناھايىتى يۈكسەك مەدەنىيەتنى ياراتقۇچىلار قاتارىدىن ئۈستۈن ئورۇن ئالغان ئىدى.


\end{flushright}
\end{document}
