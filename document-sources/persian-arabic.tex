\documentclass[a4paper]{article}

\usepackage{url}
\usepackage{enumitem}
\usepackage{setspace}
\usepackage[hang]{footmisc}
\usepackage{fontspec}
\usepackage{polyglossia}
\usepackage{titlesec}
\usepackage{xcolor}

\usepackage[
  bookmarks=true,
  colorlinks=true,
  linkcolor=linkcolor,
  urlcolor=linkcolor,
  citecolor=linkcolor,
  pdftitle={الخط الأميري},
  pdfsubject={توثيق خط المتون النسخي، الخط الأميري},
  pdfauthor={خالد حسني},
  pdfkeywords={خط, عربي, مطبعة, أميرية, أميري, يونيكود, أوبن تيب}
  ]{hyperref}

\definecolor{textcolor}  {rgb}{.25,.25,.25}
\definecolor{pagecolor}  {rgb}{1.0,.99,.97}
\definecolor{titlecolor} {rgb}{.67,.00,.05}
\definecolor{linkcolor}  {rgb}{.80,.00,.05}
\definecolor{codecolor}  {rgb}{.90,.90,.90}

\setmainlanguage {arabic}
\setotherlanguage{english}
\rightfootnoterule

\setmainfont               [Path=./generated/,Ligatures=TeX]                     {Jomhuria-Regular.ttf}
\setmonofont               [Scale=MatchLowercase]              {DejaVu Sans Mono}
\newfontfamily\arabicfont  [Path=./generated/,Script=Arabic,Numbers=Proportional]{Jomhuria-Regular.ttf}
\newfontfamily\arabicfonttt[Script=Arabic,Scale=MatchLowercase]{DejaVu Sans Mono}

\newcommand\addff[1]{\addfontfeature{RawFeature={#1}}} % add feature
\newcommand\addfl[1]{\addff{language=#1}}              % add language

\setlength{\parindent}{0pt}
\setlength{\parskip}{1em plus .2em minus .1em}
%setlength{\emergencystretch}{3em}  % prevent overfull lines
\setcounter{secnumdepth}{0}

\newfontfamily\titlefont[Path=./generated/,Script=Arabic]{Jomhuria-Regular.ttf}

\titleformat*{\section}{\Large\titlefont\color{titlecolor}}
\titleformat*{\subsection}{\large\titlefont\color{titlecolor}}
\titleformat*{\subsubsection}{\itshape\titlefont\color{titlecolor}}

\titlespacing{\section}{0pt}{*4}{*1}
\titlespacing{\subsection}{0pt}{*3}{0pt}
\titlespacing{\subsubsection}{0pt}{*2}{0pt}

\renewcommand\U[1]{\colorbox{codecolor}{\texttt{U+#1}}}

\title{Jomhuria Test-texts 1}

\begin{document}
\pagecolor{pagecolor}
\color{textcolor}

\begin{english}\maketitle\end{english}
\newpage

\begin{flushright}

% \setstretch{1.6}

خوشنویسی

خوشنویسی یا خطاطی به معنی زیبانویسی یا نوشتن همراه با خلق زیبایی است و فردی که این فرایند توسط او انجام می‌گیرد خوش نویس نام دارد، به‌خصوص زمانی که خوش نویسی حرفه ی شخص باشد. گاهی درک خوشنویسی به عنوان یک هنر مشکل است. به نظر می‌رسد برای درک و لذت بردن از تجربه بصری خوشنویسی باید بدانیم خوشنویس افزون بر نگارش یک متن، سعی داشته اثری هنری با ارزش‌های زیبایی شناختی خلق کند.[۱] از این رو خوشنویسی با نگارش سادهٔ مطالب و حتا طراحی حروف و صفحه‌آرایی متفاوت است. همچنین از آنجایی که این هنر جنبه‌هایی از سنت را در دل خود دارد باید آن را تا حدی از تایپوگرافی که مبتنی بر ارزش‌های گرافیکی مدرن و کارهای چاپی است متمایز کرد.

خوشنویسی تقریباً در تمام فرهنگ‌ها به چشم می‌خورد اما در مشرق زمین و به‌ویژه در سرزمین‌های اسلامی و ایران در قله هنرهای بصری واقع است. خوشنویسی اسلامی و بیش از آن خوشنویسی ایرانی تعادلی است حیرت‌انگیز میان تمامی اجزا و عناصر تشکیل دهندهٔ آن. تعادل میان مفید و مورد مصرف بودن از یک سو و پویایی و تغییر شکل یابندگی آن از سوی دیگر؛ تعادل میان قالب و محتوا که با آراستگی و ملایمت تام و تمام می‌تواند شکل مناسب را برای معانی مختلف فراهم آورد.

اگر در نظر داشته باشیم که خوشنویسی اسلامی و ایرانی براساس قالب‌ها و قواعد و نظام‌های بسیار مشخصی شکل می‌گیرد و هر یک از حروف قلم‌های مختلف از نظام شکلی خاص، و تا حدود زیادی غیرقابل تغییر، برخوردارند. آنگاه درمی‌یابیم که ایجاد قلم‌های تازه و یا شیوه‌های شخصی خوشنویسان بر مبنای چه ابداع و رعایت حیرت‌انگیزی پدیدار می‌شود. در خوشنویسی اندازه هر حرف و نسبت آن با سایر حروف با دقت بسیار بالایی معین شده‌است و هر حرف به صورت یک «مدول» ثابت درمی‌آید که تخطی از آن به‌منزله نادیده گرفتن توافقی چندصدساله‌است که همیشه میان خوشنویس و مخاطب برقرار بوده و با رضایت طرفین حاصل می‌آمده‌است. حتی ترتیب شکل کلمات نیز براساس مبانی مشخصی تعیین می‌شده که در «رسم‌الخط»ها و «آداب‌المشق»ها از جانب اساتید بزرگ تنظیم و ارائه شده‌است. قرارگیری کلمات نیز- یعنی «کرسی»- هر سطر هر چند نه به صورت کاملاً پیش‌بینی و تکلیف شده - که ناممکن بوده‌است - اما به‌شکل «سلیقه مطلوب» زمینه زیبایی شناختی پیشنهادی خاص خود را دارا بوده‌است.



خوشنویسی اسلامی

خوشنویسی همواره برای مسلمانان اهمیتی خاص داشته‌است. زیرا در اصل آن را هنر تجسم کلام وحی می‌دانسته‌اند. آنان خط زیبا را نه فقط در استنساخ قرآن بلکه در اغلب هنرها به‌کار می‌بردند.[۵] خوشنویسی یا خطاطی در کلیه کشورهای اسلامی همواره به عنوان والاترین هنر مورد توجه بوده است. مردمان این سرزمین‌ها به‌ویژه ایرانیان در اوج قدرت و نهایت ظرافت در این هنر تجلی یافت و خوشنویسی به مانند محوری در میان سایر هنرهای بصری ایفای نقش کرده‌است. خوشنویسی از قرون اولیه اسلامی تا کنون در کلیه کشورهای اسلامی و مناطق تحت نفوذ مسلمانان با حساسیت و قدرت تمام در اوج جریانات هنری بوده‌است.

قدیمی‌ترین نسخه قرآن به خط کوفی است و به علی بن ابی‌طالب منسوب است. پس از قرن پنجم هجری، خط کوفی تقریباً جای خود را به خطّ نسخ داد که حروف آن بر خلاف حروف زاویه‌دار خطّ کوفی منحنی و قوس‌دار است.[۶]

در اوایل قرن چهارم سال ۳۱۰ هجری قمری ابن مقله بیضاوی شیرازی خطوطی را بوجود آورد که به خطوط ششگانه یا اقلام سته معروف شدند که عبارتند از : محقق، ریحان، ثلث، نسخ، رقاع و توقیع. که وجه تمایز آنها اختلاف در شکل حروف و کلمات و نسبت سطح و دور در هر کدام می‌باشد. همچنین او برای این خطوط قواعدی وضع کرد که یه اصول دوارده‌گانه خوشنویسی معروفند و عبارتند از: ترکیب، کرسی، نسبت، ضعف، قوت، سطح، دور، صعود مجازی، نزول مجازی، اصول، صفا و شأن.[۷]

یک قرن بعد از ابن مقله، ابن بواب برای خط نسخ قواعدی تازه ایجاد کرد و این خط را کامل نمود در قرن هفتم جمال الدین یاقوت مستعصمی به این خط جان تازه‌ای بخشید و قرآن کریم را برای چندین بار کتابت نمود.[۸]



خوشنویسی ایرانی

درحالی که عمدهٔ تبدیل نگارش معمولی کلمات به خوشنویسی هنرمندانه به عهده ایرانیان بوده‌است، رفته رفته ایرانیان سبک و شیوه‌هایی مختص به خود را در خوشنویسی ابداع کردند. هرچند این شیوه‌ها و قلم‌های ابداعی در سایر کشورهای اسلامی هم طرفدارانی دارد اما بیشتر مربوط به ایران و کشورهای تحت نفوذ آن همچون کشورهای آسیای میانه، افغانستان، پاکستان و هند می‌باشد. در این منطقه نیز خوشنویسی همواره به عنوان والاترین شکل هنرهای بصری مورد توجه بوده و دارای لطافتی خاص است.

در ایران پس از فتح اسلام، خطّاطی به شیوهٔ نسخ وجود داشت. در زمان حکومت ایلخانیان، اوراق مذهّب کتاب‌ها نخستین بار با نقش‌های تزیینی زینت یافت. در زمان حکومت تیموریان در ایران، خط و خوشنویسی به اوج کمال خود رسید. میرعلی تبریزی با ترکیب خط نسخ و تعلیق، خط نستعلیق را بنیان نهاد. مشهورترین خطاط قرآن در این دوره بایسنقر میرزا بود.[۹]

آنچه که به عنوان شیوه‌ها یا قلم‌های خوشنویسی ایرانی شناخته می‌شود بیشتر برای نوشتن متون غیر مذهبی نظیر دیوان اشعار، قطعات ظریف هنری و یا برای مراسلات و مکاتبات اداری ابداع شده و به‌کار رفته‌است. این در حالی است که خط نزد اعراب و ترکان عثمانی بیشتر جنبه دینی و قدسی داشته‌است. هرچند آنان نیز برای امور منشی‌گری و غیر مذهبی قلم‌هایی را بیشتر به‌کار می‌برند، اما اوج هنرنمایی آنان - بر خلاف خوشنویسان ایرانی- در خط ثلت و نسخ و کتابت قرآن و احادیث قابل رویت است.

در ایران نیز برای امور مذهبی مانند کتابت قرآن یا احادیث و روایات و همچنین کتیبه‌نویسی مساجد و مدارس مذهبی بیشتر از خطوط ثلث و نسخ بهره می‌گرفتند که نزد اعراب رواج بیشتری دارد. هرچند ایرانیان در این قلم‌ها نیز شیوه‌هایی مجزا و مختص به خود آفریده‌اند.

خط تعلیق را می‌توان نخستین خط ایرانی دانست. خط تعلیق که ترسل نیز نامیده می‌شد از اوایل قرن هفتم پا به عرصه گذاشت و حدود یکصد سال دوام داشت تعلیق از ترکیب خطوط نسخ و رقاع به وجود آمد و کسی که این خط را قانونمند کرد خواجه تاج سلمانی بود. که بعدها به وسیله خواجه عبدالحی منشی استر ابادی بدان قواعد و اصول بیشتری بخشیده شد.[۱۰]

پس از خط تعلیق که نخستین خط شکل گرفته ایرانی بود. در قرن هشتم میرعلی تبریزی (۸۵۰ هجری قمری) از ترکیب و ادغام دو خط نسخ و تعلیق خطی بنام نسختعلیق بوجود آورد که نام آن در اثر کثرت استعمال به نستعلیق تغییر پیدا کرد. این خط بسیار مورد اقبال واقع شد و موجب تحول عظیمی در هنر خوشنویسی گردید.[۱۱] و بعدها در دوران شاه عباس صفوی به دست میرعماد حسنی به اوج زیبایی و کمال رسید.

خط نستعلیق علاوه بر خوشنویسانی که در حیطه ایران کنونی می‌زیستند. در خراسان بزرگ و کشورهای آسیای میانه، افغانستان و به‌ویژه خوشنویسان دربار گورکانیان در هندوستان (که تعلق خاطر ویژه‌ای به فرهنگ ایرانی داشتند) رشد و پیشرفت قابل توجهی کرد.

بطور کلی قرنهای نهم تا یازدهم هجری را می‌توان قرن‌های درخشان در هنر خوشنویسی در ایران دانست. در اواسط قرن یازدهم سومین خط خالص ایرانی یعنی شکسته نستعلیق به دست مرتضی قلی خان شاملو حاکم هرات با تغییر دادن خط نستعلیق ابداع شد. علت پیدایش آن را می‌توان به سبب نیاز به تندنویسی و راحت‌نویسی در امور منشی‌گری و بیش از آن ذوق و خلاقیت ایرانی دانست. همانطوری که بعد از پیدایش خط تعلیق، ایرانیان به‌خاطر سرعت در کتابت، شکسته تعلیق آن‌را نیز بوجود آوردند.[۱۲] درویش عبدالمجید طالقانی این خط را به کمال رساند.

در دوران قاجار خط کاربرد اصلی خود یعنی کتابت را از دست داد. از یک سو با تاکید و توسعه سیاه‌مشق‌نویسی به سمت هنر ناب رفت و از دیگر سو با پدید آمدن روزنامه‌ها و چاپ سنگی خود را به هنرهای کاربردی نزدیک شد. در دوران پهلوی روند نیازهای مدرن تجاری و تبلیغی حط را از کنج کتابخانه‌ها و مجموعه‌های خصوصی به بستر اجتماع کشاند و جریانات هنری مدرنی شیوه‌هایی چون نقاشیخط را به‌وجود آوردند که کماکان مورد توجه محافل هنری جهان است.

در سالهای پس از انقلاب اسلامی خوشنویسی با اقبال عمومی روبرو شد و انجمن خوشنویسان ایران که در سال ۱۳۲۹ تشکیل شده بود نقش بارزی در رشد و توسعه کمی و کیفی این هنر به عهده گرفت. همچنین خطی به نام معلی در سال‌های اخیر با تاثیر از خطوط اسلامی - ایرانی ثلث و شکسته نستعلیق توسط آقای حمید عجمی ابداع گردید که تاحدی مورد استقبال قرار گرفت. آخرین دستاورد خوشنویسی ایران را می توان به ابداعات خوشنویسی سبحان مهرداد محمدپور اشاره کرد ، خط سبحان یک خط قانونمند همانند نسخ و ثلث بوده و قابلیت کتابت و کتیبه نویسی فارسی و عربی را دارا می باشد و آن را در دسته ی خطوط کامل همانند خطوط شش گانه قرار داده اند.



فن الخط

فن الخط وهو نوع من الفنون البصرية. وهو غالبا ما يطلق على فن الكتابة (Mediavilla ١٩٩٦ : ١٧). والتعريف المعاصر لممارسة فن الخط العربي هو "فن إعطاء شكل للعلامات بطريقة معبرة، ومتناغمة، وماهرة" (Mediavilla ١٩٩٦ : 18). قصة الكتابة هي واحدة من التطور الجمالي متحدد في إطار المهارات التقنية، وسرعات الانتقال وتحديد المواد، الزمان والمكان من الشخص (Diringer ١٩٦٨ : ٤٤١). ويوصف نمط معين من الكتابة بأنه نصي، أو نسخ أبجدي (فريزر و Kwiatkowski ٢٠٠٦ ؛ جونستون ١٩٠٩ : بلايت ٦).

يتراوح الخط الحديث من كتابات وتصاميم خط اليد الوظيفى إلى القطع الفنية حيث التعبير المجرد للعلامات المكتوبة بخط اليد التي قد تحل أو لا تحل محل قراءة الحروف (Mediavilla ١٩٩٦). يختلف الخط العربي الكلاسيكي عن الطباعة وعن خط اليد الغير تقليدى، وإن كان الخطاط قد يبتكر كل هؤلاء ؛ وهي تاريخيا حروف منظمة ومع ذلك سلسة وتلقائية ومرتجلة في لحظة الكتابة (بوت ٢٠٠٦ و٢٠٠٥ ؛ Zapf ٢٠٠٧ و٢٠٠٦).

و يستمر الخط العربي في الأزدهار في أشكال دعوات الزواج والأحداث، وتصميم بنط / الطباعة، وتصميم الشعار باستخدام خط النسخ، والفن الديني، والإعلانات / التصميم الجرافيكي / وفن الخط، والنقش على القطع الحجرية ووثيقة العزاء. ويستخدم أيضا من أجل الأثاث والصور المتحركة للسينما والتلفزيون، والشهادة في القضايا، وشهادات الميلاد والوفاة، والخرائط، وغيرها من الأعمال التي تنطوي على الكتابة.


الأدوات

الأداة الرئيسية في فن الخط هو القلم، مشطوف الرأس والمدبب و الريشة. وهناك المحبره عبارة عن حبر زيتي القوام يستخدم في الكتابة. وكذلك من ادوات الخط الورق عالي الجودة.



الفنون البصرية

الفنون البصرية هي الأشكال الفنية التي تركز على إنشاء الأعمال التي هي في المقام الأول مرئية في الطبيعة، مثل الفنون التشكيلية التقليدية (الخزف ،الرسم ،الرسم الفني ،النحت،العمارة، والطباعة)، الفنون البصرية الحديثة (التصوير الفوتوغرافي ،الفيديو وصناعة الأفلام)، والتصميم والحرف اليدوية. العديد من التخصصات الفنية (الفنون المسرحية، فنون اللغة، فنون النسيج وفن الطهي) تنطوي على جوانب من الفنون البصرية، بالإضافة إلى أنواع أخرى، وبالتالي فإن هذه التعريفات ليست دقيقة.

والمفهوم المتغير. الاستخدام الحالي لمصطلح "الفنون البصرية" يتضمن الفنون الجميلة بالإضافة إلى الحرف اليدوية، ولكن هذا ليس هو الحال دائما. قبل حركة الفنون والحرف اليدوية في بريطانيا وأماكن أخرى في مطلع القرن 20th، "الفنان التشكيلي" يعود على الشخص الذي يعمل في مجالالفنون الجميلة (مثل الرسم، النحت، أو الطباعة) وليس مجال الاشغال اليدوية والحرفية، أو من يطبق ضوابط الفن. وأكد التمييز من قبل فنانين من مجال الفنون والحرف الحركية الذين يقدرون أشكال الفن العامية بقدر الأشكال الامعة. الحركة تتناقض مع المتحررون الذين سعوا لحجب الفنون العالية عن الجماهير من خلال ابقائها مقصورة على فئة معينة من الناس. [بحاجة لمصدر] مدرسة الفن تفرق بين الفنون الجميلة والحرف اليدوية في مثل هذه الطريقة التي لا يمكن اعتبار الحرفيين ممارسون للفن.


فن الكمبيوتر

الفنانين البصريين لم يعد يقتصر عمله على وسائل الفن التقليدي. أجهزة الكمبيوتر قد تؤدي إلى تعزيز الفنون البصرية من سهولة \{0الأداء\{/0\} أو قالب:Dn، للتحرير، لاستكشاف مكونات متعددة، للطباعة (بما في ذلك الطباعة ثلاثية الأبعاد.

فن الكمبيوتر هو أي فن يكون للكمبيوتر دورا من إنتاج أو عرض العمل الفني. هذا الفن يمكن أن يكون صورة، صوت، صور متحركة ،فيديو، أقراص مضغوطة، دي في دي، لعبة فيديو، موقع، خوارزمية، أداء، أو تثبيت معرض رسوم. العديد من التخصصات التقليدية الآن تقوم بدمج التقنيات الرقمية و، نتيجة لذلك، الخطوط بين العمل التقليدي للفن وعمل الوسائل الجديدة التي تم إنشاؤها باستخدام أجهزة الكمبيوتر كانت غير واضحة. على سبيل المثال، الفنان قد يجمع بين الرسم التقليدي مع الفن الحسابي وغيرها من التقنيات الرقمية. نتيجة لذلك، تحديد فن الكمبيوتر بواسطة منتجها النهائي يمكن أن يكون صعبا. ومع ذلك، هذا النوع من الفن قد بدأ بالظهور في المتاحف والمعارض الفنية، على الرغم من أنه لم تثبت شرعيتها بعد بوصفها شكلا قائما بذاته وهذه التكنولوجيا على نطاق واسع في الفن المعاصر تعتبر كأداة أكثر من أنها شكلا من أشكال الفن كما هو الحال مع التصوير الزيتي.

استخدام الكمبيوتر حجب الفرق بين الرسامون، المصورون، محررين الصور، النماذج ثلاثية الأبعاد، وفنانو الحرف اليدوية. برامج التحرير والتقديم المتطورة أدت إلى تعدد مهارات مطورين الصور. قد يصبح المصورين فنانين رقميين قد يصبح الرسامين صانعو رسوم متحركة. الحرف اليدوية قد تصبح بمساعدة الحاسوب أو استخدام الكمبيوتر لتوليد الصور كقالب. استخدام مقاطع الكمبيوتر الفنية جعلت التمييز الواضح بين الفنون البصرية ومخطط الصفحة أقل وضوحا بفضل سهولة الوصول وتحرير المقاطع الفنية في عملية ترقيم صفحات الوثيقة، وخاصة للمراقبين الذين يفتقرون للبراعة.



آرٹ

آرٹ ایسے کم یا کماں نوں آکھیا جاندا اے جیدے وچ چیزاں نوں نشاناں نال اینج دسیا جاندا اے جے اوہ سوہنیاں لگن۔ ارٹ دی بنی شی کوئی کہانی دسدی اے یا صرف کوئی سوہنا سچ یا مسوس ہون والا جذبہ۔ انسائیکلوپیڈیا بریٹانیکا آرٹ نوں اینج دسدا اے،" اوہ سوہنیاں شیواں، محول، یا تجربیاں جنہاں نوں دوجیاں نوں وی دسیا جا سکے تے جنہاں دے بنان وچ سوچ تے گر نوں ورتیا گیا ہوۓ"۔ ایتھے آرٹ وچ مورتاں (Painting)، بتی بنانا(Sculpture)، تے فوٹوگرافی آندے نیں۔ آرکیٹیکچر نوں وی ویکھے جان والے آرٹ وج گنیا جاندا اے پر اوہدے وچ اوس شے تے کم لینا اوہدے بنان دا مڈھ ہوندا اے۔ موسیقی، تھیٹر، فلم نوں آرٹ نال رلایا جاسکدا اے۔

آرٹ فلاسفی دی اک ونڈ اے جیدے وچ ایدی اصل ایدا بنانا تے ایہنوں دسنا آندے نیں۔ آرٹ اک ول یا راہ اے انسانی سوچ یا جذبے نوں دسن یا کسے ہور تک اپڑان لئی۔ ارسطو دی فلاسفی وچ آرٹ اک ودیا تے آئیڈیل نقل (Mimesis) اے۔ لیو ٹالسٹائی ایہنوں اک بندے توں دوجے بندے تک گل اپڑان دا پٹھا ول کیندا اے۔ مارٹن ہائیڈگر دی اکھ وچ آرٹ اک ایسا ول اے جیدے راہیں کوئی سنگت اپنا آپ دسدی اے۔



تریخ

وینس ولنڈورف

آرٹ دے سب توں پرانے نمونے پتھر ویلے توں لبدے نیں وینس ولنڈورف 24,000 توں 22,000 م پ دی اک زنانی دی پتھر دی بنی مورتی اے جیہڑی آسٹریا توں لبی سی۔ لاسکو، فرانس وچ زمین دے اندر غاراں وچوں 17,300 ورے پرانیاں جانوراں تے انساناں دیاں مورتاں کنداں تے بنیاں ملیاں نیں۔ ایہناں وچ انسان شکار کردا دسیا گیا اے۔ وشکارلہ پتھر ویلہ 8000-6000 م پ لنگان تے نیولتھک ویلے 6000-3000 م پ وچ انسان نے وائی بیجی تے اک تھاں تے ٹک کے رہنا سکھیا، رہتلاں گنجلیاں ہویاں تے مذہب نے وی مانتا پائی تے انسان نے اپنے فیدے لئی ہتھ نال کئی چیزاں بنانیاں سکھیاں تے فیر کانسی ویلے 3000-1000 م پ وچ پہلیاں انسانی رہتلاں پونگریاں۔ پرانے مصر، پرانے عراق، فارس، ہڑپہ رہتل، چین، پرانے یونان، روم تے انکا تے مایا دیاں رہتلاں نے اپنیاں آرٹ دیاں ریتاں اپنیاں ضرورتاں تے اپنے کول ہون والیاں شیواں توں بنایاں۔ آرٹ اودوں لکھائی دے بنن نال ٹردا اے۔ لکھائی اپنے مڈھ ویلے توں تے بپار دے کم نوں سوکھا تے لین دین نوں یاد رکھن لئی بنی۔ لکھائی پہلے مورتاں نال تے فیر وازاں نوں نشان دے کے تے اوہناں نوں جوڑ کے بنائی جاندی اے۔


پرانا عراق

پرانے عراق دا آرٹ دجلہ تے فرات دے دریاواں دے وشکار 6 ہزار ورے پہلے توں اوتھے پونگرن والیاں اشوری، سمیری، کالدی، رہتلاں وچ بنیا۔ ایتھوں دے آرکیٹیکچر وچ اٹ تے ڈاٹ دا ورتن سی تے ایدے نال زکورت، مندر تے وکھریاں پدھراں والے ہرم بناۓ گۓ۔ محلاں وچ وکھریاں پدھراں تے سوہنے رکھ لاۓ گۓ جیہڑے پرانی دنیا دیاں 7 نویکلیاں شیواں وچوں سن۔ بتی بنان دا ول لکڑ نوں چھل کے تے نقشی دبکی پتھر نال بنیا تے ایہناں نال مذہبی، شکاری تے فوجی وکھالے انساناں تے جانوراں نال بناۓ جاندے سن۔ سمیری ویلے وچ نکیاں بتیاں جناں دے کںارے تکھے، رنگلے پتھر دیاں، گنجے سر تے چھاتیاں تے ہتھ سن۔ اکادی ویلے دیاں بتیاں دے سر دے وال لمے تے داڑیاں سن جیویں نارام سین۔ عموری ویلے دیاں بتیاں نوں لاگاش گڈیا وچ سمجیا جاسکدا اے جیدے چ گڈیا نے چادر سر تے ٹوپی تے ہتھ چھاتی تے رکھے نیں۔ اشتر بوآ جیہڑا بابل وچ 575 م پ وچ بنایا گیا سی بابل دے ارکیٹیکچر ارٹ دی وڈی نشانی اے تے ایدا کج حصہ پرگامون میوزیم، برلن وچ ہے۔

گیونل شیرنی 5000 ورے پرانی لائم پتھر دی بنی اک مورتی اے۔ عراق وچ ایہ اودوں بنی جدوں پہیہ تے لکھن دا ول بنۓ گۓ۔ دکھنی عراق وچ ارک وچ گول پیپ ورگیاں مہراں تے بنیاں مورتاں کہانی سناندا اک نویکلا ول سی۔ ایہ مہراں پتھر مٹی یا کسے ہور شے دیاں وی بنیا ہوسکدیاں سن۔ رات دی رانی عراق چوں لبی کوئی 3800 ورے پرانی پتھر تے بنی عورت دیوی دی بتی اے جیدے پیر پنجھی ورگے دوالے پر، سجے کھبے شیر تے الو نیں ۔ گڈیا دی بتی 2144 - 2124 م پ ورے پرانی اے۔ ار دا سٹینڈرڈ، کھڑی بکری، تانبہ بلد تے گدھ دی یادگار پرانے عراق دے آرٹ دیاں منیاں پرمنیاں نشانیاں نیں۔

لکھائی دا ول بنن نال پرانے عراقی لکھتاں وچ وی اپنے آپ نوں دسدے نیں۔ گلگامش 17 صدی م پ وچ لکھی گئی ایدے وچ سمیری تے اکادی دیوتاواں دیاں کہانیاں نیں۔ اٹراہسس، اکادی ویلے دی اک لکھت اے جیدے وچ ہڑ آن دی کہانی دسی گئی اے۔ بابل دیاں لکھتاں وچ سب توں منی پرمنی لکھت انوما الیش اے جیدے وچ ایس کائنات دے بنن دی کہانی دتی گئی اے۔


پرانا مصر

ملکہ نیفریتیتی

پرانا مصری آرٹ 5000 م پ توں لے کے 300 تک دے ویلے تک پھیلی ہوئی مصری رہتل دی نشانی اے تے ایھدے وچ مورتاں، بتیاں، آرکیٹیکچر تے رلدیاں چیزاں آندیاں نیں ۔ ایہ مصری آرٹ پرانیاں قبراں تے یادگاری تھانواں توں لبیا تے اینج ایدھے وچ مرن مگروں جیون تے پرانی جانکاری نوں بچان تے زور اے۔ ایہ آرٹ 3000 م پ توں لے کے تیجی صدی تک تے اچیچے دوسرے تے تیسرے ٹبر ویلے وچ بنیا۔ 3000 ورے دے لمے ویلے وچ ایدا تے باہر دا کوئی اثر ناں پیا تے جس ودیا ول نال ایہ ٹریا تے بنایا گیا انت تک انج دا ای ریا۔ ایس ویلے دا منیا آرٹسٹ پیبلو پکاسو مصری آرٹ توں اثر لیندا اے۔

مصر اک سکا دیس ہون باجوں ایتھے بنیاں مورتاں ٹھیک ملیاں نیں۔ مصر دیاں بتیاں وچوں سب توں چوکھیاں جانیاں جان والیاں رامسیس II، اخناتون، نیفریتیتی، اوسکورن II، امنہوٹپ III، منکار تے تتنخامون نیں۔مصری پانڈے پتھر توں لے کے مٹی نال بندے سن تے اوہنے تے کالے یا رنگلے نقش بنے ہوندے سن۔ ایہناں وچ عام کر وچ ورتن دے پانڈے مرتبان، کڑے نیں۔

مصری مندراں, محلاں, ہرماں, قبراں, تے ہور کوٹھیاں نوں بنان وچ اٹ، لائم پتھر، ریتلا پتھر تے گرینائٹ ورتے گے۔ لکڑ تھوڑی ہون باجوں ایدا ورتن تھوڑا اے۔ خوفو دا اہرام، ابوالہول، کرناک، لکسر، ادفو دے مندر ایدے وڈے ادھارن نیں۔ مصری لکھتاں تے مورتاں بنان لئی پیپائرس ورتدے سن۔ ایہ اک بوٹے توں بنایا جاندا سی۔ ایدیاں نشانیاں مصر دے سکے موسم وچ رہ گیا ہور کلاسیکی دنیا وچ وی ایدا ورتن سی پر پر کتے ہور ناں رہیاں۔


پرانا یونان

پرانا یونانی آرٹ بتیاں، آرکیٹیکچر، پانڈے تے مورتاں وچ دسد اے تے ایس نے یورپ تے ایشیاء وچ پاکستان تک اپنے رنگ چھڈے نیں۔ یونانی پانڈے وکھریاں ویلیاں وچ وکھریاں ولان تے نمونیاں وچ بندے سن۔ پانڈے جناں وچ مرتبان تے پلیٹاں تے عام زندگی دیاں مورتاں بنیاں ہوندیاں سن۔ یونانی آرکیٹیکچر اپنی شان، سادگی تے سوہنپ وچ اپنی ادھارن آپ اے۔ پرگامون، پارتھینن، ہیفسٹس دا مندر ویکھن والیاں تھانوان نیں۔ مائلو دی وینس، سیموتھریس دی نائیکے، یونانی بتیاں دیاں ودیا ادھارن نیں۔


دکھنی ایشیاء

دکھنی ایشاء وج آرٹ دی ریت ہڑپہ رہتل توں ٹردی ہوئی اج دے ویلے تک اپڑدی اے۔ بانڈے بتیاں آرکیٹیکچر، مورتاں تے ایہ آرٹ ویکھیا جاسکدا اے۔ مہنجوداڑو دی نچدی کڑی، پنڈت بادشاہ، مہراں جنہان تے ہاتھی بلد تے سواستیکا بنے ہوۓ سن، تے مٹی دے وڈے مرتباناں تے پانڈیاں تے مورتاں اوس ویلے دے آرٹ دا وکھالہ نیں۔بت بنانا ہندو جین تے بدھ مت دی ریت رئی اے۔ اجنتا دے غار پرانی ہندستانی بتی دی اک ودیا تھان اے۔ ہندو ہزاراں وریاں توں اپنے دیوی دیوتاواں وشنو،شیوا ،لکشمی ،گنیش ،ہنومان ،اندرا،رام تے سرسوتی دیاں مورتاں تے بتیاں بنا کے مندراں تے کراں وچ رکھدے آۓ نیں۔ اتلے پنجاب وچ گندھارہ بدھ مت دے ارکیٹیکچر دی وڈی تھاں سی۔ ایہ بتیاں بدھ تے اودے جیون دے دوالے نیں تے ویلے نال ایہناں دے بنان وچ تبدیلیاں آئیآں تے یونانی رنگ وی دسدا اے۔ مغل راج ویلے ایرانی رنگ موراں وج دسدا اے۔ مغل راج ویلے دیاں مورتاں مغل بادشاہواں تے اوہناں دے جیون دے دوالے ای بنیاں نیں۔ کانگڑہ راجپوت تے پنجاب مورتاں بنان دے مغلاں مگروں وکھرے سکول سن۔ مورتاں دے ول نوں انگریزاں نے آکے اپنا رنگ وی چاڑیا۔ قطب مینار قطبادین ایبک دی جت تے دلی وج بنایا گیا اک مینار اے۔ارکیٹیکچر وچ پرانے قلعیاں مندراں دے مگروں مغل ویلے وچ مغل آرکیٹیکچر ہندستان دی پہچان بندا اے۔ تاج محل، بادشاہی مسجد، شالامار باغ، شاہی قلعہ مغل آرکیٹیکچر دے کج نشان نیں۔


اسلامی خطاطی

اسلامی خطاطی : یہ ایک فن ہے، جو لکھنے سے تعلق رکھتا ہے۔ بالخصوص عربی حروف، خوبصورت انداز میں لکھنے کا فن۔ [1] اس فن کو اسلام سے جوڑدیاگیا، اس کی وجہ عربی زبان میں یہ فن مقبول ہونا ہے۔ اسلامی سماج میں قرآن کی اہمیت اور اس کی تحریر اور تحفظ، اس فن کو فروغ ہونے میں کامیابی ملی۔ [2]

یہ فن عربی زبان ہی میں نہیں بلکہ، فارسی، ترکی، اردو، آذری زبانوں میں بھی عام ہے۔



خطاطی، بحیثیت اسلامی فن، کافی سراہی گئی۔ مذہب اسلام کے ماننے والے دنیا کے مختلف ممالک میں ہیں، اور مختلف زبانیں بولتے ہیں۔ یہ فن، مسلمانوں کی مختلف زبانوں کو جوڑنے کا کام کیا ہے۔ عربی، فارسی، اردو، سندھی، آذیری، ترکی وغیرہ زبانوں میں بھی یہ فن مقبول ہوا۔ قرآن کی کتابت میں یہ فن بہت ہی کار آمد رہا۔ چھاپہ مشین ایجاد ہونے سے پہلے، کاتب حضرات ہی ان زبانوں میں کتابت اور اشاعتوں کا بیڑا اٹھایا۔ عثمانیہ دور میں ‘دیوانی‘ خط عروج میں آیا۔ دیوانی خط کا ایک روپ ‘جلی دیوانی‘ یا ‘دیوانی الجالی‘ ہے۔ رفتہ رفتہ، ‘رقع‘ خط کا ارتقاء ہوا۔

اشیاء خطاطی[ترمیم]

رواجی طور پر خطاطی کے لیے، قلم، دوات اور رنگ۔ اہم ہیں۔

• قلم : خطاطی کے لیے مخصوص قلم بنایے جاتے ہیں۔ “برو“ یا “بانس“ یا “بمبو“ کے قلم مشہور ہیں۔ قلم کی نوک، حرفوں کی چوڑائی کے مطابق بنایے جاتے ہیں۔ حروف جتنے چوڑے، قلم کی نوک اتنی چوڑی۔

• دوات : دوات، شئی، اور ہمہ قسم کے رنگ بھی استعمال ہوتے ہیں۔

خطاطی سکھانے کے کئی مراکز بھی ہیں۔ شہر حیدرآباد دکن میں، اردو اکیڈمی، انجمن ترقی اردو، عابد علی خان ٹرسٹ، آج بھی خطاطی میں ٹریننگ دے رہے ہیں۔


مساجد اور خطاطی[ترمیم]

اسلامی “مسجد خطاطی“ یہ بھی اسی فن کا ایک جز ہے۔ مسجدوں کی، میناروں کی تعمیرات میں اس فن کو دیکھا جاسکتا ہے۔ مساجد میں قرآنی آیات، میناروں پر گنبدوں پر، اندرون گنبد پر قرآنی کلمات کا لکھا جانا فن تعمیر-فن خطاطی کا ایک مرکب ہے۔


عبارت بسم اللہ الرحمن الرحیم، اور اللہ - محمد صلی اللہ علیہ و آلہ وسلم ، مساجد میں لکھنا ایک عام رسم ہے۔ ان کے علاوہ، قرآنی آیات کا لکھا جانا بھی عام ہے


عربی زبان

عربی (العربية al-ʻarabīyah یا عربي/لغة عربية ʻarabī )، سامی زبانوں میں سب سے بڑی ہے اور عبرانی اور آرامی زبانوں سے بہت ملتی ہے۔ جدید عربی کلاسیکی عربی زبان( فصیح عربی یا اللغة العربية الفصحى ) کی تھوڑی سی بدلی ہوئی شکل ہے۔ فصیح عربی قدیم زمانے سے ہی بہت ترقی یافتہ شکل میں تھی اور قرآن کی زبان ہونے کی وجہ سے زندہ ہے۔ فصیح عربی اور بولے جانے والی عربی میں بہت فرق نہیں بلکہ ایسے ہی ہے جیسے بولے جانے والی اردو اور ادبی اردو میں فرق ہے۔ عربی زبان نے اسلام کی ترقی کی وجہ سے مسلمانوں کی دوسری زبانوں مثلاً اردو، فارسی، ترکی وغیرہ پر بڑا اثر ڈالا ہے اور ان زبانوں میں عربی کے بے شمار الفاظ موجود ہیں۔ عربی کو مسلمانوں کی مذہبی زبان کی حیثیت حاصل ہے اور تمام دنیا کے مسلمان قرآن پڑھنے کی وجہ سے عربی حروف اور الفاظ سے مانوس ہیں۔ عربی کے کئی لہجے آج کل پائے جاتے ہیں مثلاً مصری، شامی، عراقی، حجازی وغیرہ۔ مگر تمام لہجے میں بولنے والے ایک دوسرے کی بات بخوبی سمجھ سکتے ہیں اور لہجے کے علاوہ فرق نسبتاً معمولی ہے۔ یہ دائیں سے بائیں لکھی جاتی ہے اور اس میں ھمزہ سمیت 29 حروف تہجی ہیں جنہیں حروف ابجد کہا جاتا ہے۔ (Arabic language)



حروف تہجی

حروف تہجی سے مراد وہ علامتیں ہیں جنہیں لکھنے کے لیے استعمال کیا جاتا ہے۔ اکثر زبانوں میں یہ مختلف آوازوں کی علامات ہیں مگر کچھ زبانوں میں یہ مختلف تصاویر کی صورت میں بھی استعمال کیے جاتے ہیں۔ بعض ماہرینِ لسانیات کے خیال میں سب سے پہلے سامی النسل یا فنیقی لوگوں نے حروف کو استعمال کیا۔ بعد میں عربی کی شکل میں حروف نے وہ شکل پائی جو اردو میں استعمال ہوتے ہیں۔ بعض ماہرین ان کی ابتداء کو قدیم مصر سے جوڑتے ہیں۔ اردو میں جو حروف استعمال ہوتے ہیں وہ عربی سے لیے گئے ہیں۔ جنہیں حروف ابجد بھی کہا جاتا ہے۔ ان کی پرانی عربی ترتیب کچھ یوں ہے۔ ابجد ھوز حطی کلمن سعفص قرشت ثخذ ضظغ۔ انہیں حروفِ ابجد بھی کہتے ہیں۔ علمِ جفر میں ان کے ساتھ کچھ اعداد کو بھی منسلک کیا جاتا ہے۔ جو کچھ یوں ہیں۔


اس طریقہ سے اگر بسم اللہ الرحمٰن الرحیم کے اعداد شمار کیے جائیں تو786 بنتے ہیں۔ بعض اھلِ علم ھمزہ (' ء)' کو بھی ایک الگ حرف مانتے ہیں۔ اس حساب سے پھر عربی کے انتیس حروف تہجی بنتے ہیں۔ اردو زبان میں عربی کے ان حروف کے علاوہ مزید حروف استعمال ہوتے ہیں۔ اردو میں ہمزہ سمیت کل 37 حروف ہیں۔ اس کے علاوہ کچھ آوازیں حروف کے مجموعے سے بھی ظاہر کی جاتی ہیں مثلاً کھ، بھ وغیرہ۔ اردو میں فن تاریخ گوئی کی صنف موجود ہے جس میں اشعار یا ایک مصرع سے کسی واقعہ کی تاریخ برامد کی جاتی ہے۔ اردو میں حروف کے اعداد حروفِ ابجد جیسے ہی ہیں۔ اضافی حروف کے لیے بھی انہی سے مدد لی جاتی ہے۔ مثلاً 'پ' کے اعداد 'ب' کے برابر، 'ڈ' کے اعداد 'د' کے برابر، 'گ' کے اعداد 'ک' کے برابر اور 'ے' کے اعداد 'ی' کے برابر شمار کیے جاتے ہیں۔



\end{flushright}
\end{document}
